\begin{question}{1}{lógica, proposiciones}{1}{a}{3}{
\textbf{Suponga que:} 
\\
\(p = \) ``Llovió ayer por la noche".\\
\(q = \) ``Se prendieron los rociadores de pasto ayer por la noche".\\
\(r = \) ``El pasto estaba mojado hoy por la mañana".

\textbf{Traduzca al español la siguiente proposición:}
\[
(\neg p)
\]

\begin{enumerate}
    \item a) No llovió ayer por la noche.
    \item b) Se prendieron los rociadores de pasto ayer por la noche.
    \item c) Llovió ayer por la noche.
    \item d) El pasto estaba mojado hoy por la mañana.
\end{enumerate}
}
\end{question}

% ==================================================================
% PREGUNTA 2
% (Tema: lógica, proposiciones, dif:1, res:b, week:3)
% ==================================================================
\begin{question}{2}{lógica, proposiciones}{1}{b}{3}{
\textbf{Suponga que:} 
\\
\(p = \) ``Llovió ayer por la noche".\\
\(q = \) ``Se prendieron los rociadores de pasto ayer por la noche".\\
\(r = \) ``El pasto estaba mojado hoy por la mañana".\medskip

\textbf{Traduzca al español la siguiente proposición:}
\[
(r \land (\neg p))
\]

\begin{enumerate}
    \item a) El pasto estaba mojado hoy por la mañana o no llovió ayer por la noche.
    \item b) El pasto estaba mojado hoy por la mañana y no llovió ayer por la noche.
    \item c) No se prendieron los rociadores de pasto ayer por la noche y llovió ayer por la noche.
    \item d) El pasto no estaba mojado hoy por la mañana, pero llovió ayer por la noche.
\end{enumerate}
}
\end{question}

% ==================================================================
% PREGUNTA 3
% (Tema: lógica, proposiciones, dif:1, res:a, week:3)
% ==================================================================
\begin{question}{3}{lógica, proposiciones}{1}{a}{3}{
\textbf{Suponga que:} 
\\
\(p = \) ``Llovió ayer por la noche".\\
\(q = \) ``Se prendieron los rociadores de pasto ayer por la noche".\\
\(r = \) ``El pasto estaba mojado hoy por la mañana".\medskip

\textbf{Traduzca al español la siguiente proposición:}
\[
((p \land q) \lor (\neg p))
\]

\begin{enumerate}
   \item a) Llovió ayer por la noche y se prendieron los rociadores de pasto ayer por la noche, o no llovió ayer por la noche.
   \item b) El pasto estaba mojado hoy por la mañana y llovió ayer por la noche, o no se prendieron los rociadores de pasto ayer por la noche.
   \item c) No llovió ayer por la noche y se prendieron los rociadores de pasto ayer por la noche, o el pasto estaba mojado hoy por la mañana.
   \item d) Llovió ayer por la noche o el pasto estaba mojado hoy por la mañana, pero no se prendieron los rociadores de pasto ayer por la noche.
\end{enumerate}
}
\end{question}

% ==================================================================
% PREGUNTA 4
% (Tema: lógica, proposiciones, dif:2, res:c, week:3)
% ==================================================================
\begin{question}{4}{lógica, proposiciones}{2}{c}{3}{
\textbf{Suponga que:} 
\\
\(p = \) ``Llovió ayer por la noche".\\
\(q = \) ``Se prendieron los rociadores de pasto ayer por la noche".\\
\(r = \) ``El pasto estaba mojado hoy por la mañana".\medskip

\textbf{Traduzca al español la siguiente proposición:}
\[
(((\neg p) \land q) \lor (p \land (\neg q))
\]

\begin{enumerate}
   \item a) El pasto estaba mojado hoy por la mañana y llovió ayer por la noche, o se prendieron los rociadores de pasto ayer por la noche.
   \item b) Llovió ayer por la noche y el pasto estaba mojado hoy por la mañana, o no se prendieron los rociadores de pasto ayer por la noche.
   \item c) No llovió ayer por la noche y se prendieron los rociadores de pasto ayer por la noche, o llovió ayer por la noche y no se prendieron los rociadores de pasto ayer por la noche.
   \item d) No llovió ayer por la noche y el pasto estaba mojado hoy por la mañana, o los rociadores no se prendieron.
\end{enumerate}
}
\end{question}

% ==================================================================
% PREGUNTA 5
% (Tema: lógica, proposiciones, dif:2, res:c, week:3)
% ==================================================================
\begin{question}{5}{lógica, proposiciones}{2}{c}{3}{
\textbf{Suponga que:} 
\\
\(p = \) ``Llovió ayer por la noche".\\
\(q = \) ``Se prendieron los rociadores de pasto ayer por la noche".\\
\(r = \) ``El pasto estaba mojado hoy por la mañana".\medskip

\textbf{Traduzca al español la siguiente proposición:}
\[
((p \land (\neg q)) \lor r)
\]

\begin{enumerate}
   \item a) Se prendieron los rociadores de pasto ayer por la noche y llovió ayer por la noche, o el pasto estaba mojado hoy por la mañana.
   \item b) No llovió ayer por la noche y se prendieron los rociadores de pasto ayer por la noche, o el pasto estaba seco hoy por la mañana.
   \item c) Llovió ayer por la noche y no se prendieron los rociadores de pasto ayer por la noche, o el pasto estaba mojado hoy por la mañana.
   \item d) No llovió ayer por la noche y el pasto estaba mojado hoy por la mañana, pero los rociadores no se prendieron.
\end{enumerate}
}
\end{question}

% ==================================================================
% PREGUNTA 6
% (Tema: lógica, proposiciones, dif:2, res:a, week:3)
% ==================================================================
\begin{question}{6}{lógica, proposiciones}{2}{a}{3}{
\textbf{Suponga que:} 
\\
\(p = \) ``Llovió ayer por la noche".\\
\(q = \) ``Se prendieron los rociadores de pasto ayer por la noche".\\
\(r = \) ``El pasto estaba mojado hoy por la mañana".\medskip

\textbf{Traduzca al español la siguiente proposición:}
\[
\neg [(p \land (\neg q)) \lor (\neg p)]
\]

\begin{enumerate}
   \item a) No Llovió ayer por la noche y se prendieron los rociadores de pasto ayer por la noche.  
   \item b) Llovió ayer por la noche o los rociadores no se prendieron.  
   \item c) No llovió ayer por la noche y el pasto estaba mojado hoy por la mañana.  
   \item d) El pasto estaba mojado hoy por la mañana y los rociadores se prendieron.  
\end{enumerate}
}
\end{question}

% ==================================================================
% PREGUNTA 7
% (Tema: lógica, proposiciones, dif:2, res:a, week:3)
% ==================================================================
\begin{question}{7}{lógica, proposiciones}{2}{a}{3}{
\textbf{Suponga que:} 
\\
\(p = \) ``Llovió ayer por la noche".\\
\(q = \) ``Se prendieron los rociadores de pasto ayer por la noche".\\
\(r = \) ``El pasto estaba mojado hoy por la mañana".\medskip

\textbf{Traduzca al español la siguiente proposición:}
\[
((\neg p) \land q) \lor ((\neg p) \lor q)
\]

\begin{enumerate}
   \item a) No llovió ayer por la noche y se prendieron los rociadores de pasto ayer por la noche, o no llovió ayer por la noche o se prendieron los rociadores de pasto ayer por la noche.  
   \item b) Llovió ayer por la noche y no se prendieron los rociadores de pasto ayer por la noche, o el pasto estaba mojado hoy por la mañana.  
   \item c) El pasto estaba mojado hoy por la mañana y llovió ayer por la noche, o se prendieron los rociadores de pasto ayer por la noche.  
   \item d) No llovió ayer por la noche y el pasto estaba mojado hoy por la mañana, pero los rociadores no se prendieron.  
\end{enumerate}
}
\end{question}

% ==================================================================
% PREGUNTA 8
% (Tema: lógica, proposiciones, dif:2, res:d, week:3)
% ==================================================================
\begin{question}{8}{lógica, proposiciones}{2}{d}{3}{
\textbf{Suponga que:} 
\\
\(p = \) ``Llovió ayer por la noche".\\
\(q = \) ``Se prendieron los rociadores de pasto ayer por la noche".\\
\(r = \) ``El pasto estaba mojado hoy por la mañana".\medskip

\textbf{Traduzca al español la siguiente proposición:}
\[
\neg [(p \lor q) \land (\neg q)]
\]

\begin{enumerate}
   \item a) No llovió ayer por la noche y el pasto estaba mojado hoy por la mañana. 
   \item b) Llovió ayer por la noche y se prendieron los rociadores de pasto ayer por la noche.  
   \item c) El pasto estaba mojado hoy por la mañana y los rociadores no se prendieron.  
   \item d) No llovió ayer por la noche o se prendieron los rociadores de pasto ayer por la noche. 
\end{enumerate}
}
\end{question}

% ==================================================================
% PREGUNTA 9
% (Tema: lógica, proposiciones, dif:2, res:a, week:3)
% ==================================================================
\begin{question}{9}{lógica, proposiciones}{2}{a}{3}{
\textbf{Suponga que:} 
\\
\(p = \) ``Usted tiene gripa".\\
\(q = \) ``Usted no vino al último examen".\\
\(r = \) ``Usted pasa el curso". \medskip

\textbf{Traduzca al español la siguiente proposición:}
\[
(p \land q)\lor((\neg q) \land r)
\]

\begin{enumerate}
    \item a) Usted tiene gripa y no vino al último examen, o usted sí vino al último examen y pasa el curso.
    \item b) Usted no tiene gripa y no vino al último examen, o usted sí vino al último examen y pasa el curso.
    \item c) Usted tiene gripa y vino al último examen, pero no pasa el curso.
    \item d) Usted no vino al último examen y pasa el curso, pero no tiene gripa.
\end{enumerate}
}
\end{question}

% ==================================================================
% PREGUNTA 10
% (Tema: lógica, proposiciones, dif:2, res:a, week:3)
% ==================================================================
\begin{question}{10}{lógica, proposiciones}{2}{a}{3}{
\textbf{Suponga que:} 
\\
\(p = \) ``Usted tiene gripa".\\
\(q = \) ``Usted no vino al último examen".\\
\(r = \) ``Usted pasa el curso". \medskip

\textbf{Traduzca al español la siguiente proposición:}
\[
(p \lor q) \rightarrow (\neg r
)\]
\\ Tenga en cuenta que $\rightarrow$ corresponde a la implicacion y se puede tomar como un entonces o consecuencia tipo $A\rightarrow B$, sería A implica B o si A entonces B.
\begin{enumerate}
   \item a) Si usted tiene gripa o no vino al último examen, entonces no pasa el curso.  
   \item b) Usted pasa el curso solo si tiene gripa o no vino al último examen.  
   \item c) Si usted pasa el curso, entonces tiene gripa o no vino al último examen.  
   \item d) Usted no pasa el curso o no tiene gripa ni faltó al último examen.  
\end{enumerate}
}
\end{question}

% ==================================================================
% PREGUNTA 11
% (Tema: lógica, proposiciones, dif:2, res:b, week:3)
% ==================================================================
\begin{question}{11}{lógica, proposiciones}{2}{b}{3}{
\textbf{Suponga que:} 
\\
\(p = \) ``Usted tiene gripa".\\
\(q = \) ``Usted no vino al último examen".\\
\(r = \) ``Usted pasa el curso". \medskip

\textbf{Traduzca al español la siguiente proposición:}
\[
((\neg p) \lor r) \land (q \rightarrow p)
\]
\\ Tenga en cuenta que $\rightarrow$ corresponde a la implicacion y se puede tomar como un entonces o consecuencia tipo $A\rightarrow B$, sería A implica B o si A entonces B.

\begin{enumerate}
   \item a) Si usted no tiene gripa o pasa el curso, y si no vino al último examen, entonces tiene gripa.  
   \item b) Usted no tiene gripa o pasa el curso, y si usted no vino al último examen, entonces tenía gripa.  
   \item c) Usted no tiene gripa o pasa el curso, y si no vino al último examen, entonces usted no pasa el curso.  
   \item d) Si usted no tiene gripa y pasa el curso, entonces usted no vino al último examen.  
\end{enumerate}
}
\end{question}

% ==================================================================
% PREGUNTA 12
% (Tema: lógica, proposiciones, dif:2, res:c, week:3)
% ==================================================================
\begin{question}{12}{lógica, proposiciones}{2}{c}{3}{
\textbf{Suponga que:} 
\\
\(p = \) ``Usted tiene gripa".\\
\(q = \) ``Usted no vino al último examen".\\
\(r = \) ``Usted pasa el curso". \medskip

\textbf{Traduzca al español la siguiente proposición:}
\[
\neg [(p \lor q) \land (\neg r)]
\]

\begin{enumerate}
   \item a) Si usted tiene gripa o no vino al último examen, entonces pasa el curso.  
   \item b) No es cierto que usted tenga gripa o no vino al último examen, y no pasa el curso.  
   \item c) Usted pasa el curso o no tiene gripa y vino al último examen.  
   \item d) Usted no tiene gripa ni faltó al último examen, pero pasa el curso.  
\end{enumerate}
}
\end{question}

% ==================================================================
% PREGUNTA 13
% (Tema: lógica, proposiciones, dif:2, res:a, week:3)
% ==================================================================
\begin{question}{13}{lógica, proposiciones}{2}{a}{3}{
\textbf{Suponga que:} 
\\
\(p = \) ``Usted tiene gripa".\\
\(q = \) ``Usted no vino al último examen".\\
\(r = \) ``Usted pasa el curso". \medskip

\textbf{Traduzca al español la siguiente proposición:}
\[
(p \lor q) \leftrightarrow (\neg r)
\]
\\ Tenga en cuenta que $\leftrightarrow$ corresponde al si y solo si y se puede interpretar $A \leftrightarrow B$ como: A ocurre si y solo si B ocurre

\begin{enumerate}
   \item a) Usted tiene gripa o no vino al último examen si y solo si no pasa el curso.  
   \item b) Si usted no pasa el curso, entonces tuvo gripa o no vino al último examen.  
   \item c) Si usted tiene gripa o no vino al último examen, entonces no pasa el curso.  
   \item d) Si usted pasa el curso, entonces tuvo gripa o no vino al último examen.  
\end{enumerate}
}
\end{question}

% ==================================================================
% PREGUNTA 14
% (Tema: lógica, proposiciones, dif:3, res:d, week:3)
% ==================================================================
\begin{question}{14}{lógica, proposiciones}{3}{d}{3}{
\textbf{Suponga que:} 
\\
\(p = \) ``Usted tiene gripa".\\
\(q = \) ``Usted no vino al último examen".\\
\(r = \) ``Usted pasa el curso". \medskip

\textbf{Traduzca al español la siguiente proposición:}
\[
[(p \lor (\neg q)) \rightarrow (\neg r)] \lor [(r \land p) \rightarrow q]
\]
\\ Tenga en cuenta que $\rightarrow$ corresponde a la implicacion y se puede tomar como un entonces o consecuencia tipo $A\rightarrow B$, sería A implica B o si A entonces B.

\begin{enumerate}
   \item a) Si usted pasa el curso o no tuvo gripa, entonces vino al último examen; o si usted pasa el curso y tuvo gripa, entonces faltó al último examen.  
   \item b) Si usted tiene gripa o vino al último examen, entonces pasa el curso; o si usted pasa el curso y tiene gripa, entonces no faltó al último examen.  
   \item c) Si usted no pasa el curso, entonces tuvo gripa o faltó al último examen; o si usted pasa el curso y tuvo gripa, entonces no vino al último examen.  
   \item d) Si usted tiene gripa o no faltó al último examen, entonces no pasa el curso; o si usted pasa el curso y tiene gripa, entonces no faltó al último examen.
\end{enumerate}
}
\end{question}

% ==================================================================
% PREGUNTA 15
% (Tema: conjuntos, operaciones de conjuntos, dif:1, res:d, week:2)
% ==================================================================
\begin{question}{15}{conjuntos}{1}{d}{2}{
\\Para celebrar San Valentín, el CONSEFE organizó CUPICONSEFE, un domicilio de regalos entre los estudiantes, los administrativos y los profesores de la Facultad. Definimos como el universo a los estudiantes, los administrativos y los profesores de la Facultad de Economía. \textbf{Considere los siguientes conjuntos:}
\[
\begin{aligned}
R &= \{x \in U \mid x \text{ recibió un regalo por CUPICONSEFE}\},\\
E &= \{x \in U \mid x \text{ envió un regalo por CUPICONSEFE}\},\\
Est &= \{x \in U \mid x \text{ es estudiante}\},\\
Prof &= \{x \in U \mid x \text{ es profesor}\},\\
Adm &= \{x \in U \mid x \text{ es administrativo}\}.
\end{aligned}
\]

\textbf{Traduzca al español la siguiente proposición:}
\[
R \cap E \neq \emptyset
\]

\begin{enumerate}
    \item a) No hay personas que hayan recibido y enviado regalos al mismo tiempo.
    \item b) Al menos una persona que recibió un regalo no lo envió.
    \item c) Todas las personas enviaron y recibieron regalos.
    \item d) Al menos una persona envió y recibió un regalo.
\end{enumerate}
}
\end{question}
% ==================================================================
% PREGUNTA 16
% (Tema: conjuntos, operaciones de conjuntos, dif:3, res:b, week:2)
% ==================================================================
\begin{question}{16}{conjuntos}{3}{b}{2}{
\\Para celebrar San Valentín, el CONSEFE organizó CUPICONSEFE, un domicilio de regalos entre los estudiantes, los administrativos y los profesores de la Facultad. Definimos como el universo a los estudiantes, los administrativos y los profesores de la Facultad de Economía. \textbf{Considere los siguientes conjuntos:}
\[
\begin{aligned}
R &= \{x \in U \mid x \text{ recibió un regalo por CUPICONSEFE}\},\\
E &= \{x \in U \mid x \text{ envió un regalo por CUPICONSEFE}\},\\
Est &= \{x \in U \mid x \text{ es estudiante}\},\\
Prof &= \{x \in U \mid x \text{ es profesor}\},\\
Adm &= \{x \in U \mid x \text{ es administrativo}\}.
\end{aligned}
\]

\textbf{Traduzca al español la siguiente proposición:}
\[
\neg \left( Adm \subseteq \left[(R \setminus E) \cup (E \setminus R)\right] \right)
\]

\begin{enumerate}
    \item a) Ningún administrativo recibió o envió un regalo, pero no ambos.
    \item b) Al menos un administrativo envió y recibió un regalo, o no envió ni recibió ninguno.
    \item c) Todos los administrativos recibieron y enviaron un regalo.
    \item d) Ningún administrativo recibió y envió un regalo al mismo tiempo.
\end{enumerate}
}
\end{question}

% ==================================================================
% PREGUNTA 17
% (Tema: conjuntos, operaciones de conjuntos, dif:2, res:b, week:2)
% ==================================================================
\begin{question}{17}{conjuntos}{2}{b}{2}{
\\Para celebrar San Valentín, el CONSEFE organizó CUPICONSEFE, un domicilio de regalos entre los estudiantes, los administrativos y los profesores de la Facultad. Definimos como el universo a los estudiantes, los administrativos y los profesores de la Facultad de Economía. \textbf{Considere los siguientes conjuntos:}
\[
\begin{aligned}
R &= \{x \in U \mid x \text{ recibió un regalo por CUPICONSEFE}\},\\
E &= \{x \in U \mid x \text{ envió un regalo por CUPICONSEFE}\},\\
Est &= \{x \in U \mid x \text{ es estudiante}\},\\
Prof &= \{x \in U \mid x \text{ es profesor}\},\\
Adm &= \{x \in U \mid x \text{ es administrativo}\}.
\end{aligned}
\]

\textbf{Use únicamente los conjuntos definidos para escribir una expresión equivalente a:} \medskip

\textit{“Sólo los profesores que también son estudiantes recibieron regalos.”} \medskip

\begin{enumerate}
    \item a) \( R = Prof \cap Est \)  
    \item b) \( R \subseteq (Prof \cap Est) \)  
    \item c) \( R = (Prof \cap Est) \cup Adm \)  
    \item d) \( R \subseteq (Prof \cup Est) \)  
\end{enumerate}
}
\end{question}



% ==================================================================
% PREGUNTA 18
% (Tema: conjuntos, operaciones de conjuntos, dif:2, res:a, week:2)
% ==================================================================
\begin{question}{18}{conjuntos}{2}{a}{2}{
\\ \textbf{Considere el conjunto de películas que se presentan en los cines de Bogotá como universo \(U\).}\\
\textbf{Sean:}
\[
\begin{aligned}
CC &= \{x \in U \mid x\text{ se presenta en Cine Colombia}\},\\
CM &= \{x \in U \mid x\text{ se presenta en Cine-Mark}\}.
\end{aligned}
\]
\textbf{Use únicamente los conjuntos definidos para escribir una expresión equivalente a:} \medskip

\textit{“Todas las películas que se presentan en Bogotá, están en Cine Colombia pero no en Cine-Mark, o están en Cine-Mark pero no en Cine Colombia.”} \medskip

\begin{enumerate}
    \item a) \( U = (CC \setminus CM) \cup (CM \setminus CC) \)  
    \item b) \( U = CC \cup CM \)  
    \item c) \( U \subseteq (CC \cap CM) \)  
    \item d) \( U \subseteq (CC \cup CM) \)  
\end{enumerate}
}
\end{question}

% ==================================================================
% PREGUNTA 19
% (Tema: conjuntos, operaciones de conjuntos, dif:2, res:c, week:2)
% ==================================================================
\begin{question}{19}{conjuntos}{2}{c}{2}{
\\ \textbf{Considere el conjunto de películas que se presentan en los cines de Bogotá como universo \(U\).}\\
\textbf{Sean:}
\[
\begin{aligned}
CC &= \{x \in U \mid x\text{ se presenta en Cine Colombia}\},\\
CM &= \{x \in U \mid x\text{ se presenta en Cine-Mark}\}.
\end{aligned}
\]
\textbf{Traduzca al español la siguiente proposición:}
\[
(CC \cup CM)^c = CC^c \cap CM^c
\]

\begin{enumerate}
    \item a) No hay películas en Bogotá fuera de Cine Colombia y Cine-Mark.  
    \item b) Todas las películas en Bogotá están en al menos uno de los dos cines.  
    \item c) Las películas que no están en Cine Colombia y en Cine-Mark son exactamente aquellas que no están en Cine Colombia Y tampoco están en Cine-Mark..  
    \item d) Todas las películas de Bogotá que no están en Cine Colombia tampoco están en Cine-Mark.  
\end{enumerate}
}
\end{question}

% ==================================================================
% PREGUNTA 20
% (Tema: conjuntos, operaciones de conjuntos, dif:2, res:c, week:2)
% ==================================================================
\begin{question}{20}{conjuntos}{2}{c}{2}{
\textbf{Sea U el conjunto de los equipos de la NFL. Considere los siguientes conjuntos:}
\[
\begin{aligned}
AFC &= \{ x \in U \mid x \text{ hace parte de la American Football Conference}\},\\
NFC &= \{ x \in U \mid x \text{ hace parte de la National Football Conference}\},\\
Div &= \{ x \in U \mid x \text{ ganó su división}\},\\
Top_6 &= \{ x \in U \mid x \text{ está en el top 6 de su conferencia}\},\\
Top_2 &= \{ x \in U \mid x \text{ está en el top 2 de su conferencia}\},\\
PlayOFFs &= \{ x \in U \mid x \text{ pasó a los play-offs}\},\\
SB &= \{ x \in U \mid x \text{ juega en el Super Bowl}\}.
\end{aligned}
\]

\textbf{Use únicamente los conjuntos definidos para escribir una expresión equivalente a:} 
\[
\bigl[(x \in Top_6)\ \land\ (x \not\in Div)\bigr]\ \lor\ \bigl(x \in Div\bigr)
\]

\begin{enumerate}
    \item a) \((Top_6 \cap Div) \cup Div\)
    \item b) \((Top_6 \cup Div) \cap Div\)
    \item c) \((Top_6 \setminus Div)\ \cup\ Div\)
    \item d) \((Top_6 \setminus Div)\ \cap\ Div\)
\end{enumerate}
}
\end{question}

% ==================================================================
% PREGUNTA 21
% (Tema: conjuntos, funciones, dif:2, res:a, week:9)
% ==================================================================
\begin{question}{21}{conjuntos, funciones}{2}{a}{9}{
\textbf{Sea U el conjunto de los equipos de la NFL. Considere los siguientes conjuntos:}
\[
\begin{aligned}
AFC &= \{ x \in U \mid x \text{ hace parte de la American Football Conference}\},\\
NFC &= \{ x \in U \mid x \text{ hace parte de la National Football Conference}\},\\
Div &= \{ x \in U \mid x \text{ ganó su división}\},\\
Top_6 &= \{ x \in U \mid x \text{ está en el top 6 de su conferencia}\},\\
Top_2 &= \{ x \in U \mid x \text{ está en el top 2 de su conferencia}\},\\
PlayOFFs &= \{ x \in U \mid x \text{ pasó a los play-offs}\},\\
SB &= \{ x \in U \mid x \text{ juega en el Super Bowl}\}.
\end{aligned}
\]

Para $x \in U$, $d(x)$ denota el turno que le toca al equipo $x$ en el Draft del año siguiente. \medskip

\textbf{Traduzca al español la siguiente proposición:}
\[
(x \in SB \land x \in AFC)\to d(x)>10
\]
\\ Tenga en cuenta que $\rightarrow$ corresponde a la implicacion y se puede tomar como un entonces o consecuencia tipo $A\rightarrow B$, sería A implica B o si A entonces B.
\begin{enumerate}
    \item a) Si un equipo está en el Super Bowl y en la AFC, entonces su turno en el Draft es mayor a 10.  
    \item b) Todos los equipos del Super Bowl tienen un turno en el Draft mayor a 10.  
    \item c) Un equipo de la AFC solo puede estar en el Super Bowl si su turno en el Draft es mayor a 10.  
    \item d) Si un equipo tiene un turno en el Draft mayor a 10, entonces juega en el Super Bowl y es de la AFC.  
\end{enumerate}
}
\end{question}

% ==================================================================
% PREGUNTA 22
% (Tema: conjuntos, índices, dif:2, res:a, week:7)
% ==================================================================
\begin{question}{22}{conjuntos, índices}{2}{a}{7}{
\textbf{Sea U el conjunto de los equipos de la NFL. Considere los siguientes conjuntos:}
\[
\begin{aligned}
AFC &= \{ x \in U \mid x \text{ hace parte de la American Football Conference}\},\\
NFC &= \{ x \in U \mid x \text{ hace parte de la National Football Conference}\},\\
Div &= \{ x \in U \mid x \text{ ganó su división}\},\\
Top_6 &= \{ x \in U \mid x \text{ está en el top 6 de su conferencia}\},\\
Top_2 &= \{ x \in U \mid x \text{ está en el top 2 de su conferencia}\},\\
PlayOFFs &= \{ x \in U \mid x \text{ pasó a los play-offs}\},\\
SB &= \{ x \in U \mid x \text{ juega en el Super Bowl}\}.
\end{aligned}
\]

Para $x \in U$, $d(x)$ denota el turno que le toca al equipo $x$ en el Draft del año siguiente. \medskip

\textbf{Traduzca al español la siguiente proposición:}
\[
(x \in Top_6^c)\land(d(x) \in \{1,2,\dots,20\})
\]

\begin{enumerate}
    \item a) Un equipo que no está en el top 6 de su conferencia tiene un turno en el Draft entre 1 y 20.  
    \item b) Un equipo que está en el top 6 de su conferencia tiene un turno en el Draft entre 1 y 20.  
    \item c) Un equipo que no está en el top 6 de su conferencia tiene un turno en el Draft mayor a 20.  
    \item d) Un equipo con turno en el Draft entre 1 y 20 no está en el top 6 de su conferencia.  
\end{enumerate}
}
\end{question}

% ==================================================================
% PREGUNTA 23
% (Tema: conjuntos, índices, dif:2, res:a, week:7)
% ==================================================================
\begin{question}{23}{conjuntos, índices}{2}{a}{9}{
\textbf{Sea U el conjunto de los equipos de la NFL. Considere los siguientes conjuntos:}
\[
\begin{aligned}
AFC &= \{ x \in U \mid x \text{ hace parte de la American Football Conference}\},\\
NFC &= \{ x \in U \mid x \text{ hace parte de la National Football Conference}\},\\
Div &= \{ x \in U \mid x \text{ ganó su división}\},\\
Top_6 &= \{ x \in U \mid x \text{ está en el top 6 de su conferencia}\},\\
Top_2 &= \{ x \in U \mid x \text{ está en el top 2 de su conferencia}\},\\
PlayOFFs &= \{ x \in U \mid x \text{ pasó a los play-offs}\},\\
SB &= \{ x \in U \mid x \text{ juega en el Super Bowl}\}.
\end{aligned}
\]

Para $x \in U$, $d(x)$ denota el turno que le toca al equipo $x$ en el Draft del año siguiente. \medskip

\textbf{Traduzca al español la siguiente proposición:}
\[
(x \in Div \cap Top_2)\land(d(x) \geq 25)
\]

\begin{enumerate}
    \item a) Un equipo que ganó su división y está en el top 2 de su conferencia tiene un turno en el Draft de al menos 25.  
    \item b) Un equipo que ganó su división y está en el top 2 de su conferencia tiene un turno en el Draft menor a 25.  
    \item c) Si un equipo tiene un turno en el Draft mayor o igual a 25, entonces ganó su división y está en el top 2 de su conferencia.  
    \item d) Todos los equipos en el top 2 de su conferencia tienen un turno en el Draft mayor o igual a 25.  
\end{enumerate}
}
\end{question}

% ==================================================================
% PREGUNTA 24
% (Tema: conjuntos, índices, dif:2, res:a, week:7)
% ==================================================================
\begin{question}{24}{conjuntos, índices}{2}{a}{9}{
\textbf{Sea U el conjunto de los equipos de la NFL. Considere los siguientes conjuntos:}
\[
\begin{aligned}
AFC &= \{ x \in U \mid x \text{ hace parte de la American Football Conference}\},\\
NFC &= \{ x \in U \mid x \text{ hace parte de la National Football Conference}\},\\
Div &= \{ x \in U \mid x \text{ ganó su división}\},\\
Top_6 &= \{ x \in U \mid x \text{ está en el top 6 de su conferencia}\},\\
Top_2 &= \{ x \in U \mid x \text{ está en el top 2 de su conferencia}\},\\
PlayOFFs &= \{ x \in U \mid x \text{ pasó a los play-offs}\},\\
SB &= \{ x \in U \mid x \text{ juega en el Super Bowl}\}.
\end{aligned}
\]

Para $x \in U$, $d(x)$ denota el turno que le toca al equipo $x$ en el Draft del año siguiente. \medskip

\textbf{Traduzca al español la siguiente proposición:}
\[
(x \in Div \cap Top_2^c)\land(d(x) \geq 21)
\]

\begin{enumerate}
    \item a) Un equipo que ganó su división pero no está en el top 2 de su conferencia tiene un turno en el Draft de al menos 21.  
    \item b) Un equipo que no ganó su división pero está en el top 2 de su conferencia tiene un turno en el Draft de al menos 21.  
    \item c) Todos los equipos en el top 2 de su conferencia tienen un turno en el Draft mayor o igual a 21.  
    \item d) Un equipo que no ganó su división y no está en el top 2 de su conferencia tiene un turno en el Draft menor a 21.  
\end{enumerate}
}
\end{question}

% ==================================================================
% PREGUNTA 25
% (Tema: cuantificadores, proposiciones, dif:1, res:a, week:6)
% ==================================================================
\begin{question}{25}{cuantificadores, proposiciones}{1}{a}{6}{
\\ Suponga que $X$ es el conjunto de sitios para estudiar en la Universidad de los Andes y sea $P=``\text{ está lleno}"$.\\

\textbf{Traduzca al español la siguiente proposición:}
\[
\exists x \in X \text{ tal que } P.
\]

\begin{enumerate}
   \item a) Algún sitio para estudiar en los Andes está lleno.  
   \item b) Todos los sitios para estudiar en los Andes están llenos.  
   \item c) Ningún sitio para estudiar en los Andes está lleno.  
   \item d) No hay sitios para estudiar en los Andes.  
\end{enumerate}
}
\end{question}

% ==================================================================
% PREGUNTA 26
% (Tema: cuantificadores, proposiciones, dif:1, res:b, week:6)
% ==================================================================
\begin{question}{26}{cuantificadores, proposiciones}{1}{b}{6}{
\\Suponga que $X$ es el conjunto de sitios para estudiar en la Universidad de los Andes y sea $P=``\text{ está lleno}"$.\\

\textbf{Traduzca al español la siguiente proposición:}
\[
\forall x \in X, P.
\]

\begin{enumerate}
   \item a) Algún sitio para estudiar en los Andes está lleno.  
   \item b) Todos los sitios para estudiar en los Andes están llenos.  
   \item c) Ningún sitio para estudiar en los Andes está lleno.  
   \item d) Existe al menos un sitio para estudiar en los Andes que no está lleno.  
\end{enumerate}
}
\end{question}

% ==================================================================
% PREGUNTA 27
% (Tema: cuantificadores, proposiciones, dif:1, res:c, week:6)
% ==================================================================
\begin{question}{27}{cuantificadores, proposiciones}{1}{c}{6}{
\\Suponga que $X$ es el conjunto de sitios para estudiar en la Universidad de los Andes y sea $P=``\text{ está lleno}"$.\\

\textbf{Traduzca al español la siguiente proposición:}
\[
\exists x \in X \text{ tal que } (\neg P).
\]

\begin{enumerate}
   \item a) Todos los sitios para estudiar en los Andes están llenos.  
   \item b) Ningún sitio para estudiar en los Andes está lleno.  
   \item c) Algún sitio para estudiar en los Andes no está lleno.  
   \item d) Todos los sitios para estudiar en los Andes no están llenos.  
\end{enumerate}
}
\end{question}

% ==================================================================
% PREGUNTA 28
% (Tema: cuantificadores, proposiciones, dif:1, res:b, week:6)
% ==================================================================
\begin{question}{28}{cuantificadores, proposiciones}{1}{b}{6}{
\\Suponga que $X$ es el conjunto de sitios para estudiar en la Universidad de los Andes y sea $P=``\text{ está lleno}"$.\\

\textbf{Traduzca al español la siguiente proposición:}
\[
\forall x \in X, (\neg P).
\]

\begin{enumerate}
   \item a) Todos los sitios para estudiar en los Andes están llenos.  
   \item b) Ningún sitio para estudiar en los Andes está lleno.  
   \item c) Algún sitio para estudiar en los Andes está lleno.  
   \item d) No hay sitios para estudiar en los Andes.  
\end{enumerate}
}
\end{question}

% ==================================================================
% PREGUNTA 29
% (Tema: cuantificadores, proposiciones, dif:1, res:b, week:6)
% ==================================================================
\begin{question}{29}{cuantificadores, proposiciones}{1}{b}{6}{
\\Suponga que $X$ es el conjunto de sitios para estudiar en la Universidad de los Andes y sean $P=``\text{ está lleno}"$ y $G=``\text{ tiene conexión a internet}"$.\\

\textbf{Traduzca al español la siguiente proposición:}
\[
\exists x \in X\text{ tal que } P \land G.
\]

\begin{enumerate}
   \item a) Todos los sitios para estudiar en los Andes están llenos y tienen conexión a internet.  
   \item b) Algún sitio para estudiar en los Andes está lleno y tiene conexión a internet.  
   \item c) Ningún sitio para estudiar en los Andes está lleno ni tiene conexión a internet.  
   \item d) Todos los sitios para estudiar en los Andes tienen conexión a internet.  
\end{enumerate}
}
\end{question}

% ==================================================================
% PREGUNTA 30
% (Tema: cuantificadores, proposiciones, dif:2, res:c, week:6)
% ==================================================================
\begin{question}{30}{cuantificadores, proposiciones}{2}{c}{6}{
\\Suponga que $X$ es el conjunto de sitios para estudiar en la Universidad de los Andes y sean $P=``\text{ está lleno}"$ y $G=``\text{ tiene conexión a internet}"$.\\

\textbf{Traduzca al español la siguiente proposición:}
\[
\forall x \in X, P \rightarrow G.
\]
\\ Tenga en cuenta que $\rightarrow$ corresponde a la implicacion y se puede tomar como un entonces o consecuencia tipo $A\rightarrow B$, sería A implica B o si A entonces B.
\begin{enumerate}
   \item a) Todos los sitios para estudiar en los Andes tienen conexión a internet.  
   \item b) Ningún sitio para estudiar en los Andes tiene conexión a internet.  
   \item c) Para todo sitio de estudio en la universidad de los andes si este esta lleno, entonces tiene conexión a internet.  
   \item d) Si un sitio para estudiar en los Andes tiene conexión a internet, entonces está lleno.  
\end{enumerate}
}
\end{question}

% ==================================================================
% PREGUNTA 31
% (Tema: cuantificadores, proposiciones, dif:2, res:a, week:6)
% ==================================================================
\begin{question}{31}{cuantificadores, proposiciones}{2}{a}{6}{
\\ \textbf{Suponga que $X$ es el conjunto de sitios para estudiar en la Universidad de los Andes y sean $P=``\text{ está lleno}"$ y $G=``\text{ tiene conexión a internet}"$.}\\

\textbf{Traduzca al español la siguiente proposición:}
\[
\forall x \in X, P \lor (\neg G).
\]

\begin{enumerate}
    \item a) Todos los sitios para estudiar en los Andes están llenos o no tienen conexión a internet.
    \item b) Ningún sitio para estudiar en los Andes está lleno y todos tienen conexión a internet.
    \item c) Algún sitio para estudiar en los Andes está vacío y tiene conexión a internet.
    \item d) Todos los sitios para estudiar en los Andes tienen conexión a internet y no están llenos.
\end{enumerate}
}
\end{question}

% ==================================================================
% PREGUNTA 32
% (Tema: cuantificadores, proposiciones, dif:2, res:c, week:6)
% ==================================================================
\begin{question}{32}{cuantificadores, proposiciones}{2}{c}{6}{
\\ \textbf{Suponga que $X$ es el conjunto de sitios para estudiar en la Universidad de los Andes y sean $P=``\text{ está lleno}"$ y $G=``\text{ tiene conexión a internet}"$.}\\

\textbf{Traduzca al español la siguiente proposición:}
\[
\exists x \in X \text{ tal que } (\neg P) \land G.
\]

\begin{enumerate}
    \item a) Todos los sitios para estudiar en los Andes están llenos y no tienen conexión a internet.
    \item b) Ningún sitio para estudiar en los Andes está lleno y todos tienen conexión a internet.
    \item c) Algún sitio para estudiar en los Andes no está lleno y tiene conexión a internet.
    \item d) Todos los sitios para estudiar en los Andes tienen conexión a internet y están llenos.
\end{enumerate}
}
\end{question}

% ==================================================================
% PREGUNTA 33
% (Tema: cuantificadores, proposiciones, dif:2, res:a, week:6)
% ==================================================================
\begin{question}{33}{cuantificadores, proposiciones}{2}{a}{6}{
\\ \textbf{Suponga que $X$ es el conjunto de sitios para estudiar en la Universidad de los Andes y sean $P=``\text{ está lleno}"$, $G=``\text{ tiene conexión a internet}"$ y $H=``\text{ tiene aire acondicionado}"$.}\\

\textbf{Traduzca al español la siguiente proposición:}
\[
\forall x \in X, (P \land G) \rightarrow H.
\]
\\ Tenga en cuenta que $\rightarrow$ corresponde a la implicacion y se puede tomar como un entonces o consecuencia tipo $A\rightarrow B$, sería A implica B o si A entonces B.
\begin{enumerate}
    \item a) Para todo sitio de estudio en los Andes que está lleno y tiene conexión entonces tiene aire acondicionado.
    \item b) Si un sitio para estudiar en los Andes tiene aire acondicionado, entonces tiene conexión a internet.
    \item c) Algún sitio para estudiar en los Andes está lleno y tiene aire acondicionado.
    \item d) Si todos los sitios para estudiar en los Andes tienen conexión a internet, entonces están llenos y tienen aire acondicionado.
\end{enumerate}
}
\end{question}

% ==================================================================
% PREGUNTA 34
% (Tema: cuantificadores, proposiciones, dif:3, res:b, week:6)
% ==================================================================
\begin{question}{34}{cuantificadores, proposiciones}{3}{b}{6}{
\\ \textbf{Suponga que $X$ es el conjunto de sitios para estudiar en la Universidad de los Andes y sean $P=``\text{ está lleno}"$, $G=``\text{ tiene conexión a internet}"$ y $H=``\text{ tiene aire acondicionado}"$.}\\

\textbf{Traduzca al español la siguiente proposición:}
\[
(\exists x \in X \text{ tal que } (P \land H)) \lor (\forall y \in X, (\neg G) \rightarrow (\neg H)).
\]
\\ Tenga en cuenta que $\rightarrow$ corresponde a la implicacion y se puede tomar como un entonces o consecuencia tipo $A\rightarrow B$, sería A implica B o si A entonces B.
\begin{enumerate}
    \item a) Todos los sitios para estudiar en los Andes tienen aire acondicionado o están llenos.
    \item b) Algún sitio para estudiar en los Andes está lleno y tiene aire acondicionado, o todos los sitios sin conexión a internet no tienen aire acondicionado.
    \item c) Ningún sitio para estudiar en los Andes está lleno o tiene aire acondicionado.
    \item d) Todos los sitios con aire acondicionado están llenos y tienen conexión a internet.
\end{enumerate}
}
\end{question}

% ==================================================================
% PREGUNTA 35
% (Tema: cuantificadores, proposiciones, dif:3, res:a, week:6)
% ==================================================================
\begin{question}{35}{cuantificadores, proposiciones}{3}{a}{6}{
\\ \textbf{Suponga que $X$ es el conjunto de sitios para estudiar en la Universidad de los Andes y sean $P=``\text{ está lleno}"$, $G=``\text{ tiene conexión a internet}"$ y $H=``\text{ tiene aire acondicionado}"$.}\\

\textbf{Traduzca al español la siguiente proposición:}
\[
(\exists x \in X \text{ tal que }P \land G) \rightarrow (\exists y \in X \text{ tal que } H).
\]
\\ Tenga en cuenta que $\rightarrow$ corresponde a la implicacion y se puede tomar como un entonces o consecuencia tipo $A\rightarrow B$, sería A implica B ó si A entonces B.
\begin{enumerate}
    \item a) Si existe un sitio para estudiar en los Andes que está lleno y tiene conexión a internet, entonces hay un sitio con aire acondicionado.
    \item b) Si todos los sitios para estudiar en los Andes tienen conexión a internet, entonces todos tienen aire acondicionado.
    \item c) Si hay un sitio para estudiar en los Andes con aire acondicionado, entonces no está lleno ni tiene conexión a internet.
    \item d) Si un sitio para estudiar en los Andes no tiene aire acondicionado, entonces está lleno y tiene conexión a internet.
\end{enumerate}
}
\end{question}

% ==================================================================
% PREGUNTA 36
% (Tema: cuantificadores, proposiciones, dif:1, res:a, week:6)
% ==================================================================
\begin{question}{36}{cuantificadores, proposiciones}{1}{a}{6}{
\\ \textbf{Suponga que $X$ es el conjunto de primíparos de economía y sea $P=``$ pasa más de 5 horas diarias en clase".}\\

\textbf{Traduzca al español la siguiente proposición:}
\[
\exists x \in X \text{ tal que } P.
\]

\begin{enumerate}
    \item a) Algún primíparo de economía pasa más de 5 horas diarias en clase.
    \item b) Todos los primíparos de economía pasan más de 5 horas diarias en clase.
    \item c) Ningún primíparo de economía pasa más de 5 horas diarias en clase.
    \item d) Todos los primíparos de economía pasan menos de 5 horas diarias en clase.
\end{enumerate}
}
\end{question}

% ==================================================================
% PREGUNTA 37
% (Tema: cuantificadores, proposiciones, dif:1, res:b, week:6)
% ==================================================================
\begin{question}{37}{cuantificadores, proposiciones}{1}{b}{6}{
\\ \textbf{Suponga que $X$ es el conjunto de primíparos de economía y sea $P=``$ pasa más de 5 horas diarias en clase''.}\\

\textbf{Traduzca al español la siguiente proposición:}
\[
\forall x \in X, P.
\]

\begin{enumerate}
\item a) Algún primíparo de economía pasa más de 5 horas diarias en clase.
\item b) Todos los primíparos de economía pasan más de 5 horas diarias en clase.
\item c) Ningún primíparo de economía pasa más de 5 horas diarias en clase.
\item d) Existe al menos un primíparo de economía que pasa menos de 5 horas diarias en clase.
\end{enumerate}
}
\end{question}

% ==================================================================
% PREGUNTA 38
% (Tema: cuantificadores, proposiciones, dif:1, res:c, week:6)
% ==================================================================
\begin{question}{38}{cuantificadores, proposiciones}{1}{c}{6}{
\\ \textbf{Suponga que $X$ es el conjunto de primíparos de economía y sea $P=``$ pasa más de 5 horas diarias en clase''.}\\

\textbf{Traduzca al español la siguiente proposición:}
\[
\exists x \in X \text{ tal que } \neg P.
\]

\begin{enumerate}
\item a) Todos los primíparos de economía pasan más de 5 horas diarias en clase.
\item b) Ningún primíparo de economía pasa más de 5 horas diarias en clase.
\item c) Algún primíparo de economía pasa menos de 5 horas diarias en clase.
\item d) Todos los primíparos de economía pasan menos de 5 horas diarias en clase.
\end{enumerate}
}
\end{question}

% ==================================================================
% PREGUNTA 39
% (Tema: cuantificadores, proposiciones, dif:1, res:b, week:6)
% ==================================================================
\begin{question}{39}{cuantificadores, proposiciones}{1}{b}{6}{
\\ \textbf{Suponga que $X$ es el conjunto de primíparos de economía y sea $P=``$ pasa más de 5 horas diarias en clase''.}\\

\textbf{Traduzca al español la siguiente proposición:}
\[
\forall x \in X, \neg P.
\]

\begin{enumerate}
\item a) Todos los primíparos de economía pasan más de 5 horas diarias en clase.
\item b) Ningún primíparo de economía pasa más de 5 horas diarias en clase.
\item c) Algún primíparo de economía pasa más de 5 horas diarias en clase.
\item d) No hay primíparos de economía.
\end{enumerate}
}
\end{question}

% ==================================================================
% PREGUNTA 40
% (Tema: cuantificadores, proposiciones, dif:1, res:a, week:6)
% ==================================================================
\begin{question}{40}{cuantificadores, proposiciones}{1}{a}{6}{
\textbf{Suponga que $X$ es el conjunto de primíparos de economía y sean:}\\
$P=$ `` pasa más de 5 horas diarias en clase''\\
$G=$ `` estudia al menos 3 horas fuera de clase diariamente''\\
$A=$ `` asiste a todas las clases sin faltar''\\

\textbf{Traduzca al español la siguiente proposición:}
\[
\exists x \in X \text{ tal que } P \land G.
\]

\begin{enumerate}
\item a) Algún primíparo de economía pasa más de 5 horas diarias en clase y estudia al menos 3 horas fuera de clase diariamente.
\item b) Todos los primíparos de economía pasan más de 5 horas diarias en clase y estudian al menos 3 horas fuera de clase.
\item c) Ningún primíparo de economía pasa más de 5 horas diarias en clase ni estudia al menos 3 horas fuera de clase.
\item d) Todos los primíparos de economía estudian al menos 3 horas fuera de clase diariamente, pero no necesariamente pasan más de 5 horas en clase.
\end{enumerate}
}
\end{question}

% ==================================================================
% PREGUNTA 41
% (Tema: cuantificadores, proposiciones, dif:3, res:d, week:6)
% ==================================================================
\begin{question}{41}{cuantificadores, proposiciones}{3}{d}{6}{
\textbf{Suponga que $X$ es el conjunto de primíparos de economía y sean:}\\
$P=$ `` pasa más de 5 horas diarias en clase''\\
$G=$ `` estudia al menos 3 horas fuera de clase diariamente''\\
$A=$ `` asiste a todas las clases sin faltar''\\

\textbf{Traduzca al español la siguiente proposición:}
\[
\forall x \in X,\; P \Rightarrow G.
\]
\\ Tenga en cuenta que $\Rightarrow$ corresponde a la implicacion y se puede tomar como un entonces o consecuencia tipo $A\Rightarrow B$, sería A implica B o si A entonces B.
\begin{enumerate}
\item a) Algún primíparo de economía que pasa más de 5 horas en clase no estudia fuera de clase.
\item b) Si un primíparo de economía estudia al menos 3 horas fuera de clase, entonces pasa más de 5 horas diarias en clase.
\item c) Todos los primíparos de economía que pasan más de 5 horas en clase también asisten a todas sus clases sin faltar.
\item d) Si un primíparo de economía pasa más de 5 horas diarias en clase, entonces también estudia al menos 3 horas fuera de clase diariamente.
\end{enumerate}
}
\end{question}

% ==================================================================
% PREGUNTA 42
% (Tema: cuantificadores, proposiciones, dif:3, res:a, week:6)
% ==================================================================
\begin{question}{42}{cuantificadores, proposiciones}{3}{a}{6}{
\textbf{Suponga que $X$ es el conjunto de primíparos de economía y sean:}\\
$P=$ `` pasa más de 5 horas diarias en clase''\\
$G=$ `` estudia al menos 3 horas fuera de clase diariamente''\\
$A=$ `` asiste a todas las clases sin faltar''\\

\textbf{Traduzca al español la siguiente proposición:}
\[
\forall x \in X,\; (P \land G) \Rightarrow A.
\]
\\ Tenga en cuenta que $\Rightarrow$ corresponde a la implicacion y se puede tomar como un entonces o consecuencia tipo $A\Rightarrow B$, sería A implica B o si A entonces B.
\begin{enumerate}
\item a) Para todos los primiparos si pasa más de 5 horas diarias en clase y estudia al menos 3 horas fuera de clase diariamente, entonces asiste a todas sus clases sin faltar.
\item b) Si un primíparo de economía pasa más de 5 horas diarias en clase, entonces estudia al menos 3 horas fuera de clase diariamente.
\item c) Si un primíparo de economía no estudia al menos 3 horas fuera de clase diariamente, entonces no asiste a todas sus clases sin faltar.
\item d) Todos los primíparos de economía que estudian fuera de clase diariamente asisten a todas las clases sin faltar.
\end{enumerate}
}
\end{question}

% ==================================================================
% PREGUNTA 43
% (Tema: cuantificadores, proposiciones, dif:2, res:c, week:6)
% ==================================================================
\begin{question}{43}{cuantificadores, proposiciones}{3}{c}{6}{
\textbf{Suponga que $X$ es el conjunto de primíparos de economía y sean:}\\
$P=$ `` pasa más de 5 horas diarias en clase''\\
$G=$ `` estudia al menos 3 horas fuera de clase diariamente''\\
$A=$ `` asiste a todas las clases sin faltar''\\

\textbf{Traduzca al español la siguiente proposición:}
\[
(\exists x \in X \text{ tal que } P \lor A) \Rightarrow (\forall y \in X,\; G \Rightarrow P).
\]
\\ Tenga en cuenta que $\Rightarrow$ corresponde a la implicacion y se puede tomar como un entonces o consecuencia tipo $A\Rightarrow B$, sería A implica B o si A entonces B.
\begin{enumerate}
\item a) Si un primíparo de economía pasa más de 5 horas en clase o asiste a todas sus clases sin faltar, entonces todos los primíparos asisten a todas sus clases sin faltar.
\item b) Si todos los primíparos de economía que estudian al menos 3 horas fuera de clase diariamente pasan más de 5 horas diarias en clase, entonces todos asisten a todas sus clases sin faltar.
\item c) Si al menos un primíparo de economía pasa más de 5 horas diarias en clase o asiste a todas sus clases sin faltar, entonces todos los primíparos que estudian al menos 3 horas fuera de clase diariamente también pasan más de 5 horas en clase.
\item d) Si algún primíparo de economía no estudia al menos 3 horas fuera de clase diariamente, entonces no pasa más de 5 horas en clase.
\end{enumerate}
}
\end{question}

% ==================================================================
% PREGUNTA 44
% (Tema: cuantificadores, proposiciones, dif:3, res:c, week:6)
% ==================================================================
\begin{question}{44}{cuantificadores, proposiciones}{1}{c}{6}{
\textbf{Suponga que $X$ es el conjunto de primíparos de economía y sean:}\\
$P=$ `` pasa más de 5 horas diarias en clase''\\
$G=$ `` estudia al menos 3 horas fuera de clase diariamente''\\
$A=$ `` asiste a todas las clases sin faltar''\\

\textbf{Traduzca al español la siguiente proposición:}
\[
(\exists x \in X\text{ tal que } P \land G \land \neg A)\]

\begin{enumerate}
\item a) Si un primíparo de economía asiste a todas sus clases sin faltar, entonces estudia al menos 3 horas fuera de clase diariamente y pasa más de 5 horas diarias en clase.
\item b) Si algún primíparo de economía pasa más de 5 horas diarias en clase y estudia al menos 3 horas fuera de clase diariamente, entonces todos los primíparos que estudian fuera de clase diariamente también asisten a todas sus clases sin faltar.
\item c) Existe un Primiparo de economía que pasa más de 5 horas en clase, estudia al meno 3 horas fuera de clase diariamente y no asiste a todas las clases.
\item d) Si un primíparo de economía estudia fuera de clase diariamente, entonces o bien asiste a todas sus clases sin faltar o bien no pasa más de 5 horas diarias en clase.
\end{enumerate}
}
\end{question}

% ==================================================================
% PREGUNTA 45
% ==================================================================
\begin{question}{45}{lógica, proposiciones}{1}{b}{3}{
\textbf{Suponga que:} 
\\
\(p = \) ``Llovió ayer por la noche".\\
\(q = \) ``Se prendieron los rociadores de pasto ayer por la noche".\\
\(r = \) ``El pasto estaba mojado hoy por la mañana".\medskip

\textbf{Traduzca al español la siguiente proposición:}
\[
(\neg p) \lor (q \land r)
\]

\begin{enumerate}
    \item a) No llovió ayer por la noche y se prendieron los rociadores de pasto ayer por la noche y el pasto estaba mojado hoy por la mañana.
    \item b) No llovió ayer por la noche o, se prendieron los rociadores de pasto ayer por la noche y el pasto estaba mojado hoy por la mañana.
    \item c) Llovió ayer por la noche y no se prendieron los rociadores de pasto ayer por la noche o el pasto estaba seco hoy por la mañana.
    \item d) El pasto no estaba mojado hoy por la mañana y llovió ayer por la noche.
\end{enumerate}
}
\end{question}

% ==================================================================
% PREGUNTA 46
% ==================================================================
\begin{question}{46}{lógica, proposiciones}{2}{d}{3}{
\textbf{Suponga que:} 
\\
\(p = \) ``Llovió ayer por la noche".\\
\(q = \) ``Se prendieron los rociadores de pasto ayer por la noche".\\
\(r = \) ``El pasto estaba mojado hoy por la mañana".\medskip

\textbf{Traduzca al español la siguiente proposición:}
\[
(p \rightarrow r)
\]
\\ Tenga en cuenta que $\rightarrow$ corresponde a la implicacion y se puede tomar como un entonces o consecuencia tipo $A\rightarrow B$, sería A implica B o si A entonces B.
\begin{enumerate}
    \item a) Si llovió ayer por la noche, entonces se prendieron los rociadores.
    \item b) Si el pasto estaba mojado hoy por la mañana, entonces llovió ayer por la noche.
    \item c) Si se prendieron los rociadores, entonces llovió ayer por la noche.
    \item d) Si llovió ayer por la noche, entonces el pasto estaba mojado hoy por la mañana.
\end{enumerate}
}
\end{question}

% ==================================================================
% PREGUNTA 47
% ==================================================================
\begin{question}{47}{lógica, proposiciones}{2}{a}{3}{
\textbf{Suponga que:} 
\\
\(p = \) ``Llovió ayer por la noche".\\
\(q = \) ``Se prendieron los rociadores de pasto ayer por la noche".\\
\(r = \) ``El pasto estaba mojado hoy por la mañana".\medskip

\textbf{Traduzca al español la siguiente proposición:}
\[
(p \land (q \rightarrow r))
\]
\\ Tenga en cuenta que $\Rightarrow$ corresponde a la implicacion y se puede tomar como un entonces o consecuencia tipo $A\Rightarrow B$, sería A implica B o si A entonces B.
\begin{enumerate}
    \item a) Llovió ayer por la noche y, si se prendieron los rociadores, entonces el pasto estaba mojado hoy por la mañana.
    \item b) Llovió ayer por la noche o los rociadores no se prendieron.
    \item c) Si llovió ayer por la noche y se prendieron los rociadores, entonces el pasto estaba mojado hoy por la mañana.
    \item d) El pasto estaba mojado hoy por la mañana solo si llovió ayer por la noche.
\end{enumerate}
}
\end{question}

% ==================================================================
% PREGUNTA 48
% ==================================================================
\begin{question}{48}{lógica, proposiciones}{1}{c}{3}{
\textbf{Suponga que:} 
\\
\(p = \) ``Llovió ayer por la noche".\\
\(q = \) ``Se prendieron los rociadores de pasto ayer por la noche".\\
\(r = \) ``El pasto estaba mojado hoy por la mañana".\medskip

\textbf{Traduzca al español la siguiente proposición:}
\[
((\neg p) \land (q \lor r))
\]

\begin{enumerate}
    \item a) Si llovió ayer por la noche, entonces se prendieron los rociadores o el pasto estaba mojado hoy por la mañana.
    \item b) Si no se prendieron los rociadores, entonces llovió ayer por la noche o el pasto estaba mojado hoy por la mañana.
    \item c) No llovió ayer por la noche, y se prendieron los rociadores o el pasto estaba mojado hoy por la mañana.
    \item d) Si no llovió ayer por la noche, entonces el pasto estaba seco hoy por la mañana y no se prendieron los rociadores.
\end{enumerate}
}
\end{question}

% ==================================================================
% PREGUNTA 49
% ==================================================================
\begin{question}{49}{lógica, proposiciones}{1}{b}{3}{
\textbf{Suponga que:} 
\\
\(p = \) ``Llovió ayer por la noche".\\
\(q = \) ``Se prendieron los rociadores de pasto ayer por la noche".\\
\(r = \) ``El pasto estaba mojado hoy por la mañana".\medskip

\textbf{Traduzca al español la siguiente proposición:}
\[
\neg (p \lor r)
\]

\begin{enumerate}
    \item a) No llovió ayer por la noche o el pasto no estaba mojado hoy por la mañana.
    \item b) No llovió ayer por la noche y el pasto no estaba mojado hoy por la mañana.
    \item c) Llovió ayer por la noche y el pasto estaba mojado hoy por la mañana.
    \item d) Se prendieron los rociadores y el pasto estaba mojado hoy por la mañana.
\end{enumerate}
}
\end{question}

% ==================================================================
% PREGUNTA 50
% ==================================================================
\begin{question}{50}{lógica, proposiciones}{2}{c}{3}{
\textbf{Suponga que:} 
\\
\(p = \) ``Llovió ayer por la noche".\\
\(q = \) ``Se prendieron los rociadores de pasto ayer por la noche".\\
\(r = \) ``El pasto estaba mojado hoy por la mañana".\medskip

\textbf{Traduzca al español la siguiente proposición:}
\[
(q \leftrightarrow r)
\]
\\ Tenga en cuenta que $\leftrightarrow$ corresponde al si y solo si y se puede interpretar $A \leftrightarrow B$ como: A ocurre si y solo si B ocurre.

\begin{enumerate}
    \item a) Si se prendieron los rociadores de pasto, entonces el pasto estaba mojado hoy por la mañana.
    \item b) Si el pasto estaba mojado hoy por la mañana, entonces se prendieron los rociadores.
    \item c) Los rociadores se prendieron ayer por la noche si y solo si el pasto estaba mojado hoy por la mañana.
    \item d) Los rociadores no se prendieron o el pasto estaba seco hoy por la mañana.
\end{enumerate}
}
\end{question}

% ==================================================================
% PREGUNTA 51
% ==================================================================
\begin{question}{51}{lógica, proposiciones}{1}{d}{3}{
\textbf{Suponga que:} 
\\
\(p = \) ``Llovió ayer por la noche".\\
\(q = \) ``Se prendieron los rociadores de pasto ayer por la noche".\\
\(r = \) ``El pasto estaba mojado hoy por la mañana".\medskip

\textbf{Traduzca al español la siguiente proposición:}
\[
\neg (p \land q)
\]

\begin{enumerate}
    \item a) Llovió ayer por la noche y se prendieron los rociadores de pasto ayer por la noche.
    \item b) No llovió ayer por la noche y no se prendieron los rociadores.
    \item c) No llovió ayer por la noche o no se prendieron los rociadores.
    \item d) No llovió ayer por la noche o no se prendieron los rociadores de pasto ayer por la noche.
\end{enumerate}
}
\end{question}

% ==================================================================
% PREGUNTA 52
% ==================================================================
\begin{question}{52}{lógica, proposiciones}{1}{a}{3}{
\textbf{Suponga que:} 
\\
\(p = \) ``Llovió ayer por la noche".\\
\(q = \) ``Se prendieron los rociadores de pasto ayer por la noche".\\
\(r = \) ``El pasto estaba mojado hoy por la mañana".\medskip

\textbf{Traduzca al español la siguiente proposición:}
\[
(p \lor (\neg r))
\]

\begin{enumerate}
    \item a) Llovió ayer por la noche o el pasto no estaba mojado hoy por la mañana.
    \item b) Llovió ayer por la noche y el pasto estaba mojado hoy por la mañana.
    \item c) No llovió ayer por la noche y el pasto no estaba mojado hoy por la mañana.
    \item d) El pasto estaba mojado hoy por la mañana o no se prendieron los rociadores.
\end{enumerate}
}
\end{question}

% ==================================================================
% PREGUNTA 53
% ==================================================================
\begin{question}{53}{lógica, proposiciones}{1}{c}{3}{
\textbf{Suponga que:} 
\\
\(p = \) ``Llovió ayer por la noche".\\
\(q = \) ``Se prendieron los rociadores de pasto ayer por la noche".\\
\(r = \) ``El pasto estaba mojado hoy por la mañana".\medskip

\textbf{Traduzca al español la siguiente proposición:}
\[
((p \lor q) \land r)
\]

\begin{enumerate}
    \item a) Llovió ayer por la noche o se prendieron los rociadores de pasto ayer por la noche.
    \item b) Si llovió ayer por la noche o se prendieron los rociadores, entonces el pasto estaba mojado hoy por la mañana.
    \item c) Llovió ayer por la noche o se prendieron los rociadores de pasto ayer por la noche, y además el pasto estaba mojado hoy por la mañana.
    \item d) El pasto estaba seco hoy por la mañana y no se prendieron los rociadores.
\end{enumerate}
}
\end{question}

% ==================================================================
% PREGUNTA 54
% ==================================================================
\begin{question}{54}{lógica, proposiciones}{2}{b}{3}{
\textbf{Suponga que:} 
\\
\(p = \) ``Usted estudió para el examen".\\
\(q = \) ``Usted aprobó el examen".\\
\(r = \) ``Usted celebró con sus amigos".\medskip

\textbf{Traduzca al español la siguiente proposición:}
\[
(p \rightarrow q) \land (q \rightarrow r)
\]
\\ Tenga en cuenta que $\rightarrow$ corresponde a la implicacion y se puede tomar como un entonces o consecuencia tipo $A\rightarrow B$, sería A implica B o si A entonces B.
\begin{enumerate}
    \item a) Si usted celebró con sus amigos, entonces estudió para el examen y aprobó.
    \item b) Si usted estudió para el examen, entonces aprobó; y si aprobó, entonces celebró con sus amigos.
    \item c) Si usted no estudió, entonces no aprobó ni celebró.
    \item d) Usted aprobó solo si estudió y celebró con sus amigos.
\end{enumerate}
}
\end{question}

% ==================================================================
% PREGUNTA 55
% ==================================================================
\begin{question}{55}{lógica, proposiciones}{2}{c}{3}{
\textbf{Suponga que:} 
\\
\(p = \) ``El informe fue entregado a tiempo".\\
\(q = \) ``El jefe quedó satisfecho".\\
\(r = \) ``Habrá aumento de salario".\medskip

\textbf{Traduzca al español la siguiente proposición:}
\[
p \land q \land r
\]

\begin{enumerate}
    \item a) Si el informe fue entregado a tiempo, entonces habrá aumento de salario.
    \item b) Si hubo aumento de salario, entonces el jefe no quedó satisfecho.
    \item c) El informe fue entregado a tiempo; además, el jefe quedo satisfecho y habrá aumento de salario.
    \item d) Si no hubo aumento de salario, entonces el informe no fue entregado a tiempo.
\end{enumerate}
}
\end{question}

% ==================================================================
% PREGUNTA 56
% ==================================================================
\begin{question}{56}{lógica, proposiciones}{1}{b}{3}{
\textbf{Suponga que:} 
\\
\(p = \) ``El motor está dañado".\\
\(q = \) ``El auto no arranca".\\
\(r = \) ``El auto necesita reparación".\medskip

\textbf{Traduzca literalmente al español la siguiente proposición:}
\[
((p \lor q) \land (\neg p \lor r))
\]

\begin{enumerate}
    \item a) El motor está dañado o el auto no arranca, pero no necesita reparación si el motor no está dañado.
    \item b) El motor está dañado o el auto no arranca; y, el motor no está dañado o el auto necesita reparación.
    \item c) El motor no está dañado o el auto arranca, y el auto no necesita reparación si el motor funciona bien.
    \item d) El auto necesita reparación solo si el motor está dañado y no arranca.
\end{enumerate}
}
\end{question}

% ==================================================================
% PREGUNTA 57
% ==================================================================
\begin{question}{57}{lógica, proposiciones}{3}{a}{3}{
\textbf{Suponga que:} 
\\
\(p = \) ``Pedro estudia".\\
\(q = \) ``Pedro aprueba".\\
\(r = \) ``Pedro va a la fiesta".\medskip

\textbf{Traduzca literalmente al español la siguiente proposición:}
\[
((p \rightarrow q) \lor (r \rightarrow q))
\]
\\ Tenga en cuenta que $\rightarrow$ corresponde a la implicacion y se puede tomar como un entonces o consecuencia tipo $A\rightarrow B$, sería A implica B o si A entonces B.
\begin{enumerate}
    \item a) Si Pedro estudia entonces aprueba; ó, si pedro va la fiesta entonces aprueba.
    \item b) Pedro estudia o va a la fiesta solo si aprueba.
    \item c) Pedro aprueba o estudia y va a la fiesta.
    \item d) Pedro aprueba y no estudia ni va a la fiesta.
\end{enumerate}
}
\end{question}

% ==================================================================
% PREGUNTA 58
% ==================================================================
\begin{question}{58}{lógica, proposiciones}{2}{d}{3}{
\textbf{Suponga que:} 
\\
\(p = \) ``María cocina".\\
\(q = \) ``Juan lava los platos".\\
\(r = \) ``La cocina queda limpia".\medskip

\textbf{Traduzca al español la siguiente proposición:}
\[
\neg ((p \land q) \rightarrow r)
\]
\\ Tenga en cuenta que $\rightarrow$ corresponde a la implicacion y se puede tomar como un entonces o consecuencia tipo $A\rightarrow B$, sería A implica B o si A entonces B.

\begin{enumerate}
    \item a) María cocina y Juan lava los platos si y solo si la cocina queda limpia.
    \item b) Si María cocina y Juan lava los platos, entonces la cocina no queda limpia.
    \item c) Es falso que si la cocina queda limpia, entonces María cocinó y Juan lavó los platos.
    \item d) Es falso que si María cocina y Juan lava los platos, entonces la cocina queda limpia.
\end{enumerate}
}
\end{question}

% ==================================================================
% PREGUNTA 59
% ==================================================================
\begin{question}{59}{lógica, proposiciones}{2}{a}{3}{
\textbf{Suponga que:} 
\\
\(p = \) ``Estudio lógica".\\
\(q = \) ``Apruebo el examen".\\
\(r = \) ``Obtengo beca".\medskip

\textbf{Traduzca literalmente al español la siguiente proposición:}
\[
(p \leftrightarrow (q \lor r))
\]
\\ Tenga en cuenta que $\leftrightarrow$ corresponde al si y solo si y se puede interpretar $A \leftrightarrow B$ como: A ocurre si y solo si B ocurre.
\begin{enumerate}
    \item a) Estudio lógica; si y solo si, apruebo el examen o obtengo beca.
    \item b) Si apruebo el examen o obtengo beca, entonces estudio lógica.
    \item c) Si estudio lógica, entonces apruebo el examen y obtengo beca.
    \item d) Estudio lógica solo si no apruebo el examen y no obtengo beca.
\end{enumerate}
}
\end{question}

% ==================================================================
% PREGUNTA 60
% ==================================================================
\begin{question}{60}{lógica, proposiciones}{3}{b}{3}{
\textbf{Suponga que:} 
\\
\(p = \) ``Sale el sol".\\
\(q = \) ``Hay calor".\\
\(r = \) ``Uso bloqueador solar".\medskip

\textbf{Traduzca literalmente al español la siguiente proposición:}
\[
\neg [ (p \rightarrow q) \lor (\neg q \rightarrow r) ]
\]

\begin{enumerate}
    \item a) Es falso que si no hace calor, entonces uso bloqueador solar o si sale el sol, hace calor.
    \item b) Es falso, que si sale el sol entonces hace calor; o, si no hace calor entonces uso bloqueador solar.
    \item c) Es falso que si sale el sol entonces hace calor y uso bloqueador solar.
    \item d) Es cierto que si sale el sol entonces hace calor o uso bloqueador solar.
\end{enumerate}
}
\end{question}

% ==================================================================
% PREGUNTA 61
% ==================================================================
\begin{question}{61}{lógica, proposiciones}{1}{d}{3}{
\textbf{Suponga que:} 
\\
\(p = \) ``Carlos lee libros".\\
\(q = \) ``Carlos escribe artículos".\\
\(r = \) ``Carlos es un buen investigador".\medskip

\textbf{Traduzca al español la siguiente proposición:}
\[
 r \leftrightarrow (p \land q)
\]
\\ Tenga en cuenta que $\leftrightarrow$ corresponde al si y solo si y se puede interpretar $A \leftrightarrow B$ como: A ocurre si y solo si B ocurre.
\begin{enumerate}
    \item a) Si Carlos es un buen investigador, entonces lee libros y escribe artículos.
    \item b) Carlos es buen investigador si y solo si lee libros y escribe artículos.
    \item c) Si Carlos lee libros o escribe artículos, entonces es buen investigador.
    \item d) Carlos es buen investigador si y solo si es falso que no es buen investigador o no lee libros o no escribe artículos.
\end{enumerate}
}
\end{question}

% ==================================================================
% PREGUNTA 62
% ==================================================================
\begin{question}{62}{lógica, proposiciones}{1}{c}{3}{
\textbf{Suponga que:} 
\\
\(p = \) ``Voy al gimnasio".\\
\(q = \) ``Como saludable".\\
\(r = \) ``Tengo buena salud".\medskip

\textbf{Traduzca al español la siguiente proposición:}
\[
[ p \land q \land (\neg r) ]
\]

\begin{enumerate}
    \item a) Voy al gimnasio y como saludable o no tengo buena salud.
    \item b) Es falso que voy al gimnasio, como saludable o no tengo buena salud.
    \item c) Voy al gimnasio y  como saludable y no tengo buena salud.
    \item d) Tengo buena salud si y solo si voy al gimnasio o como saludable.
\end{enumerate}
}
\end{question}

% ==================================================================
% PREGUNTA 63
% ==================================================================
\begin{question}{63}{lógica, proposiciones}{3}{a}{3}{
\textbf{Suponga que:} 
\\
\(p = \) ``Lluvia intensa".\\
\(q = \) ``Inundación en la ciudad".\\
\(r = \) ``Suspensión de clases".\medskip

\textbf{Traduzca al español la siguiente proposición:}
\[
(p \rightarrow q) \land (q \rightarrow r)
\]
\\ Tenga en cuenta que $\rightarrow$ corresponde a la implicacion y se puede tomar como un entonces o consecuencia tipo $A\rightarrow B$, sería A implica B o si A entonces B.

\begin{enumerate}
    \item a) Si hay lluvia intensa, entonces hay inundación en la ciudad, y si hay inundación en la ciudad, entonces se suspenden las clases.
    \item b) Si hay lluvia intensa o hay inundación en la ciudad, entonces hay suspensión de clases.
    \item c) Hay suspensión de clases si y solo si hay lluvia intensa.
    \item d) Si no hay lluvia intensa, entonces no hay suspensión de clases.
\end{enumerate}
}
\end{question}

% ==================================================================
% PREGUNTA 64
% ==================================================================
\begin{question}{64}{lógica, proposiciones}{3}{b}{3}{
\textbf{Suponga que:} 
\\
\(p = \) ``Tengo tarea".\\
\(q = \) ``Salgo con mis amigos".\\
\(r = \) ``Llego tarde a casa".\medskip

\textbf{Traduzca al español la siguiente proposición:}
\[
((p \land q) \lor r) \rightarrow q
\]
\\ Tenga en cuenta que $\rightarrow$ corresponde a la implicacion y se puede tomar como un entonces o consecuencia tipo $A\rightarrow B$, sería A implica B o si A entonces B.
\begin{enumerate}
    \item a) Si tengo tarea y salgo con mis amigos o llego tarde a casa, entonces no salgo con mis amigos.
    \item b) Si tengo tarea y salgo con mis amigos o, llego tarde a casa; entonces salgo con mis amigos.
    \item c) Si salgo con mis amigos entonces tengo tarea o llego tarde a casa.
    \item d) Salgo con mis amigos o llego tarde a casa si y solo si tengo tarea.
\end{enumerate}
}
\end{question}

% ==================================================================
% PREGUNTA 65
% ==================================================================
\begin{question}{65}{lógica, proposiciones}{1}{b}{3}{
\textbf{Suponga que:} 
\\
\(p = \) ``Estudio matemáticas".\\
\(q = \) ``Estudio lógica".\\
\(r = \) ``Estudio programación".\\
\(s = \) ``Estudio economía".\medskip

\textbf{Traduzca al español la siguiente proposición:}
\[
p \land q
\]

\begin{enumerate}
    \item a) No estudio matemáticas ni estudio lógica.
    \item b) Estudio matemáticas y estudio lógica.
    \item c) Estudio programación y estudio economía.
    \item d) Estudio lógica o estudio economía.
\end{enumerate}
}
\end{question}

% ==================================================================
% PREGUNTA 66
% ==================================================================
\begin{question}{66}{lógica, proposiciones}{1}{a}{3}{
\textbf{Suponga que:} 
\\
\(p = \) ``Leo libros".\\
\(q = \) ``Escribo poemas".\\
\(r = \) ``Toco guitarra".\\
\(s = \) ``Canto canciones".\medskip

\textbf{Traduzca al español la siguiente proposición:}
\[
\neg s
\]

\begin{enumerate}
    \item a) No canto canciones.
    \item b) Canto canciones.
    \item c) No leo libros.
    \item d) Escribo poemas.
\end{enumerate}
}
\end{question}

% ==================================================================
% PREGUNTA 67
% ==================================================================
\begin{question}{67}{lógica, proposiciones}{1}{d}{3}{
\textbf{Suponga que:} 
\\
\(p = \) ``Llueve".\\
\(q = \) ``Hace sol".\\
\(r = \) ``Hace viento".\\
\(s = \) ``Nieva".\medskip

\textbf{Traduzca al español la siguiente proposición:}
\[
q \lor r
\]

\begin{enumerate}
    \item a) Llueve y hace viento.
    \item b) Nieva o llueve.
    \item c) Nieva y hace sol.
    \item d) Hace sol o hace viento.
\end{enumerate}
}
\end{question}

% ==================================================================
% PREGUNTA 68
% ==================================================================
\begin{question}{68}{lógica, proposiciones}{2}{a}{3}{
\textbf{Suponga que:} 
\\
\(p = \) ``Salgo a correr".\\
\(q = \) ``Practico yoga".\\
\(r = \) ``Voy al gimnasio".\\
\(s = \) ``Hago pilates".\medskip

\textbf{Traduzca al español la siguiente proposición:}
\[
(p \land q) \lor r
\]

\begin{enumerate}
    \item a) Salgo a correr y practico yoga, o voy al gimnasio.
    \item b) Hago pilates o salgo a correr y practico yoga.
    \item c) No voy al gimnasio ni hago pilates.
    \item d) Practico yoga o hago pilates.
\end{enumerate}
}
\end{question}

% ==================================================================
% PREGUNTA 69
% ==================================================================
\begin{question}{69}{lógica, proposiciones}{1}{b}{3}{
\textbf{Suponga que:} 
\\
\(p = \) ``Desayuné".\\
\(q = \) ``Almorcé".\\
\(r = \) ``Cené".\\
\(s = \) ``Meriendé".\medskip

\textbf{Traduzca al español la siguiente proposición:}
\[
\neg p \lor q
\]

\begin{enumerate}
    \item a) No almorcé o desayuné.
    \item b) No desayuné o almorcé.
    \item c) Desayuné y cené.
    \item d) No cené ni merendé.
\end{enumerate}
}
\end{question}

% ==================================================================
% PREGUNTA 70
% ==================================================================
\begin{question}{70}{lógica, proposiciones}{1}{c}{3}{
\textbf{Suponga que:} 
\\
\(p = \) ``Viajo en avión".\\
\(q = \) ``Viajo en tren".\\
\(r = \) ``Viajo en bus".\\
\(s = \) ``Viajo en barco".\medskip

\textbf{Traduzca al español la siguiente proposición:}
\[
\neg (p \land r)
\]

\begin{enumerate}
    \item a) Viajo en avión o viajo en bus.
    \item b) No viajo en tren ni en barco.
    \item c) No, viajo en avión y en bus al mismo tiempo.
    \item d) Viajo en avión y en tren al mismo tiempo.
\end{enumerate}
}
\end{question}

% ==================================================================
% PREGUNTA 71
% ==================================================================
\begin{question}{71}{lógica, proposiciones}{1}{d}{3}{
\textbf{Suponga que:} 
\\
\(p = \) ``Tengo celular".\\
\(q = \) ``Tengo computador".\\
\(r = \) ``Tengo tablet".\\
\(s = \) ``Tengo televisor".\medskip

\textbf{Traduzca al español la siguiente proposición:}
\[
(p \lor q) \land (r \lor s)
\]

\begin{enumerate}
    \item a) Tengo celular y computador y tablet y televisor.
    \item b) No tengo celular ni computador ni tablet ni televisor.
    \item c) Tengo celular o computador o tablet o televisor.
    \item d) Tengo celular o computador; y, además tengo tablet o televisor.
\end{enumerate}
}
\end{question}

% ==================================================================
% PREGUNTA 72
% ==================================================================
\begin{question}{72}{lógica, proposiciones}{1}{a}{3}{
\textbf{Suponga que:} 
\\
\(p = \) ``Tengo perro".\\
\(q = \) ``Tengo gato".\\
\(r = \) ``Tengo pez".\\
\(s = \) ``Tengo loro".\medskip

\textbf{Traduzca al español la siguiente proposición:}
\[
\neg q
\]

\begin{enumerate}
    \item a) No tengo gato.
    \item b) Tengo perro.
    \item c) No tengo pez.
    \item d) Tengo gato.
\end{enumerate}
}
\end{question}

% ==================================================================
% PREGUNTA 73
% ==================================================================
\begin{question}{73}{lógica, proposiciones}{1}{c}{3}{
\textbf{Suponga que:} 
\\
\(p = \) ``Como pizza".\\
\(q = \) ``Como hamburguesa".\\
\(r = \) ``Como pasta".\\
\(s = \) ``Como ensalada".\medskip

\textbf{Traduzca al español la siguiente proposición:}
\[
(p \lor q) \lor (r \lor s)
\]

\begin{enumerate}
    \item a) No como pizza ni hamburguesa ni pasta ni ensalada.
    \item b) Como pizza y hamburguesa y pasta y ensalada.
    \item c) Como pizza o hamburguesa o pasta o ensalada.
    \item d) Solo como ensalada.
\end{enumerate}
}
\end{question}

% ==================================================================
% PREGUNTA 74
% ==================================================================
\begin{question}{74}{lógica, proposiciones}{1}{b}{3}{
\textbf{Suponga que:} 
\\
\(p = \) ``Veo televisión".\\
\(q = \) ``Escucho radio".\\
\(r = \) ``Uso internet".\\
\(s = \) ``Leo el periódico".\medskip

\textbf{Traduzca al español la siguiente proposición:}
\[
\neg r
\]

\begin{enumerate}
    \item a) Uso internet.
    \item b) No uso internet.
    \item c) Veo televisión.
    \item d) Escucho radio.
\end{enumerate}
}
\end{question}
% ==================================================================
% PREGUNTA 75
% ==================================================================
\begin{question}{75}{conjuntos}{1}{d}{2}{
\\ Considere el universo de estudiantes de la Facultad de Ingeniería. Sean:
\[
\begin{aligned}
A &= \{x \in U \mid x \text{ estudia Ingeniería de Sistemas}\},\\
B &= \{x \in U \mid x \text{ estudia Ingeniería Civil}\}.
\end{aligned}
\]
\textbf{Traduzca:} $A \cap B = \emptyset$
\begin{enumerate}
    \item a) Hay estudiantes que estudian ambas ingenierías.
    \item b) Ningún estudiante estudia Ingeniería de Sistemas.
    \item c) Ningún estudiante estudia Ingeniería Civil.
    \item d) Ningún estudiante estudia ingeniería de sistemas he ingeniería civil.
\end{enumerate}
}
\end{question}
% ==================================================================
% PREGUNTA 76
% ==================================================================
\begin{question}{76}{conjuntos}{1}{a}{2}{\\ Considere el universo de estudiantes de la Facultad de Ingeniería. Sean:
\[
\begin{aligned}
A &= \{x \in U \mid x \text{ estudia Ingeniería de Sistemas}\},\\
B &= \{x \in U \mid x \text{ estudia Ingeniería Civil}\}.
\end{aligned}
\]
 \textbf{Traduzca:} $A \subseteq U$
\begin{enumerate}
    \item a) Todos los estudiantes de Ingeniería de Sistemas pertenecen a la facultad de Ingeniería.
    \item b) Ningún estudiante pertenece a Ingeniería de Sistemas.
    \item c) Todos los estudiantes pertenecen a Ingeniería Civil.
    \item d) Hay estudiantes de Ingeniería de Sistemas que no están en $U$.
\end{enumerate}
}
\end{question}
% ==================================================================
% PREGUNTA 77
% ==================================================================
\begin{question}{77}{conjuntos}{1}{b}{2}{\\ Considere el universo de estudiantes de la Facultad de Ingeniería. Sean:
\[
\begin{aligned}
A &= \{x \in U \mid x \text{ estudia Ingeniería de Sistemas}\},\\
B &= \{x \in U \mid x \text{ estudia Ingeniería Civil}\}.
\end{aligned}
\]
 \textbf{Traduzca:} $A \cup B = U$
\begin{enumerate}
    \item a) Hay estudiantes que no pertenecen ni a $A$ ni a $B$.
    \item b) Todos los estudiantes de la facultad de ingeniería estudian Ingeniería Civil o de sistemas.
    \item c) Ningún estudiante pertenece a ambos conjuntos.
    \item d) Ningún estudiante pertenece al conjunto $U$.
\end{enumerate}
}
\end{question}
% ==================================================================
% PREGUNTA 78
% ==================================================================
\begin{question}{78}{conjuntos}{1}{c}{2}{
\\ Sean $M$ los estudiantes que manejan inglés y $N$ los que manejan francés. \\ \textbf{Traduzca:} $M \cap N \neq \emptyset$
\begin{enumerate}
    \item a) Nadie maneja ambos idiomas.
    \item b) Todos manejan ambos idiomas.
    \item c) Al menos un estudiante maneja inglés y francés.
    \item d) Ningún estudiante sabe inglés.
\end{enumerate}
}
\end{question}
% ==================================================================
% PREGUNTA 79
% ==================================================================
\begin{question}{79}{conjuntos}{1}{a}{2}{Sean $M$ los estudiantes que manejan inglés y $N$ los que manejan francés.\\
\textbf{Traduzca:} $M^c \cap N = \emptyset$
\begin{enumerate}
    \item a) Todos los que manejan francés también manejan inglés.
    \item b) Ningún estudiante maneja francés.
    \item c) Hay estudiantes que manejan sólo francés.
    \item d) Ningún estudiante maneja ambos idiomas.
\end{enumerate}
}
\end{question}
% ==================================================================
% PREGUNTA 80
% ==================================================================
\begin{question}{80}{conjuntos}{1}{b}{2}{
\\ Considere $A=$ estudiantes que asisten a clase de cálculo y $B=$ estudiantes que asisten a clase de álgebra. \textbf{Traduzca:} $(A \setminus B) = \emptyset$
\begin{enumerate}
    \item a) Hay estudiantes que asisten solo a cálculo.
    \item b) Todo estudiante que asiste a cálculo también asiste a álgebra.
    \item c) Todo estudiante que asiste a álgebra también asiste a cálculo.
    \item d) No hay estudiantes que asistan a ambas clases.
\end{enumerate}
}
\end{question}
% ==================================================================
% PREGUNTA 81
% ==================================================================
\begin{question}{81}{conjuntos}{2}{a}{2}{
\\ Considere $A=$ personas que tienen membresía en el gimnasio y $B=$ personas que asisten a clases de yoga. \textbf{Traduzca:} $(A \cup B)^c = A^c \cap B^c$
\begin{enumerate}
    \item a) Las personas que no van al gimnasio ni a yoga son las mismas que no van al gimnasio y no van a yoga.
    \item b) las persoans que van a gimansio van a yoga.
    \item c) las personas que van a yoga van a gimnasio.
    \item d) el universo es las personas que van a yoga a gimnasio y hacen pilates.
\end{enumerate}
}
\end{question}
% ==================================================================
% PREGUNTA 82
% ==================================================================
\begin{question}{82}{conjuntos}{1}{c}{2}{
\\ Considere $P=$ profesores de la universidad de los Andes, $E=$ estudiantes de la misma universidad, $A=$ administrativos. \textbf{Traduzca:} $P \cap E = \emptyset$
\begin{enumerate}
    \item a) Hay personas que son profesores y estudiantes a la vez.
    \item b) Los administrativos son también profesores.
    \item c) Nadie es profesor y estudiante a la vez en la universidad de los andes.
    \item d) Todos son profesores o estudiantes.
\end{enumerate}
}
\end{question}
% ==================================================================
% PREGUNTA 83
% ==================================================================
\begin{question}{83}{conjuntos}{1}{d}{2}{
\\ Considere $L=$ libros de matemáticas en la biblioteca y $N=$ novelas en la misma biblioteca. \textbf{Traduzca:} $L \cap N = \emptyset$
\begin{enumerate}
    \item a) Todos los libros son novelas.
    \item b) Algunos libros son de matemáticas y novela a la vez.
    \item c) Ningún libro es de matemáticas.
    \item d) Ningún libro es simultáneamente de matemáticas y novela.
\end{enumerate}
}
\end{question}
% ==================================================================
% PREGUNTA 84
% ==================================================================
\begin{question}{10}{conjuntos}{1}{b}{2}{
\\ Considere $L=$ libros de matemáticas, $N=$ novelas, y $B=$ la colección de la biblioteca general. \textbf{Traduzca:} $(L \cup N) \subseteq B$
\begin{enumerate}
    \item a) Todos los libros de matemáticas y novelas están fuera de la biblioteca.
    \item b) Todos los libros de matemáticas o novelas están en la biblioteca.
    \item c) Ningún libro de matemáticas o novelas está en la biblioteca.
    \item d) Sólo las novelas están en la biblioteca.
\end{enumerate}
}
\end{question}
% ==================================================================
% PREGUNTA 85
% ==================================================================
\begin{question}{16}{conjuntos}{1}{c}{2}{
\\ Considere $A=$ alumnos que aprueban matemáticas y $B=$ alumnos que aprueban física. \textbf{Traduzca:} $A \cap B \neq \emptyset$
\begin{enumerate}
    \item a) Todos los alumnos aprueban ambas materias.
    \item b) Ningún alumno aprueba matemáticas.
    \item c) Al menos un alumno aprueba tanto matemáticas como física.
    \item d) Ningún alumno aprueba ambas materias.
\end{enumerate}
}
\end{question}
% ==================================================================
% PREGUNTA 86
% ==================================================================
\begin{question}{86}{conjuntos}{1}{b}{2}{
\\ Considere $A=$ empleados con contrato fijo, $B=$ empleados que trabajan remoto, $C=$ empleados que reciben bonificación. \textbf{Traduzca:} $(A \cap B^c) \subseteq C$
\begin{enumerate}
\item a) Todo empleado fijo que trabaja remoto recibe bonificación.
\item b) Todo empleado que es fijo y no trabaja remoto, recibe bonificación.
\item c) Ningún empleado fijo que trabaja remoto recibe bonificación.
\item d) Los empleados remotos no pueden tener contrato fijo.
\end{enumerate}
}
\end{question}
% ==================================================================
% PREGUNTA 87
% ==================================================================
\begin{question}{87}{conjuntos}{1}{c}{2}{
\\ Considere $P=$ pacientes vacunados contra gripe, $Q=$ pacientes vacunados contra COVID, $R=$ pacientes con enfermedades respiratorias. \textbf{Traduzca:} $R \subseteq P \cup Q$
\begin{enumerate}
\item a) Todo paciente vacunado contra gripe o COVID tiene una enfermedad respiratoria.
\item b) Ningún paciente con enfermedad respiratoria ha sido vacunado.
\item c) Todo paciente con enfermedad respiratoria ha sido vacunado contra gripe o COVID.
\item d) Algunos pacientes con enfermedad respiratoria no han sido vacunados.
\end{enumerate}
}
\end{question}
% ==================================================================
% PREGUNTA 88
% ==================================================================
\begin{question}{88}{conjuntos}{2}{c}{2}{
\\ Considere $A=$ personas que hablan inglés, $B=$ personas que trabajan en empresas multinacionales, $C=$ personas que han vivido en el extranjero.
\textbf{Traduzca:} $C \subseteq (A \cap B) \cup (A^c \cap B^c)$
\begin{enumerate}
\item a) Toda persona que ha vivido en el extranjero habla inglés y trabaja en una multinacional.
\item b) Si alguien ha vivido en el extranjero, entonces habla inglés o no trabaja en una multinacional.
\item c) Toda persona que ha vivido en el extranjero, o bien habla inglés y trabaja en una multinacional, o bien no cumple ninguna de esas dos condiciones.
\item d) Toda persona que no ha vivido en el extranjero habla inglés y trabaja en una multinacional.
\end{enumerate}
}
\end{question}
% ==================================================================
% PREGUNTA 89
% ==================================================================
\begin{question}{89}{conjuntos}{2}{a}{2}{
\\ Considere $A=$ personas que consumen cafeína diariamente, $B=$ personas con trastornos de sueño, $C=$ personas que practican deporte.
\textbf{Traduzca:} $(A \cap C^c) \cup (B \cap C^c) \subseteq A \cup B$
\begin{enumerate}
\item a) Toda persona que consume cafeína y no practica deporte, o tiene trastornos del sueño y no practica deporte, necesariamente pertenece al grupo que consume cafeína o tiene trastornos del sueño.
\item b) Toda persona que hace deporte no tiene trastornos del sueño ni consume cafeína.
\item c) Las personas que hacen deporte no pueden estar en $A \cup B$.
\item d) Sólo quienes practican deporte están excluidos de $A \cup B$.
\end{enumerate}
}
\end{question}
% ==================================================================
% PREGUNTA 90
% ==================================================================
\begin{question}{90}{conjuntos}{1}{c}{2}{
\\ Considere $A=$ habitantes con acceso a agua potable, $B=$ habitantes con acceso a energía eléctrica, $C=$ habitantes con acceso a internet.
\textbf{Traduzca:} $A \cap (B \cup C)^c$
\begin{enumerate}
\item a) Habitantes con acceso a internet y agua potable, pero sin energía eléctrica.
\item b) Habitantes sin acceso a ningún servicio.
\item c) Habitantes con agua potable pero sin acceso a energía eléctrica ni internet.
\item d) Habitantes con energía eléctrica y acceso a internet, pero sin agua potable.
\end{enumerate}
}
\end{question}
% ==================================================================
% PREGUNTA 91
% ==================================================================
\begin{question}{91}{conjuntos}{2}{d}{2}{
\\ Considere $A =$ usuarios de redes sociales, $B =$ usuarios mayores de 30 años, $C =$ usuarios que comparten noticias. \textbf{Traduzca:} $(A \cap B) \setminus C \neq \emptyset$
\begin{enumerate}
    \item a) Todos los usuarios mayores de 30 años comparten noticias.
    \item b) Ningún usuario de redes sociales mayor de 30 años comparte noticias.
    \item c) Solo los usuarios menores de 30 años comparten noticias.
    \item d) Existe al menos un usuario mayor de 30 años con redes sociales que no comparte noticias.
\end{enumerate}
}
\end{question}

% ==================================================================
% PREGUNTA 92
% ==================================================================
\begin{question}{92}{conjuntos}{2}{b}{2}{
\\ Considere $P =$ pacientes con diabetes, $Q =$ pacientes con hipertensión, $R =$ pacientes que hacen ejercicio regularmente. \textbf{Traduzca:} $P \cup Q \subseteq R^c$
\begin{enumerate}
    \item a) Algunos pacientes con diabetes o hipertensión hacen ejercicio.
    \item b) Todo paciente con diabetes o hipertensión no hace ejercicio regularmente.
    \item c) Todos los pacientes que hacen ejercicio tienen diabetes o hipertensión.
    \item d) Solo los pacientes sin diabetes ni hipertensión hacen ejercicio.
\end{enumerate}
}
\end{question}

% ==================================================================
% PREGUNTA 93
% ==================================================================
\begin{question}{93}{conjuntos}{3}{a}{2}{
\\ Considere $X =$ empleados con maestría, $Y =$ empleados en gerencia, $Z =$ empleados con salario mayor a \$5,000. \textbf{Traduzca:} $(X \cap Y) \triangle Z = \emptyset$ (donde $\triangle$ es diferencia simétrica osea $A \triangle B=(A\setminus B) \cup (B \setminus A)$)
\begin{enumerate}
    \item a) Los empleados con maestría y en gerencia son los unicos con salario mayor a \$5000.
    \item b) Los empleados en gerencia nunca tienen maestría.
    \item c) Todos los empleados con salario alto están en gerencia.
    \item d) La maestría y la gerencia son excluyentes para salarios altos.
\end{enumerate}
}
\end{question}

% ==================================================================
% PREGUNTA 94
% ==================================================================
\begin{question}{94}{conjuntos}{3}{c}{2}{
\\ Considere $L =$ estudiantes de literatura, $M =$ estudiantes de música, $A =$ estudiantes becados. \textbf{Traduzca:} $A \cap (L \triangle M) = A \cap (L \cup M)$ (donde $\triangle$ es diferencia simétrica osea $A \triangle B=(A\setminus B) \cup (B \setminus A)$)
\begin{enumerate}
    \item a) Los becados estudian exclusivamente literatura o música.
    \item b) Ningún becado estudia ambas disciplinas.
    \item c) Los estudiantes  que estudian literatura o música no estudian ambas simultáneamente.
    \item d) Todos los becados estudian literatura y música.
\end{enumerate}
}
\end{question}

% ==================================================================
% PREGUNTA 95
% ==================================================================
\begin{question}{95}{conjuntos}{1}{b}{2}{
\\ Considere $T =$ turistas europeos, $A =$ turistas que visitan Asia, $L =$ turistas que hablan más de 3 idiomas. \textbf{Traduzca:} $T \cap A \subseteq L^c$
\begin{enumerate}
    \item a) Todos los turistas europeos en Asia hablan muchos idiomas.
    \item b) Ningún turista europeo que visita Asia habla más de 3 idiomas.
    \item c) Los turistas multilingües evitan Asia.
    \item d) Solo los no europeos en Asia son multilingües.
\end{enumerate}
}
\end{question}

% ==================================================================
% PREGUNTA 96
% ==================================================================
\begin{question}{96}{conjuntos}{2}{d}{2}{
\\ Considere $C =$ compradores frecuentes, $D =$ compradores con descuento especial, $P =$ compradores que pagan en efectivo. \textbf{Traduzca:} $(C \setminus D) \cap P^c \neq \emptyset$
\begin{enumerate}
    \item a) Todos los compradores frecuentes tienen descuento.
    \item b) Los compradores con descuento siempre pagan en efectivo.
    \item c) No hay compradores frecuentes sin descuento.
    \item d) Existe al menos un comprador frecuente sin descuento que no paga en efectivo.
\end{enumerate}
}
\end{question}

% ==================================================================
% PREGUNTA 97
% ==================================================================
\begin{question}{97}{conjuntos}{3}{a}{2}{
\\ Considere $E =$ empresas con certificación ambiental, $R =$ empresas que reciclan, $S =$ empresas con más de 100 empleados. \textbf{Traduzca:} $E \subseteq (R \cap S) \cup (R^c \cap S^c)$
\begin{enumerate}
    \item a) Las empresas certificadas o bien reciclan y tienen más de 100 empleados, o bien no reciclan y tienen 100 o menos empleados.
    \item b)Las empresas certificadas o bien reciclan y tienen más de 100 empleados, o bien no reciclan y tienen  menos de 100 empleados. 
    \item c) Las empresas certificadas nunca reciclan.
    \item d) La certificación requiere reciclar y ser grande.
\end{enumerate}
}
\end{question}

% ==================================================================
% PREGUNTA 98
% ==================================================================
\begin{question}{98}{conjuntos}{3}{c}{2}{
\\ Considere $V =$ votantes urbanos, $P =$ votantes con educación superior, $J =$ votantes jóvenes (18-30 años). \\\textbf{Traduzca:} $J \cap (V \triangle P) = \emptyset$ (donde $\triangle$ es diferencia simétrica osea $A \triangle B=(A\setminus B) \cup (B \setminus A)$)
\begin{enumerate}
    \item a) Los jóvenes votantes urbanos no tienen educación superior.
    \item b) Los votantes jóvenes son exclusivamente rurales.
    \item c) Para votantes jóvenes: ser urbano y tener educación superior son condiciones equivalentes (ambas o ninguna).
    \item d) La educación superior excluye a votantes urbanos jóvenes.
\end{enumerate}
}
\end{question}

% ==================================================================
% PREGUNTA 99
% ==================================================================
\begin{question}{99}{conjuntos}{2}{b}{2}{
\\ Considere $F =$ familias con hijos, $M =$ familias con mascotas, $C =$ familias en condominio. \textbf{Traduzca:} $(F \cap M) \setminus C \supsetneq \emptyset$
\begin{enumerate}
    \item a) Todas las familias con hijos y mascotas viven en condominio.
    \item b) Existe al menos una familia con hijos y mascotas que no vive en condominio.
    \item c) Las mascotas son exclusivas de familias sin hijos.
    \item d) Los condominios no permiten mascotas.
\end{enumerate}
}
\end{question}

% ==================================================================
% PREGUNTA 100
% ==================================================================
4\begin{question}{100}{conjuntos}{3}{a}{2}{
\\ Considere $T =$ trabajadores con horario flexible, $H =$ trabajadores que usan home office, $P =$ trabajadores con productividad alta. \textbf{Traduzca:} $P \cap (T \cup H)^c = \emptyset$
\begin{enumerate}
    \item a)No hay trabajadores productivos que no tengan horario flexible y no usen home office.
    \item b) La productividad es independiente del horario.
    \item c) Solo los trabajadores sin horario flexible son productivos.
    \item d) El home office reduce la productividad.
\end{enumerate}
}
\end{question}
% ==================================================================
% PREGUNTA 101
% ==================================================================
\begin{question}{101}{conjuntos}{1}{a}{2}{
Sea $A = \{1,2,3\}$ y $B = \{2,3,4\}$. \textbf{¿Cuál es $A \cap B$?}
\begin{enumerate}
    \item a) $\{2,3\}$.
    \item b) $\{1,2,3,4\}$.
    \item c) $\{1,4\}$.
    \item d) $\emptyset$.
\end{enumerate}
}
\end{question}

% ==================================================================
% PREGUNTA 102
% ==================================================================
\begin{question}{102}{conjuntos}{1}{b}{2}{
Sea $U = \{1,2,3,4,5\}$ el conjunto universal y $A = \{1,2\}$. \textbf{¿Cuál es $A^c$?}
\begin{enumerate}
    \item a) $\{1,2\}$.
    \item b) $\{3,4,5\}$.
    \item c) $\{1,2,3,4,5\}$.
    \item d) $\emptyset$.
\end{enumerate}
}
\end{question}

% ==================================================================
% PREGUNTA 103
% ==================================================================
\begin{question}{103}{conjuntos}{1}{c}{2}{
Si $A = \{a,b,c\}$ y $B = \{c,d,e\}$, \textbf{¿Cuál es $A \cup B$?}
\begin{enumerate}
    \item a) $\{c\}$.
    \item b) $\{a,b,d,e\}$.
    \item c) $\{a,b,c,d,e\}$.
    \item d) $\{a,b\}$.
\end{enumerate}
}
\end{question}

% ==================================================================
% PREGUNTA 104
% ==================================================================
\begin{question}{104}{conjuntos}{1}{d}{2}{
Dados $A = \{1,2\}$ y $B = \{2,3\}$, \textbf{¿Cuál es $A \setminus B$?}
\begin{enumerate}
    \item a) $\{2\}$.
    \item b) $\{1,2,3\}$.
    \item c) $\{3\}$.
    \item d) $\{1\}$.
\end{enumerate}
}
\end{question}

% ==================================================================
% PREGUNTA 105
% ==================================================================
\begin{question}{105}{conjuntos}{1}{a}{2}{
Si $P = \{x \mid x \text{ es par y } 1 \leq x \leq 10\}$, \textbf{¿Cuál es $P$?}
\begin{enumerate}
    \item a) $\{2,4,6,8,10\}$.
    \item b) $\{1,3,5,7,9\}$.
    \item c) $\{1,2,3,4,5,6,7,8,9,10\}$.
    \item d) $\emptyset$.
\end{enumerate}
}
\end{question}

% ==================================================================
% PREGUNTA 106
% ==================================================================
\begin{question}{106}{lógica, proposiciones}{2}{a}{3}{
Sea $P = $ es un número primo. \textbf{ El numero 4 cumple $\neg P$?}
\begin{enumerate}
    \item a) Verdadero.
    \item b) Falso.
    \end{enumerate}
}
\end{question}

% ==================================================================
% PREGUNTA 107
% ==================================================================
\begin{question}{107}{lógica}{1}{c}{3}{
\\ Si $Q$ es está lloviendo y $R$ es el suelo está mojado, \textbf{¿Cómo se expresa Si está lloviendo, entonces el suelo está mojado? \\ Tenga en cuenta que $\rightarrow$ corresponde a la implicacion y se puede tomar como un entonces o consecuencia tipo $A\rightarrow B$, sería A implica B o si A entonces B.}
\begin{enumerate}
    \item a) $Q \land R$.
    \item b) $Q \lor R$.
    \item c) $Q \rightarrow R$.
    \item d) $\neg Q$.
\end{enumerate}
}
\end{question}

% ==================================================================
% PREGUNTA 108
% ==================================================================
\begin{question}{108}{lógica, proposiciones2  }{1}{a}{3}{
Dado $P$: 3 es mayor que 5 y $Q$: 2 es par, \textbf{¿Cuál es el valor de verdad de $P \lor Q$?}
\begin{enumerate}
    \item a) Verdadero.
    \item b) Falso.
    \item c) No se puede determinar.
    \item d) Ninguna de las anteriores.
\end{enumerate}}
\end{question}

% ==================================================================
% PREGUNTA 109
% ==================================================================
\begin{question}{109}{conjuntos}{1}{d}{2}{
\\ En una universidad hay 100 estudiantes: 60 estudian Matemáticas, 40 estudian Física y 20 estudian ambas. \textbf{¿Cuántos estudian solo Matemáticas?}
\begin{enumerate}
    \item a) 20.
    \item b) 40.
    \item c) 60.
    \item d) 40.
\end{enumerate}
}
\end{question}

% ==================================================================
% PREGUNTA 110
% ==================================================================
\begin{question}{110}{conjuntos}{1}{b}{2}{
\\ En una universidad hay 100 estudiantes: 60 estudian Matemáticas, 40 estudian Física y 20 estudian ambas. \textbf{¿Cuántos estudiantes no estudian ninguna de las dos materias?}
\begin{enumerate}
    \item a) 0.
    \item b) 20.
    \item c) 40.
    \item d) 60.
\end{enumerate}
}
\end{question}

% ==================================================================
% PREGUNTA 111
% ==================================================================
\begin{question}{111}{conjuntos}{1}{c}{2}{
\\Si $A \subseteq B$, \textbf{¿Cuál de las siguientes afirmaciones es siempre verdadera?}
\begin{enumerate}
    \item a) $A \cap B = \emptyset$.
    \item b) $A \cup B = \emptyset$.
    \item c) $A \cap B = A$.
    \item d) $A \setminus B = A$.
\end{enumerate}
}
\end{question}

% ==================================================================
% PREGUNTA 112
% ==================================================================
\begin{question}{112}{conjuntos}{1}{a}{2}{
\\Dado $A = \{1,2\}$, \textbf{¿Cuántos elementos tiene $\mathcal{P}(A)$ (el conjunto potencia de A)?}
\begin{enumerate}
    \item a) 4.
    \item b) 2.
    \item c) 3.
    \item d) 1.
\end{enumerate}
}
\end{question}

% ==================================================================
% PREGUNTA 113
% ==================================================================
\begin{question}{113}{lógica}{1}{d}{3}{
Sea $P$: Hace calor y $Q$: Es verano. \textbf{¿Cómo se expresa No es verano?}
\begin{enumerate}
    \item a) $P \land Q$.
    \item b) $P \lor Q$.
    \item c) $P \rightarrow Q$.
    \item d) $\neg Q$.
\end{enumerate}
}
\end{question}

% ==================================================================
% PREGUNTA 114
% ==================================================================
\begin{question}{114}{lógica}{1}{b}{3}{
\\Si $P$ es verdadero y $Q$ es falso, \textbf{¿cuál es el valor de $P \land Q$?}
\begin{enumerate}
    \item a) Verdadero.
    \item b) Falso.
    \item c) No se puede determinar.
    \item d) Ninguna de las anteriores.
\end{enumerate}
}
\end{question}

% ==================================================================
% PREGUNTA 115
% ==================================================================
\begin{question}{115}{conjuntos}{1}{c}{2}{
\\ En un grupo de 50 personas: 30 hablan inglés, 20 hablan francés y 10 hablan ambos idiomas. \textbf{¿Cuántas personas hablan solo inglés?}
\begin{enumerate}
    \item a) 10.
    \item b) 20.
    \item c) 20.
    \item d) 30.
\end{enumerate}
}
\end{question}

% ==================================================================
% PREGUNTA 116% ==================================================================
\begin{question}{116}{conjuntos}{1}{a}{2}{
\\En un grupo de 50 personas: 30 hablan inglés, 20 hablan francés y 10 hablan ambos idiomas.  \textbf{¿Cuántas personas no hablan ninguno de estos idiomas?}
\begin{enumerate}
    \item a) 10.
    \item b) 20.
    \item c) 30.
    \item d) 40.
\end{enumerate}
}
\end{question}

% ==================================================================
% PREGUNTA 117
% ==================================================================
\begin{question}{117}{conjuntos}{1}{b}{2}{
\\ Si $A = \{1,2,3\}$ y $B = \{3,4,5\}$, \textbf{¿Cuál es $A \times B$?}
\begin{enumerate}
    \item a) $\{(1,3),(2,4),(3,5)\}$.
    \item b) $\{(1,3),(1,4),(1,5),(2,3),(2,4),(2,5),(3,3),(3,4),(3,5)\}$.
    \item c) $\{3\}$.
    \item d) $\{1,2,3,4,5\}$.
\end{enumerate}
}
\end{question}

% ==================================================================
% PREGUNTA 118
% ==================================================================
\begin{question}{118}{lógica, proposiciones}{1}{c}{3}{
\\ Si $P$ es falso y $Q$ es verdadero, \textbf{¿Cuál es el valor de $(\neg P) \lor Q$?}
\begin{enumerate}
    \item a) Falso.
    \item b) No se puede determinar.
    \item c) Verdadero.
    \item d) Ninguna de las anteriores.
\end{enumerate}
}
\end{question}

% ==================================================================
% PREGUNTA 119
% ==================================================================
\begin{question}{119}{conjuntos}{1}{d}{2}{
\\Dados $A = \{a,b\}$ y $B = \{1,2\}$, \textbf{¿Cuál es $B \times A$?}
\begin{enumerate}
    \item a) $\{(a,1),(a,2),(b,1),(b,2)\}$.
    \item b) $\{(a,b),(1,2)\}$.
    \item c) $\{a,b,1,2\}$.
    \item d) $\{(1,a),(1,b),(2,a),(2,b)\}$.
\end{enumerate}
}
\end{question}

% ==================================================================
% PREGUNTA 120
% ==================================================================
\begin{question}{120}{lógica}{2}{a}{3}{
\\Sea $P$: Está soleado y $Q$: Vamos al parque. \textbf{¿Cómo se expresa Si está soleado, entonces vamos al parque? \\Tenga en cuenta que $\rightarrow$ corresponde a la implicacion y se puede tomar como un entonces o consecuencia tipo $A\rightarrow B$, sería A implica B o si A entonces B.}
\begin{enumerate}
    \item a) $P \rightarrow Q$.
    \item b) $P \land Q$.
    \item c) $P \lor Q$.
    \item d) $\neg P$.
\end{enumerate}
}
\end{question}

% ==================================================================
% PREGUNTA 121
% ==================================================================
\begin{question}{121}{conjuntos}{1}{b}{2}{
\\ En una encuesta a 100 personas: 60 leen el periódico A, 40 leen el periódico B y 20 leen ambos. \textbf{¿Cuántos leen solo el periódico A?}
\begin{enumerate}
    \item a) 20.
    \item b) 40.
    \item c) 60.
    \item d) 80.
\end{enumerate}
}
\end{question}

% ==================================================================
% PREGUNTA 122
% ==================================================================
\begin{question}{122}{conjuntos}{1}{c}{2}{
\\ En una encuesta a 100 personas: 60 leen el periódico A, 40 leen el periódico B y 20 leen ambos. \textbf{¿Cuántas personas no leen ningún periódico?}
\begin{enumerate}
    \item a) 0.
    \item b) 10.
    \item c) 20.
    \item d) 40.
\end{enumerate}
}
\end{question}

% ==================================================================
% PREGUNTA 123
% ==================================================================
\begin{question}{123}{lógica}{1}{d}{3}{
\\ Dadas las proposiciones $P$: 2+2=4 y $Q$: 3 es par, \textbf{¿Cuál es el valor de $P \land (\neg Q)$?}
\begin{enumerate}
    \item a) Falso.
    \item b) No se puede determinar.
    \item c) Falso y Verdadero.
    \item d) Verdadero.
\end{enumerate}
}
\end{question}

% ==================================================================
% PREGUNTA 124
% ==================================================================
\begin{question}{124}{conjuntos}{1}{d}{2}{
\\ Si $A = \{1,2,3\}$, \textbf{¿Cuál de los siguientes es subconjunto de A?}
\begin{enumerate}
    \item a) $\{1,2,\emptyset\}$.
    \item b) $\{4\}$.
    \item c) $\{1,2,3,4\}$.
    \item d) $\emptyset$ y $\{1,2\}$.
\end{enumerate}
}
\end{question}

% ==================================================================
% PREGUNTA 125
% ==================================================================
\begin{question}{125}{lógica}{1}{b}{3}{
\\ Sea $P$: Hace frío y $Q$: Es invierno. \textbf{¿Cómo se expresa Hace frío y es invierno?}
\begin{enumerate}
    \item a) $P \lor Q$.
    \item b) $P \land Q$.
    \item c) $P \rightarrow Q$.
    \item d) $\neg P$.
\end{enumerate}
}
\end{question}
% ==================================================================
% PREGUNTA 126
% (Tema: cuantificadores, dif:2, res:b, week:7)
% ==================================================================
\begin{question}{126}{cuantificadores}{2}{b}{6}{
\\ Sea $X$ el conjunto de estudiantes de Matemáticas y $A:$  el conjunto de estudiantes que aprobó Cálculo, $B:$ el conjunto de estudiantes que aprobó Álgebra. \textbf{¿Qué expresa $\exists x \in X$ tal que $(x\in A \land (x\notin B))$?}
\begin{enumerate}
    \item a) Todos aprobaron Cálculo pero no Álgebra.
    \item b) Algún estudiante aprobó Cálculo pero no Álgebra.
    \item c) Nadie aprobó ambos cursos.
    \item d) Quienes aprobaron Cálculo también aprobaron Álgebra.
\end{enumerate}
}
\end{question}

% ==================================================================
% PREGUNTA 127
% (Tema: cuantificadores, dif:2, res:d, week:7)
% ==================================================================
\begin{question}{127}{cuantificadores}{2}{d}{6}{
\\ Para $X$ y $A, B$, Sea $X$ el conjunto de estudiantes de Matemáticas y $A:$  el conjunto de estudiantes que aprobó Cálculo, $B:$ el conjunto de estudiantes que aprobó Álgebra. \textbf{¿Qué significa $\forall x \in X, x\in A \rightarrow x\in B$? \\Tenga en cuenta que $\rightarrow$ corresponde a la implicacion y se puede tomar como un entonces o consecuencia tipo $A\rightarrow B$, sería A implica B o si A entonces B.}
\begin{enumerate}
    \item a) Solo los que aprobaron Álgebra aprobaron Cálculo.
    \item b) Quienes no aprobaron Álgebra tampoco aprobaron Cálculo.
    \item c) Aprobar Cálculo es equivalente a aprobar Álgebra.
    \item d) Todos los estudiantes que aprobaron Cálculo también aprobaron Álgebra.
\end{enumerate}
}
\end{question}

% ==================================================================
% PREGUNTA 128
% (Tema: cuantificadores, dif:3, res:c, week:7)
% ==================================================================
\begin{question}{128}{cuantificadores, funciones}{3}{c}{6}{
\\ Sea $P: $ es el conjunto de tuplas (x,y) donde y es un estudiante y x es profesor de y. \\ ¿Qué expresa $\forall  y\in Est$, $\exists x \in Prof$ tq $ (x,y)\in P$? \\ Prof es el conjunto de profesores y Est el conjuto de estudiantes
\begin{enumerate}
    \item a) Hay un profesor para todos los estudiantes.
    \item b) Ningún profesor tiene estudiantes.
    \item c) Cada estudiante tiene al menos un profesor.
    \item d) Algunos profesores no tienen estudiantes.
\end{enumerate}
}
\end{question}


% ==================================================================
% PREGUNTA 129
% (Tema: cuantificadores, dif:3, res:a, week:7)
% ==================================================================
\begin{question}{129}{cuantificadores}{3}{a}{6}{
\\ Sea $P: $ es el conjunto de tuplas (x,y) donde y es un estudiante y x es profesor de y. \\ \\ \textbf{¿Qué significa $\exists x \in Prof $ tq $\forall y \in Est, (x,y) \in P$? \\Prof es el conjunto de profesores y Est el conjuto de estudiantes}
\begin{enumerate}
    \item a) Existe un profesor que enseña a todos los estudiantes.
    \item b) Todos los profesores enseñan a algún estudiante.
    \item c) Ningún estudiante tiene más de un profesor.
    \item d) Los profesores solo enseñan a un estudiante.
\end{enumerate}
}
\end{question}

% ==================================================================
% PREGUNTA 130
% (Tema: lógica, dif:3, res:b, week:8)
% ==================================================================
\begin{question}{130}{lógica, funciones}{1}{b}{6}{
\\Sea $X$: El conjunto de computadores,  $D:$  el conjunto de computadores dañado, $L:$ el conjunto de computadores en la biblioteca. \\ \textbf{¿Qué expresa $\exists x \in X$ tq $x\in D \land x\in L$?}
\begin{enumerate}
    \item a) Todos los computadores en la biblioteca están dañados.
    \item b) Hay al menos un computador dañado en la biblioteca.
    \item c) Los computadores dañados no están en la biblioteca.
    \item d) Ningún computador en la biblioteca funciona.
\end{enumerate}
}
\end{question}

% ==================================================================
% PREGUNTA 131
% (Tema: lógica, dif:3, res:d, week:8)
% ==================================================================
\begin{question}{131}{lógica}{2}{d}{6}{
\\Sea $X$: El conjunto de computadores,  $D:$  el conjunto de computadores dañado, $L:$ el conjunto de computadores en la biblioteca.  \\ \textbf{¿Qué significa $\forall x \in X,  x\in L\leftrightarrow x\in D$?}
\begin{enumerate}
    \item a) Los computadores en la biblioteca no están dañados.
    \item b) Solo los computadores dañados están en la biblioteca.
    \item c) Todos los computadores están dañados o no existen.
    \item d) Todo computador esta en la biblioteca si y solo si esta dañado.
\end{enumerate}
}
\end{question}

% ==================================================================
% PREGUNTA 132
% (Tema: cuantificadores, dif:3, res:c, week:8)
% ==================================================================
\begin{question}{132}{cuantificadores}{1}{c}{6}{
\\ Sea $R$: el conjunto de residentes, $E$ el conjunto de estudiantes de medicina, $P:$ el conjunto de estudiantes que paga matrícula completa. \\ \textbf{¿Qué expresa $\forall x \in E, x\in R \leftrightarrow x\in P$?}
\begin{enumerate}
    \item a) Solo los residentes pagan matrícula.
    \item b) Ningún estudiante residente paga matrícula.
    \item c) Un estudiante de medicina es residentes si y solo si pagan matricula completa.
    \item d) Los no residentes no pagan matrícula.
\end{enumerate}
}
\end{question}

% ==================================================================
% PREGUNTA 133
% (Tema: lógica, dif:2, res:a, week:8)
% ==================================================================
\begin{question}{133}{lógica}{1}{a}{6}{
\\Sea $A:$ Es el conjunto de salones de la universidad de los andes, $O:$  el conjunto de salones ocupados, $C:$ el conjunto de salones con computador. \\ \textbf{¿Qué expresa $\neg( \exists x \in A $ tq $  x\in O \land x\in C)$?}
\begin{enumerate}
    \item a) No hay aulas ocupadas con computador.
    \item b) Todas las aulas ocupadas tienen computador.
    \item c) Algunas aulas sin ocupar tienen computador.
    \item d) Las aulas con computador nunca están ocupadas.
\end{enumerate}
}
\end{question}

% ==================================================================
% PREGUNTA 134
% (Tema: lógica, dif:3, res:b, week:9)
% ==================================================================
\begin{question}{134}{lógica}{1}{b}{6}{
\\Defina $M:$ es el conjunto de materias, $D:$ es el conjunto de materias difíciles y $H:$ el conjunto de materias de matematicas. \\\textbf{¿Qué expresa $ \exists x \in M $ tq $ x\in D \land x\in H$?}
\begin{enumerate}
    \item a) No hay materias dificiles de matematicas.
    \item b) Existe al menos una materia dificil que es de matematicas.
    \item c) Todas las materias difíciles son de matematicas.
    \item d) No existen materias dificiles.
\end{enumerate}
}
\end{question}

% ==================================================================
% PREGUNTA 135
% (Tema: lógica, dif:3, res:d, week:9)
% ==================================================================
\begin{question}{135}{lógica}{1}{d}{6}{
\\ Defina $M:$  es el conjunto de materias, $D:$ es el conjunto de materias difíciles. \\ \textbf{¿Qué significa $\exists x \in M $ tq $ x\in D$?}
\begin{enumerate}
    \item a) Todas las materias son dificiles.
    \item b) Hay materias que no son dificiles.
    \item c) Solo hay materias dificiles.
    \item d) Existe al menos una materia difícil.
\end{enumerate}
}
\end{question}

% ==================================================================
% PREGUNTA 136
% (Tema: cuantificadores, dif:3, res:a, week:9)
% ==================================================================
\begin{question}{136}{cuantificadores}{2}{a}{6}{
\\Sea $P: $el conjunto de profesores, $E: $el conjunto de estudiantes, $T: $es el conjunto de tuplas (x,y) donde x es profesor del estudiante y. \\\textbf{¿Qué expresa $\forall x \in E,  \exists y \in P $ tq $(y,x) \in T $?}
\begin{enumerate}
    \item a) Todo estudiante tiene al menos un profesor.
    \item b) Hay profesores que enseñan a todos.
    \item c) Los profesores solo enseñan a estudiantes.
    \item d) Ningún estudiante tiene profesor.
\end{enumerate}
}
\end{question}

% ==================================================================
% PREGUNTA 137
% (Tema: lógica, dif:3, res:c, week:9)
% ==================================================================
\begin{question}{137}{lógica}{1}{c}{6}{
\\Defina $S:$ es el conjunto de salas de estudios, $C:$ es el conjunto de salas de estudio que tienen computadores, $R:$ es el conjunto de salas de estudio que están reservada. \\\textbf{¿Qué expresa $\forall x \in S, x\in C \leftrightarrow x\in R^c$? \\ Tenga en cuenta que $\leftrightarrow$ corresponde al si y solo si y se puede interpretar $A \leftrightarrow B$ como: A ocurre si y solo si B ocurre}
\begin{enumerate}
    \item a) Las salas con computadores no existen.
    \item b) Solo las salas reservadas tienen computadores.
    \item c) Una sala de estudio tiene computadores si y solo si no está reservada.
    \item d) Las salas sin computadores están reservadas.
\end{enumerate}
}
\end{question}

% ==================================================================
% PREGUNTA 138
% (Tema: lógica, dif:3, res:b, week:10)
% ==================================================================
\begin{question}{138}{lógica}{1}{d}{6}{
\\Sea $A:$ es el conjunto de artículos, $P:$ es el conjunto de articulos disponibles, $D:$ es el conjunto de articulos de comida. \\\textbf{¿Qué expresa $\forall x \in A, x\in P \lor x\in D$?}
\begin{enumerate}
    \item a) Los artículos siempre estan disponibles.
    \item b) Todo artículo está disponible o no es de comida.
    \item c) Ningún artículo está disponible.
    \item d) Para todos los articulos se cumple que estan disponibles o son de comida.
\end{enumerate}
}
\end{question}

% ==================================================================
% PREGUNTA 139
% (Tema: cuantificadores, dif:3, res:d, week:10)
% ==================================================================
\begin{question}{139}{cuantificadores}{1}{d}{6}{
\\Defina $V:$ es el conjunto de cursos vacacionales, $C:$ es el conjunto de cursos, $M:$ es el conjunto de cursos que tiene matrícula. \\\textbf{¿Qué expresa $\exists x \in C $ tq  $x\in V \land x\in M $?}
\begin{enumerate}
    \item a) Ningún curso vacacional tiene matrícula.
    \item b) Todos los cursos tienen matrícula.
    \item c) Los cursos no vacacionales no tienen matrícula.
    \item d) Existe un curso que es vacacional y tiene matricula.
\end{enumerate}
}
\end{question}

% ==================================================================
% PREGUNTA 140
% (Tema: lógica, dif:3, res:a, week:10)
% ==================================================================
\begin{question}{140}{lógica}{2}{a}{6}{
\\Sea $E:$ el conjunto de examenes, $D:$ es el conjunto de examenes difíciles, $A:$ es el conjunto de examenes aprobado. \\\textbf{¿Qué expresa $\forall x \in E,x\in D \leftrightarrow x\in A^c$?}
\begin{enumerate}
    \item a) Para todos los examenes se cumple que es dificil si y solo si no es aprobado.
    \item b) Los exámenes aprobados son difíciles.
    \item c) No hay exámenes difíciles.
    \item d) Todos los exámenes son aprobados.
\end{enumerate}
}
\end{question}

% ==================================================================
% PREGUNTA 141
% (Tema: lógica, dif:3, res:c, week:10)
% ==================================================================
\begin{question}{141}{lógica}{2}{c}{6}{
\\Defina $P:$ es el conjunto de parqueaderos, $O:$ es el conjunto de parqueaderos ocupados, $G:$ es el conjunto de parqueaderos gratuitos. \\\textbf{¿Qué expresa $(\exists x \in P$ tq $ x\in O) \land (\forall y \in P, y\in G \leftrightarrow y\in O^c)$?}
\begin{enumerate}
    \item a) Todos los parqueaderos están ocupados.
    \item b) Los parqueaderos gratuitos no existen.
    \item c) Hay al menos un parqueadero ocupado y para todos los parqueaderos se cumple que son gratuitos si y solo si no estan ocupados.
    \item d) Solo los parqueaderos pagos están ocupados.
\end{enumerate}
}
\end{question}

% ==================================================================
% PREGUNTA 142
% (Tema: cuantificadores, dif:3, res:b, week:11)
% ==================================================================
\begin{question}{142}{cuantificadores}{1}{b}{6}{
\\Sea $B:$ es el conjunto de bibliotecas, $L:$ es el conjunto de bibliotecas quemtiene libros, $A:$ es el conjunto de bibliotecas abiertas. \\\textbf{¿Qué expresa $\forall x\in B, x\in  L \land x\in A$?}
\begin{enumerate}
    \item a) Algunas bibliotecas tienen libros y están abiertas.
    \item b) Todas las bibliotecas tienen libros y están abiertas.
    \item c) Solo las bibliotecas abiertas tienen libros.
    \item d) Las bibliotecas sin libros están cerradas.
\end{enumerate}
}
\end{question}

% ==================================================================
% PREGUNTA 143
% (Tema: lógica, dif:3, res:d, week:11)
% ==================================================================
\begin{question}{143}{lógica}{1}{d}{6}{\\ Defina $C:$ es el conjunto de cafés, $F:$ es el conjunto de cafés fuertes, $G:$ es el conjunto de cafés gourmet. \\ \textbf{¿Qué expresa $(\exists x \in C$ tq $ x\in F) \land (\forall y\in C, y\in G \leftrightarrow y\in F^c)$?}
\begin{enumerate}
    \item a) Todos los cafés son fuertes.
    \item b) Los cafés gourmet no existen.
    \item c) Ningún café fuerte es gourmet.
    \item d) Existe al menos un café fuerte; y, para todos los cafés se cumple que es gourmet si y solo si no es fuerte.
\end{enumerate}
}
\end{question}

% ==================================================================
% PREGUNTA 144
% (Tema: lógica, dif:3, res:a, week:11)
% ==================================================================
\begin{question}{144}{lógica}{1}{a}{6}{\\ Sea $L:$ es el conjunto de laboratorios, $Q:$ es el conjunto de laboratorios que tienen equipos, $R:$ es el conjunto de laboratorios que requiere reserva. \\\textbf{¿Qué expresa $\forall x \in L,   x\in Q \lor x\in R$?}
\begin{enumerate}
    \item a) Todo laboratorio tiene equipos o requiere reserva.
    \item b) Los laboratorios sin equipos no requieren reserva.
    \item c) Solo los laboratorios con equipos existen.
    \item d) Los laboratorios requieren equipos y reserva.
\end{enumerate}
}
\end{question}

% ==================================================================
% PREGUNTA 145
% (Tema: lógica, dif:3, res:c, week:11)
% ==================================================================
\begin{question}{145}{lógica}{1}{c}{6}{\\Defina $E:$ es el conjunto de estudiantes, $B:$ es el conjunto de estudiantes becados, $T:$ es el conjunto de estuidantes que trabaja. \\\textbf{¿Qué expresa $\forall x\in E, x\in B \leftrightarrow  x\in T^c$?}
\begin{enumerate}
    \item a) Los estudiantes becados no existen.
    \item b) Todos los estudiantes trabajan.
    \item c) Un estudiante es becado si y solo si no trabaja.
    \item d) Los estudiantes no becados trabajan.
\end{enumerate}
}
\end{question}

% ==================================================================
% PREGUNTA 146
% (Tema: cuantificadores, dif:3, res:b, week:12)
% ==================================================================
\begin{question}{146}{cuantificadores}{1}{b}{6}{
\\ Sea $P:$ es el conjunto de profesores, $C:$ es el conjunto de profesores catedráticos, $I:$ es el conjunto de profesores que investiga. \\\textbf{¿Qué expresa $\forall x \in P, x\in C \lor x\in I$?}
\begin{enumerate}
    \item a) Los profesores no son catedráticos.
    \item b) Todo profesor es catedrático o investiga.
    \item c) Solo los catedráticos investigan.
    \item d) Los investigadores no son profesores.
\end{enumerate}
}
\end{question}

% ==================================================================
% PREGUNTA 147
% (Tema: lógica, dif:3, res:d, week:12)
% ==================================================================
\begin{question}{147}{lógica}{1}{d}{6}{
\\Defina $A:$ es el conjunto de aulas, $M:$ es el conjunto de aulas con multimedia, $V:$ es el conjunto de aulas que está ventilada. \\\textbf{¿Qué expresa $(\exists x \in A $ tq $ x\in M) \land (\forall y \in A,  y\in V \leftrightarrow y\in M)$?}
\begin{enumerate}
    \item a) Las aulas ventiladas no existen.
    \item b) Todas las aulas tienen multimedia.
    \item c) Las aulas sin multimedia no están ventiladas.
    \item d) Hay aulas con multimedia y se cumple que para todas tienen multimedia si y solo si son ventiladas.
\end{enumerate}
}
\end{question}

% ==================================================================
% PREGUNTA 148
% (Tema: lógica, dif:3, res:a, week:12)
% ==================================================================
\begin{question}{148}{lógica}{1}{a}{6}{
\\Defina $D:$ conjunto de departamentos, $F:$ conjunto de departamentos de física, $G:$ conjunto de departamentos que otorgan grados. \\\textbf{¿Qué expresa $\forall x \in D,\ (x \in F \land x \in G)$?}
\begin{enumerate}
    \item a) Todos los departamentos son de fisica y otorgan grados.
    \item b) Solo los departamentos con grados son de física.
    \item c) Ningún departamento otorga grados.
    \item d) Los departamentos sin grados no son de física.
\end{enumerate}
}
\end{question}


% ==================================================================
% PREGUNTA 149
% (Tema: lógica, dif:3, res:c, week:12)
% ==================================================================
\begin{question}{149}{lógica}{1}{c}{6}{
\\Defina $R:$ conjunto de restaurantes, $U:$ conjunto de restaurantes universitarios, $A:$ conjunto de restaurantes que aceptan carné. \\\textbf{¿Qué expresa $\forall x \in R,\ (x \in U \leftrightarrow x \in A)$?}
\begin{enumerate}
    \item a) Los restaurantes universitarios no aceptan carné.
    \item b) Solo los no universitarios aceptan carné.
    \item c) Para todos los retsaurantes se cumple que es universitario si y solo si aceptan carnet.
    \item d) Los restaurantes no universitarios no existen.
\end{enumerate}
}
\end{question}


% ==================================================================
% PREGUNTA 150
% (Tema: cuantificadores, dif:3, res:b, week:13)
% ==================================================================
\begin{question}{150}{cuantificadores}{2}{b}{6}{
\\Defina $E:$ conjunto de estudiantes, $I:$ conjunto de estudiantes internacionales, $B:$ conjunto de estudiantes con beca. \\\textbf{¿Qué expresa $\forall x \in E,\ (x \in I \leftrightarrow x \in B)$?}
\begin{enumerate}
    \item a) Algunos estudiantes internacionales tienen beca.
    \item b) Para todos los estudiantes se cumple que son internacionales si y solo si tienen beca.
    \item c) Los estudiantes con beca son internacionales.
    \item d) Los internacionales  tienen beca.
\end{enumerate}
}
\end{question}

% ==================================================================
% PREGUNTA 151
% 
% ==================================================================

\begin{question}{151}{índices}{1}{a}{7}{
\\Sea $A_i = \{x \in \mathbb{N} \mid x \leq i\}$ para $i \in \{1,2,3\}$. \textbf{¿Cuál es $\bigcup_{i=1}^3 A_i$?}
\begin{enumerate}
    \item a) $\{1,2,3\}$
    \item b) $\{1,2\}$
    \item c) $\{3\}$
    \item d) $\emptyset$
\end{enumerate}
}
\end{question}
% ==================================================================
% PREGUNTA 152
% 
% ==================================================================
\begin{question}{152}{índices}{1}{c}{7}{
\\Si $B_j = \{j, j+1\}$ para $j \in \{1,3,5\}$, \textbf{¿Cuál es $\bigcap_{j \in \{1,3\}} B_j$?}
\begin{enumerate}
    \item a) $\{1,3\}$
    \item b) $\{2,4\}$
    \item c) $\emptyset$
    \item d) $\{1,2,3,4\}$
\end{enumerate}
}
\end{question}
% ==================================================================
% PREGUNTA 153
% 
% ==================================================================
\begin{question}{153}{índices}{1}{a}{7}{
\\Verdadero o Falso: Para $C_k = \{k^2\}$ con $k \in \{1,2,3\}$, $\bigcup_{k=1}^3 C_k = \{1,4,9\}$.
\begin{enumerate}
    \item a) Verdadero
    \item b) Falso
\end{enumerate}
}
\end{question}
% ==================================================================
% PREGUNTA 154
% 
% ==================================================================
\begin{question}{154}{argumentación}{1}{a}{5}{
\\En una demostración directa de $P \rightarrow Q$, \textbf{¿Qué se asume inicialmente?}
\begin{enumerate}
    \item a) $P$ es verdadero
    \item b) $Q$ es verdadero
    \item c) $P$ es falso
    \item d) $P$ es verdadero y $Q$ es falso
\end{enumerate}
}
\end{question}
% ==================================================================
% PREGUNTA 155
% 
% ==================================================================
\begin{question}{155}{argumentación}{1}{a}{5}{
\\Verdadero o Falso: En una demostración directa de $A \subseteq B$, se toma un elemento arbitrario $x \in A$ y se muestra que $x \in B$.
\begin{enumerate}
    \item a) Verdadero
    \item b) Falso
\end{enumerate}
}
\end{question}
% ==================================================================
% PREGUNTA 156
% 
% ==================================================================
\begin{question}{156}{argumentación}{3}{c}{7}{
\\Si queremos demostrar que no $\exists x\in U $ tq $x\in P$ , \textbf{¿Qué técnica usaríamos?}
\begin{enumerate}
    \item a) Contraejemplo
    \item b) Inducción
    \item c) Demostrar $\forall x \in U, (x\notin P)$
    \item d) Reducción al absurdo
\end{enumerate}
}
\end{question}
% ==================================================================
% PREGUNTA 157
% 
% ==================================================================
\begin{question}{157}{demostraciones}{1}{b}{5}{
\\Para demostrar que $A \cap (B \cup C) = (A \cap B) \cup (A \cap C)$, \textbf{¿Qué técnica es más apropiada para una demostración directa?}
\begin{enumerate}
    \item a) Contradicción
    \item b) Doble inclusión
    \item c) Contraejemplo
    \item d) Inducción
\end{enumerate}
}
\end{question}
% ==================================================================
% PREGUNTA 158
% 
% ==================================================================
\begin{question}{158}{demostraciones}{1}{a}{5}{
\\Verdadero o Falso: Para demostrar que dos conjuntos no son iguales, basta encontrar un elemento que pertenezca a uno pero no al otro.
\begin{enumerate}
    \item a) Verdadero
    \item b) Falso
\end{enumerate}
}
\end{question}
% ==================================================================
% PREGUNTA 159
% 
% ==================================================================
\begin{question}{159}{demostraciones}{1}{d}{5}{
\\En la demostración de que $A \setminus B = A \cap B^c$, \textbf{¿Qué paso lógico es esencial, para una demostracion directa?}
\begin{enumerate}
    \item a) Usar inducción
    \item b) Asumir lo contrario
    \item c) Dividir en casos
    \item d) Mostrar que ambos lados contienen los mismos elementos
\end{enumerate}
}
\end{question}
% ==================================================================
% PREGUNTA 160
% 
% ==================================================================
\begin{question}{160}{índices, argumentación}{1}{c}{7}{
\\Para $n \in \mathbb{N} $ sea $D_n = \{x \in \mathbb{Z} \mid -n \leq x \leq n\}$. \textbf{¿Cómo se expresaría Existe un conjunto $D_n$ que contiene al 10?}
\begin{enumerate}
    \item a) $\forall n \in \mathbb{N}, 10 \in D_n$
    \item b) $\bigcap_{n=1}^\infty D_n$ contiene 10
    \item c) $\exists n \in \mathbb{N}$ tal que $10 \in D_n$
    \item d) $D_{10} \subseteq D_n$ para algún $n$
\end{enumerate}
}
\end{question}
% ==================================================================
% PREGUNTA 161
% ==================================================================
\begin{question}{161}{conjuntos}{1}{b}{2}{
\\Un conjunto $B$ se llama \textbf{intermedio} entre $A$ y $C$ si cumple:
\begin{itemize}
    \item (1) $|A \cap B| \geq |C \cap A|$
    \item (2) $|C \cap B| \geq |C \cap A|$
    \item (3) $|A \cup C| \geq |B|$.
\end{itemize}
Si $A = \{1,2,3\}$, $C = \{3,4,5\}$, ¿cuál de los siguientes es un conjunto intermedio?
\begin{enumerate}
    \item a) $\{1\}$ 
    \item b) $\{3,6\}$ 
    \item c) $\{1,2,3,4,5,6\}$ 
    \item d) $\emptyset$ 
\end{enumerate}
}
\end{question}

% ==================================================================
% PREGUNTA 162
% ==================================================================
\begin{question}{162}{lógica}{2}{a}{2}{
\\Decimos que $B'$ está \textbf{mejor cimentado} que $B$ (entre $A$ y $C$) si:
\begin{itemize}
    \item (1) $|A \cap B'| \geq |A \cap B|$ y $|C \cap B'| \geq |C \cap B|$.
    \item (2) Al menos una desigualdad es estricta. 
\end{itemize}
Si $A = \{1,2\}$, $C = \{2,3\}$, $B = \{2\}$, $B' = \{2,4,3\}$, entonces:
\begin{enumerate}
    \item a) $B'$ está mejor cimentado que $B$
    \item b) $B$ está mejor cimentado que $B'$
    \item c) Ninguno es mejor cimentado
    \item d) No se puede determinar
\end{enumerate}
}
\end{question}

% ==================================================================
% PREGUNTA 163
% ==================================================================
\begin{question}{163}{demostraciones}{3}{c}{5}{
\\En una demostración por \textbf{contradicción} de que $A \subseteq B$, el paso inicial es:
\begin{enumerate}
    \item a) Tomar $x \in A$ arbitrario
    \item b) Mostrar que $B \subseteq A$
    \item c) Asumir que existe $x \in A$ con $x \notin B$
    \item d) Probar que $|A| \leq |B|$
\end{enumerate}
}
\end{question}

% ==================================================================
% PREGUNTA 164
% ==================================================================
\begin{question}{164}{índices}{1}{d}{7}{
\\Dado $ i\in \mathbb{N}$, sea $S_i = \{x \in \mathbb{N} \mid x \text{ es múltiplo de } i\}$. La expresión $\bigcap_{i \in \{2,3\}} S_i$ representa:
\begin{enumerate}
    \item a) Múltiplos de 2
    \item b) Múltiplos de 3
    \item c) Números pares o impares
    \item d) Múltiplos de 6
\end{enumerate}
}
\end{question}

% ==================================================================
% PREGUNTA 165
% ==================================================================
\begin{question}{165}{conjuntos}{1}{b}{5}{
\\Dados conjuntos $A$ y $B$, la \textbf{diferencia} $A \setminus B$ se define como:
\begin{enumerate}
    \item a) $\{x \mid x \in A \lor x \in B\}$
    \item b) $\{x \mid x \in A \land x \notin B\}$
    \item c) $\{x \mid x \notin A \land x \in B\}$
    \item d) $\{x \mid x \notin A \lor x \notin B\}$
\end{enumerate}
}
\end{question}

% ==================================================================
% PREGUNTA 166
% ==================================================================
\begin{question}{166}{demostraciones}{1}{a}{5}{
\\Para demostrar que $A \cap B = \emptyset$, una estrategia válida es:
\begin{enumerate}
    \item a) Mostrar que no existe $x$ tal que $x \in A$ y $x \in B$
    \item b) Probar que $A \subseteq B^c$ y $B \subseteq A^c$
    \item c) Ambas anteriores
    \item d) Ninguna de las anteriores
\end{enumerate}
}
\end{question}

% ==================================================================
% PREGUNTA 167
% ==================================================================
\begin{question}{167}{lógica}{1}{d}{2}{
\\La \textbf{cardinalidad} $|A|$ de un conjunto finito $A$ se define como:
\begin{enumerate}
    \item a) El mayor elemento de $A$
    \item b) La suma de sus elementos
    \item c) El promedio de sus elementos
    \item d) El número de elementos en $A$
\end{enumerate}
}
\end{question}

% ==================================================================
% PREGUNTA 168
% ==================================================================
\begin{question}{168}{argumentación}{1}{c}{5}{
\\En una demostración por \textbf{doble inclusión} de $A = B$, se debe:
\begin{enumerate}
    \item a) Mostrar $A \subseteq B$ o $B \subseteq A$
    \item b) Encontrar un elemento común
    \item c) Probar $A \subseteq B$ y $B \subseteq A$
    \item d) Calcular $|A| = |B|$
\end{enumerate}
}
\end{question}

% ==================================================================
% PREGUNTA 169
% ==================================================================
\begin{question}{169}{conjuntos}{1}{b}{5}{
\\El \textbf{complemento} $A^c$ de un conjunto $A$ respecto al universal $U$ es:
\begin{enumerate}
    \item a) $\{x \mid x \in A\}$
    \item b) $\{x \in U \mid x \notin A\}$
    \item c) $\{x \notin U \mid x \in A\}$
    \item d) $U \setminus U$
\end{enumerate}
}
\end{question}

% ==================================================================
% PREGUNTA 170
% ==================================================================
\begin{question}{170}{índices}{1}{a}{7}{
\\Para familias de conjuntos con $i\in \mathbb{N}$ hasta algun n natural $\{A_i\}_{i=1}^n$, la \textbf{unión indexada} $\bigcup_{i=1}^n A_i$ es:
\begin{enumerate}
    \item a) $\{x \mid \exists i \text{ tal que } x \in A_i\}$
    \item b) $\{x \mid \forall i, x \in A_i\}$
    \item c) El conjunto con más elementos
    \item d) La intersección de todos
\end{enumerate}
}
\end{question}

% ==================================================================
% PREGUNTA 171
% ==================================================================
\begin{question}{171}{demostraciones}{2}{a}{6}{
\\Un \textbf{contraejemplo} para refutar $\forall x \in U, x\in P$ es:
\begin{enumerate}
    \item a) Un $x\in U$ donde $x\notin P$ 
    \item b) Un $x \in U$ donde $x\in P$ 
    \item c) Todos los $x\in U$ donde $x \in P$ 
    \item d) Ningún $x\in U$ cumple $x\in P$
\end{enumerate}
}
\end{question}

% ==================================================================
% PREGUNTA 172
% ==================================================================
\begin{question}{172}{argumentación}{1}{a}{5}{
\\En una demostración  \textbf{directa} $p\rightarrow q$, es necesario asumir que p es verdadero?:
\begin{enumerate}
    \item a) Verdadero
    \item b) Falso
   
\end{enumerate}
}
\end{question}

% ==================================================================
% PREGUNTA 173
% ==================================================================
\begin{question}{173}{conjuntos}{1}{d}{2}{
\\Un conjunto $B$ se llama \textbf{intermedio} entre $A$ y $C$ si cumple:
\begin{itemize}
    \item (1) $|A \cap B| \geq |C \cap A|$
    \item (2) $|C \cap B| \geq |C \cap A|$
    \item (3) $|A \cup C| \geq |B|$.
\end{itemize}Dados $A = \{1,2\}$ y $C = \{2,3\}$, un conjunto \textbf{intermedio} $B$ podría ser:
\begin{enumerate}
    \item a) $\{1\}$ 
    \item b) $\{4\}$ 
    \item c) $\{1,2,3,4\}$ 
    \item d) $\{2,5\}$ 
\end{enumerate}
}
\end{question}

% ==================================================================
% PREGUNTA 174
% ==================================================================
\begin{question}{174}{lógica}{1}{b}{2}{
\\La expresión $A \triangle B$ (\textbf{diferencia simétrica}) se define como:
\begin{enumerate}
    \item a) $(A \cup B) \setminus (A \cap B)$
    \item b) $(A \setminus B) \cup (B \setminus A)$
    \item c) Ambas anteriores
    \item d) Ninguna de las anteriores
\end{enumerate}
}
\end{question}

% ==================================================================
% PREGUNTA 175
% ==================================================================
\begin{question}{175}{demostraciones}{1}{a}{5}{
\\Para demostrar que $A \subseteq B \cup C$, una estrategia válida es:
\begin{enumerate}
    \item a) Tomar $x \in A$ y mostrar $x \in B$ o $x \in C$
    \item b) Probar que $B \subseteq A$ y $C \subseteq A$
    \item c) Mostrar $A \cap B \cap C \neq \emptyset$
    \item d) Calcular $|A| \leq |B| + |C|$
\end{enumerate}
}
\end{question}

% ==================================================================
% PREGUNTA 176
% ==================================================================
\begin{question}{176}{índices}{1}{d}{7}{
\\Dado $k\in \mathbb{N}$ si $T_k = \{k-1,k, k+1,k+2\}$ , entonces $\bigcap_{k=1}^3 T_k$ es:
\begin{enumerate}
    \item a) $\{1,2,3,4\}$
    \item b) $\{2\}$
    \item c) $\emptyset$
    \item d) $\{2,3\}$
\end{enumerate}
}
\end{question}

% ==================================================================
% PREGUNTA 177
% ==================================================================
\begin{question}{177}{argumentación}{3}{d}{5}{
\\En una demostración de $P \Rightarrow Q$, \textbf{no} es válido:
\begin{enumerate}
    \item a) Asumir $P$ y deducir $Q$
    \item b) Usar contradicción
    \item c) Probar $\neg Q \Rightarrow \neg P$
    \item d) Asumir $Q$ es falsa sin usar $P$
\end{enumerate}
}
\end{question}

% ==================================================================
% PREGUNTA 178
% ==================================================================
\begin{question}{178}{conjuntos}{1}{b}{2}{
\\El \textbf{producto cartesiano} $A \times B$ se define como:
\begin{enumerate}
    \item a) $\{a \cup b \mid a \in A, b \in B\}$
    \item b) $\{(a,b) \mid a \in A, b \in B\}$
    \item c) $\{a \cap b \mid a \in A, b \in B\}$
    \item d) $\{a \in A \mid b \in B\}$
\end{enumerate}
}
\end{question}

% ==================================================================
% PREGUNTA 179
% ==================================================================
\begin{question}{179}{lógica}{1}{a}{2}{
\\Para conjuntos $A$ y $B$, la \textbf{igualdad} $A = B$ significa:
\begin{enumerate}
    \item a) $A \subseteq B$ y $B \subseteq A$
    \item b) $|A| = |B|$
    \item c) $A \cap B \neq \emptyset$
    \item d) $A \cup B = \emptyset$
\end{enumerate}
}
\end{question}

% ==================================================================
% PREGUNTA 180
% ==================================================================
\begin{question}{180}{demostraciones}{1}{c}{5}{
\\Al demostrar propiedades de conjuntos, es esencial:
\begin{enumerate}
    \item a) Usar siempre inducción
    \item b) Calcular cardinalidades
    \item c) Trabajar con elementos arbitrarios
    \item d) Evitar definiciones formales
\end{enumerate}
}
\end{question}
% ==================================================================
% PREGUNTA 181 (Índices)
% ==================================================================
\begin{question}{181}{índices}{1}{a}{7}{
\\Dado $i\in \mathbb{N} $ sea \( A_i = \{x \in \mathbb{N} \mid i \leq x\} \). ¿Cuál es \( \bigcup_{i=1}^3 A_i \)?
\begin{enumerate}
    \item a) $\mathbb{N}$
    \item b) \(\{1, 2,4,6,8,...\}\)
    \item c) \(\{3,4,5,6,7,8,9,...\}\)
    \item d) \(\emptyset\)
\end{enumerate}
}
\end{question}

% ==================================================================
% PREGUNTA 182 (Índices)
% ==================================================================
\begin{question}{182}{índices}{2}{a}{7}{
\\Dado \( j \in \{1, 3, 5\} \) si \( B_j = \{j, j+2\} \) , ¿cuál es \( \bigcap_{j \in \{1, 3\}} B_j \)?
\begin{enumerate}
    \item a) \(\{3\}\)
    \item b) \(\{2, 4\}\)
    \item c) \(\emptyset\)
    \item d) \(\{1, 2, 3, 4\}\)
\end{enumerate}
}
\end{question}

% ==================================================================
% PREGUNTA 183 (Índices)
% ==================================================================
\begin{question}{183}{índices}{1}{b}{7}{
\\Verdadero o Falso: Dado \( k \in \{2, 4, 6\} \) si \( C_k = \{2k\} \) , \( \bigcup_{k=1}^3 C_k = \{1, 4, 9\} \).
\begin{enumerate}
    \item a) Verdadero
    \item b) Falso
\end{enumerate}
}
\end{question}

% ==================================================================
% PREGUNTA 184 (Índices)
% ==================================================================
\begin{question}{184}{índices}{2}{d}{7}{
\\Sea $i\in \mathbb{N}$ si \( S_i = \{x \in \mathbb{N} \mid x \text{ es múltiplo de } i\} \). La expresión \( \bigcap_{i \in \{7, 3\}} S_i \) representa:
\begin{enumerate}
    \item a) Múltiplos de 2
    \item b) Múltiplos de 3
    \item c) Números pares o impares
    \item d) Múltiplos de 21
\end{enumerate}
}
\end{question}
% ==================================================================
% PREGUNTA 185 (Índices)
% ==================================================================
\begin{question}{185}{índices}{1}{b}{7}{
\\Sea \( k \in \{1,2,3\} \) y \( B_k = \{k, k+1\} \) . Entonces \( \bigcup_{k=1}^{3} B_k \) es igual a:
\begin{enumerate}
    \item a) \{1,2,3,4,5\}
    \item b) \{1,2,3,4\}
    \item c) \{1,2,3\}
    \item d) \{1,2,3,4,5\}
\end{enumerate}
}
\end{question}

% ==================================================================
% PREGUNTA 186 (Índices)
% ==================================================================
\begin{question}{186}{índices}{1}{c}{7}{
\\Sea \( k \in \{ 1 ,2 ,3\} \) y \( C_k = \{x \in \mathbb{N} \mid x = 2k \} \) . ¿Cuál es la unión \( \bigcup_{k=2}^{3} C_k \)?
\begin{enumerate}
    \item a) \{2\}
    \item b) \{2,4\}
    \item c) \{4,6\}
    \item d) \{1,2,3,4,5,6\}
\end{enumerate}
}
\end{question}

% ==================================================================
% PREGUNTA 187 (Índices)
% ==================================================================
\begin{question}{187}{índices}{1}{a}{7}{\\Sea  \( k \in \{1, 2, 3, 4\} \) y \( D_k = \{x \in \mathbb{N} \mid x = 5k\} \). Entonces, la intersección \( \bigcap_{k=1}^{4} D_k \) es:
\begin{enumerate}
    \item a) $\emptyset$
    \item b) \{20\}
    \item c) \{5,10,15,20\}
    \item d) \{0\}
\end{enumerate}
}
\end{question}

% ==================================================================
% PREGUNTA 188 (Conjuntos)
% ==================================================================
\begin{question}{188}{conjuntos}{1}{c}{4}{\\
Si \( A = \{x \in \mathbb{Z} \mid x > 0\} \) y \( B = \{x \in \mathbb{Z} \mid x \text{ es impar}\} \), ¿qué representa \( A \cap B \)?
\begin{enumerate}
    \item a) Todos los números enteros negativos
    \item b) Todos los números impares
    \item c) Todos los números naturales impares
    \item d) Todos los números naturales pares
\end{enumerate}
}
\end{question}

% ==================================================================
% PREGUNTA 189 (Conjuntos)
% ==================================================================
\begin{question}{189}{conjuntos}{1}{b}{2}{\\
Sean \( A = \{1,2,3\} \), \( B = \{3,4,5\} \). ¿Cuál es \( A \setminus B \)?
\begin{enumerate}
    \item a) \{1,2,3,4,5\}
    \item b) \{1,2\}
    \item c) \{4,5\}
    \item d) \{3\}
\end{enumerate}
}
\end{question}

% ==================================================================
% PREGUNTA 190 (Cuantificadores)
% ==================================================================
\begin{question}{190}{cuantificadores}{1}{a}{7}{\\
Suponga que \( P = \text{Es el conjunto de estudiantes de matemáticas} \) y \( X \) es el conjunto de todos los estudiantes de la universidad.\\ ¿Qué significa \( \forall x \in X, x\in P \)?
\begin{enumerate}
    \item a) Todos los estudiantes de la universidad estudian matemáticas.
    \item b) Algún estudiante de la universidad estudia matemáticas.
    \item c) Ningún estudiante estudia matemáticas.
    \item d) Todos los estudiantes de matemáticas están en la universidad.
\end{enumerate}
}
\end{question}

% ==================================================================
% PREGUNTA 191 (Cuantificadores)
% ==================================================================
\begin{question}{191}{cuantificadores}{1}{c}{7}{\\
Suponga que \( Q = \text{es el conjunto de estudiantes que vive en Bogotá} \), con \( X = \text{estudiantes de Uniandes} \). ¿Qué significa \( \exists x \in X \text{ tal que }x \notin Q \)?
\begin{enumerate}
    \item a) Todos los estudiantes de Uniandes viven en Bogotá.
    \item b) Ningún estudiante vive en Bogotá.
    \item c) Al menos un estudiante de los andes no vive en Bogotá.
    \item d) Todos los estudiantes no viven en Bogotá.
\end{enumerate}
}
\end{question}

% ==================================================================
% PREGUNTA 192 (Conjuntos)
% ==================================================================
\begin{question}{192}{conjuntos}{1}{d}{2}{\\
Sean \( A = \{x \in \mathbb{N} \mid x \text{ divisible por 5} \} \) y \( B = \{x \in \mathbb{N} \mid x \text{ divisible por 3} \} \). ¿Qué representa \( A \cap B \)?
\begin{enumerate}
    \item a) Múltiplos de 3
    \item b) Múltiplos de 2
    \item c) Números primos
    \item d) Múltiplos de 15
\end{enumerate}
}
\end{question}

% ==================================================================
% PREGUNTA 193 (Índices)
% ==================================================================
\begin{question}{193}{índices}{1}{a}{7}{\\
Para \( k \in \{1,2,3\} \) sea \( F_k = \{x \in \mathbb{N} \mid x = k + 10\} \) , $G_k=\{x\in F_k \mid x$ es divisible por $2 \}$. ¿Cuál es la unión \( \bigcup_{k=1}^{3} G_k \)?
\begin{enumerate}
    \item a) \{12\}
    \item b) \{1,2,3\}
    \item c) \{10,11,12,13\}
    \item d) \{10,11,12\}
\end{enumerate}
}
\end{question}
% ==================================================================
% PREGUNTA 194 (Índices)
% ==================================================================
\begin{question}{194}{índices}{3}{c}{7}{\\
Sea \( S_k = \{n \in \mathbb{N} \mid n \equiv 1 \ (\text{mod } k)\} \). ($n \equiv 1  \ (\text{mod } k)$ significa que el residuo de dividir n por k es 1) 
¿Cuál es la intersección \( \bigcap_{k \in \{2,3\}} S_k \)?
\begin{enumerate}
    \item a) \{1, 4, 7, 10, \dots\}
    \item b) \{1, 5, 7, 11, \dots\}
    \item c) \{1, 7, 13, 19, \dots\}
    \item d) \{1, 6, 11, 16, \dots\}
\end{enumerate}
}
\end{question}

% ==================================================================
% PREGUNTA 195 (Índices)
% ==================================================================
\begin{question}{195}{índices}{2}{d}{7}{\\
Para $k\in\mathbb{N}$ sea \( T_k = \{n \in \mathbb{N} \mid n \text{ divisible entre } k \text{ y } k+1\} \). ¿Qué representa \( \bigcup_{k=2}^{4} T_k \)?
\begin{enumerate}
    \item a) Múltiplos de 6, 12 y 20
    \item b) Números primos
    \item c) Múltiplos de 2, 3, y 4
    \item d) Múltiplos de 6, 12 o 20
\end{enumerate}
}
\end{question}

% ==================================================================
% PREGUNTA 196 (Índices)
% ==================================================================
\begin{question}{196}{índices}{3}{b}{7}{\\
Para \( k \in \{1,2,3,4\} \) sea \( A_k = \{x \in \mathbb{N} \mid x = k^2 + 1\} \) . ¿Cuál es el conjunto \( \bigcup_{k=1}^{4} A_k \)?
\begin{enumerate}
    \item a) \{2, 5, 10, 17\}
    \item b) \{2, 5, 10, 17\}
    \item c) \{1, 4, 9, 16\}
    \item d) \{2, 3, 5, 6\}
\end{enumerate}
}
\end{question}

% ==================================================================
% PREGUNTA 197 (Índices)
% ==================================================================
\begin{question}{197}{índices}{1}{c}{7}{\\
Con \( k \in \{0, 1, 2, \dots, 4\} \) sea \( A_k = \{x \in \mathbb{N} \mid x = 2k + 3\} \) . ¿Qué representa \( \bigcap_{k=0}^{4} A_k \)?
\begin{enumerate}
    \item a) \{3, 5, 7, 9, 11\}
    \item b) \{4, 6, 8, 10, 12\}
    \item c) $\emptyset$
    \item d) \{3\}
\end{enumerate}
}
\end{question}

% ==================================================================
% PREGUNTA 198 (Conjuntos)
% ==================================================================
\begin{question}{198}{conjuntos}{1}{d}{2}{\\
Sea \( A = \{x \in \mathbb{Z} \mid x \text{ par}\} \), \( B = \{x \in \mathbb{Z} \mid x \text{ negativo}\} \). ¿Cuál es \( A \cap B \)?
\begin{enumerate}
    \item a) \{-1, -3, -5, \dots\}
    \item b) \{0, 2, 4, \dots\}
    \item c) $\{x \in \mathbb{Z} \mid x \text{ negativo}\}$
    \item d) \{-2, -4, -6, \dots\}
\end{enumerate}
}
\end{question}

% ==================================================================
% PREGUNTA 199 (Conjuntos)
% ==================================================================
\begin{question}{199}{conjuntos}{1}{b}{2}{\\
Sean \( A = \{x \in \mathbb{Z} \mid x \text{ múltiplo de 4}\} \), \( B = \{x \in \mathbb{Z} \mid x \text{ múltiplo de 6}\} \). ¿Qué representa \( A \cap B \)?
\begin{enumerate}
    \item a) Múltiplos de 6
    \item b) Múltiplos de 12
    \item c) Múltiplos de 24
    \item d) Múltiplos de 10
\end{enumerate}
}
\end{question}

% ==================================================================
% PREGUNTA 200 (Cuantificadores)
% ==================================================================
\begin{question}{200}{cuantificadores}{1}{c}{6}{\\
Sea \( X \) el conjunto de los números reales y \( P(x) = x^2 \geq 0\). ¿Cuál es el valor de verdad de \( \forall x \in X, P(x) \)?
\begin{enumerate}
    \item a) Falso, porque no todos los cuadrados son positivos
    \item b) Falso, porque \( x = -1 \) no cumple
    \item c) Verdadero, porque todo número real al cuadrado es mayor o igual que cero
    \item d) Falso, porque los negativos no cumplen
\end{enumerate}
}
\end{question}

% ==================================================================
% PREGUNTA 201 (Cuantificadores)
% ==================================================================
\begin{question}{201}{cuantificadores}{1}{a}{6}{\\
Sea \( X = \mathbb{R} \setminus \{0\} \), y \( P(x) = \frac{1}{x}> 0 \). ¿Qué valor de verdad tiene \( \forall x \in X, P(x) \)?
\begin{enumerate}
    \item a) Falso, porque \( x < 0 \) implica \( 1/x < 0 \)
    \item b) Verdadero, porque \( 1/x \) existe
    \item c) Verdadero, porque el inverso es positivo
    \item d) Verdadero, porque \( x \neq 0 \)
\end{enumerate}
}
\end{question}

% ==================================================================
% PREGUNTA 202 (Cuantificadores)
% ==================================================================
\begin{question}{202}{cuantificadores}{3}{d}{7}{\\
Sea \( P(x) = x^2 - 4x + 3 = 0 \). ¿Qué representa la expresión \( \exists x \in \mathbb{N} \text{ tal que } P(x) \)?
\begin{enumerate}
    \item a) No hay ningún natural que lo cumpla
    \item b) Todos los naturales lo cumplen
    \item c) Lo cumple solo \( x = 0 \)
    \item d) Existen naturales que lo cumplen
\end{enumerate}
}
\end{question}
% ==================================================================
% PREGUNTA 203 (Lógica Proposicional)
% ==================================================================
\begin{question}{203}{proposiciones}{2}{c}{3}{\\
Sean las proposiciones: 
$p$: La aplicación tiene errores, 
$q$: Se libera una actualización, 
$r$: Los usuarios están satisfechos. Interprete:
$$((p \leftrightarrow q) \land (q \leftrightarrow r)) \land (p \leftrightarrow r)$$
\begin{enumerate}
    \item a) Si hay errores, entonces los usuarios están satisfechos.
    \item b) Si se libera una actualización, entonces hay errores y satisfacción.
    \item c) Hay errores si y solo si se libera una actualización y  se libera una actualizacion si y solo si los usuarios están satisfechos; y,  hay errores si y solo si los usuarios están satisfechos.
    \item d) Si no hay errores, entonces no se actualiza la aplicación.
\end{enumerate}
}
\end{question}

% ==================================================================
% PREGUNTA 204 (Cuantificadores y conjuntos)
% ==================================================================
\begin{question}{204}{cuantificadores, conjuntos}{2}{b}{7}{\\
Sea $X$ el conjunto de los cursos universitarios y sean $P =$ el conjunto de cursos que tiene más de 50 estudiantes, $Q =$ el conjunto de cursos que requiere más de 5 prácticas. \\Interprete:
$\neg(\exists x \in X $ tq $ x\in P \land x \notin Q)$
\begin{enumerate}
    \item a) Todos los cursos con prácticas tienen menos de 50 estudiantes.
    \item b) Todo curso universitario no tiene más de 50 estudiantes o requiere más de 5 practicas.
    \item c) Ningún curso requiere prácticas y tiene más de 50 estudiantes.
    \item d) Ningún curso tiene más de 50 estudiantes.
\end{enumerate}
}
\end{question}

% ==================================================================
% PREGUNTA 205 (Lógica y conjuntos con contexto)
% ==================================================================
\begin{question}{205}{lógica, conjuntos}{1}{d}{4}{\\
En una universidad, $E$ es el conjunto de estudiantes, $A$ de asistentes, $P$ de profesores. Interprete:
$$E \cap A \neq \emptyset \land A \cap P = \emptyset$$
\begin{enumerate}
    \item a) Algunos profesores también son asistentes.
    \item b) Algunos asistentes no son estudiantes.
    \item c) Todos los asistentes son estudiantes.
    \item d) Algunos estudiantes también son asistentes, pero ningún profesor es asistente.
\end{enumerate}
}
\end{question}

% ==================================================================
% PREGUNTA 206 (Equivalencias lógicas)
% ==================================================================
\begin{question}{206}{proposiciones}{1}{a}{3}{\\
¿Cuál de las siguientes proposiciones es lógicamente equivalente a $A \rightarrow B$? Recuerde que logicamente equivalente es que tengan la misma tabla de verdad.
\begin{center}
\begin{tabular}{|c|c|c|}
\hline
$A$ & $B$ & $A \rightarrow B$ \\
\hline
V & V & V \\
V & F & F \\
F & V & V \\
F & F & V \\
\hline
\end{tabular}
\end{center}

\begin{enumerate}
    \item a) $\neg A \lor B$
    \item b) $\neg A \land B$
    \item c) $A \rightarrow B$
    \item d) $A \land \neg B$
\end{enumerate}
}
\end{question}

% ==================================================================
% PREGUNTA 207 (Lógica con cuantificadores)
% ==================================================================
\begin{question}{207}{cuantificadores}{1}{c}{3}{
Sea $X = \{1, 2, 3\}$ y $P(x) = x^2 < 10$. ¿Cuál de las siguientes proposiciones es verdadera?
\begin{enumerate}
    \item a) $\exists x \in X \text{ tal que } \neg P(x)$
    \item b) $\forall x \in X, P(x)$
    \item c) $\exists x \in X \text{ tal que } P(x)$
    \item d) $\forall x \in X, \neg P(x)$
\end{enumerate}
}
\end{question}

% ==================================================================
% PREGUNTA 208 (Negación de proposiciones compuestas)
% ==================================================================
\begin{question}{208}{proposiciones}{1}{b}{3}{\\
¿Cuál es la negación correcta de la proposición: Si Juan estudia, entonces aprueba el examen?
\begin{enumerate}
    \item a) Si Juan no estudia, entonces no aprueba.
    \item b) Juan estudia y no aprueba.
    \item c) Juan no estudia o no aprueba.
    \item d) Juan no estudia y aprueba.
\end{enumerate}
}
\end{question}

% ==================================================================
% PREGUNTA 209 (Conjuntos con álgebra de subconjuntos)
% ==================================================================
\begin{question}{209}{conjuntos}{1}{d}{2}{\\
Sean $A = \{x \in \mathbb{N} \mid x$ divisible por 3$\}$ y $B = \{x \in \mathbb{N} \mid x$ divisible por 5$\}$. ¿Qué representa $A \cap B$?
\begin{enumerate}
    \item a) Múltiplos de 3
    \item b) Múltiplos de 5
    \item c) Múltiplos de 10
    \item d) Múltiplos de 15
\end{enumerate}
}
\end{question}

% ==================================================================
% PREGUNTA 210 (Lógica Proposicional con tres variables)
% ==================================================================
\begin{question}{210}{proposiciones}{1}{c}{3}{\\
Sean $p$: Es lunes, $q$: Hay clase, $r$: Llueve. ¿Qué representa la proposición $(p \land q) \lor \neg r$?
\begin{enumerate}
    \item a) Es lunes, hay clase y llueve.
    \item b) Si no llueve, entonces es lunes y hay clase.
    \item c) Es lunes y hay clase; o no llueve.
    \item d) Si es lunes o hay clase, entonces no llueve.
\end{enumerate}
}
\end{question}

% ==================================================================
% PREGUNTA 211 (Cuantificadores con condición compuesta)
% ==================================================================
\begin{question}{211}{cuantificadores}{1}{a}{7}{\\
Sea $X = \{x \in \mathbb{N} \mid x < 10\}$ y $P: $ es el conjunto de todos los impares que cumplen que son mayor a 3. ¿Qué representa $\forall x \in X, x\notin P$?
\begin{enumerate}
    \item a) Ningún número menor que 10 es impar y mayor que 3.
    \item b) Todos los números impares menores que 10 son mayores que 3.
    \item c) Todos los números menores que 10 son pares.
    \item d) Existe un número menor que 10 que es impar y mayor que 3.
\end{enumerate}
}
\end{question}

% ==================================================================
% PREGUNTA 212 (Lógica con tablas de verdad implícitas)
% ==================================================================
\begin{question}{212}{proposiciones}{1}{b}{3}{\\
¿Cuál es el valor de verdad de la proposición $(p \lor \neg p) \land (p \land \neg p)$?
\begin{enumerate}
    \item a) Verdadera
    \item b) Falsa
    \item c) Depende de $p$
    \item d) Indeterminada
\end{enumerate}
}
\end{question}
% ==================================================================
% PREGUNTA 213
% (Tema: lógica, dif:3, res:c, week:12)
% ==================================================================
\begin{question}{213}{lógica}{1}{c}{7}{
\\Sea $T:$  es el conjunto de trabajadores, $R:$ es el conjunto de trabajadores que recibe salario, $F:$ es el conjunto de trabajadores que está en formación.\\ \textbf{¿Qué significa $\forall x \in T,  (x\in R \lor x\in F)$?}
\begin{enumerate}
    \item a) Todos los trabajadores reciben salario y están en formación.
    \item b) Ningún trabajador está sin salario.
    \item c) Todo trabajador o bien recibe salario o está en formación.
    \item d) Solo los trabajadores en formación reciben salario.
\end{enumerate}
}
\end{question}

% ==================================================================
% PREGUNTA 214
% (Tema: lógica, dif:4, res:b, week:12)
% ==================================================================
\begin{question}{214}{lógica}{1}{b}{7}{
\\ Sea $S:$  es el conjunto de sofware , $L:$  es  el conjunto de sofware libre, $I:$ es el conjunto de sofware que requiere instalación. \\ \textbf{¿Qué expresa $\neg (\exists x\in S, $ tq $ (X\in L \land x\notin I))$?}
\begin{enumerate}
    \item a) Todo software libre requiere instalación.
    \item b) No existe software libre que no requiera instalación.
    \item c) Hay software libre que no requiere instalación.
    \item d) Ningún software requiere instalación.
\end{enumerate}
}
\end{question}

% ==================================================================
% PREGUNTA 215
% (Tema: cuantificadores, dif:4, res:d, week:12)
% ==================================================================
\begin{question}{215}{cuantificadores}{1}{d}{7}{
\\Sea $A:$ es el conjunto de alumnos, $T:$ es el conjunto de alumnos que toma tutorías, $P:$ es el conjunto de alumnos que pasa la materia.\\ \textbf{¿Qué significa $\exists x \in A $ tq $( X\notin T \land x\in P)$?}
\begin{enumerate}
    \item a) Ningún alumno aprueba sin tutorías.
    \item b) Todos los alumnos que aprueban toman tutorías.
    \item c) Solo los alumnos con tutoría aprueban.
    \item d) Hay al menos un alumno que aprueba sin tomar tutorías.
\end{enumerate}
}
\end{question}

% ==================================================================
% PREGUNTA 216
% (Tema: lógica, dif:4, res:a, week:12)
% ==================================================================
\begin{question}{216}{lógica}{1}{a}{7}{
\\Defina $E:$ conjunto de equipos, $N:$ conjunto de equipos nuevos, $F:$ conjunto de equipos que funcionan bien.\\ \textbf{¿Qué expresa $\forall x \in E,\ (x \in N \leftrightarrow x \in F)$?}
\begin{enumerate}
    \item a) Todo equipo funciona bien si y solo si es nuevo.
    \item b) Todo equipo nuevo funciona bien.
    \item c) Ningún equipo viejo funciona bien.
    \item d) Algunos equipos viejos no funcionan bien.
\end{enumerate}
}
\end{question}


% ==================================================================
% PREGUNTA 217
% (Tema: lógica, dif:3, res:c, week:12)
% ==================================================================
\begin{question}{217}{lógica}{1}{c}{7}{
\\Defina $A:$ conjunto de alumnos, $L:$ conjunto de alumnos que leen libros, $I:$ conjunto de alumnos que investigan.\\ \textbf{¿Qué expresa $\forall x \in A,\ (x \notin L) \leftrightarrow (x \notin I)$?}
\begin{enumerate}
    \item a) Todo alumno que no investiga, no lee libros.
    \item b) Todo alumno que lee libros, no investiga.
    \item c) Todo alumno no lee libros si y solo si tampoco investiga.
    \item d) Existe al menos un alumno que no lee libros ni investiga.
\end{enumerate}
}
\end{question}


% ==================================================================
% PREGUNTA 218
% (Tema: lógica, dif:4, res:b, week:13)
% ==================================================================
\begin{question}{218}{lógica}{1}{b}{7}{
\\Defina $P:$ conjunto de proyectos, $F:$ conjunto de proyectos financiados, $R:$ conjunto de proyectos con resultados.\\ \textbf{¿Qué expresa $\forall x \in P,\ (x \in R \rightarrow x \in F)$?}
\begin{enumerate}
    \item a) Todo proyecto tiene resultados.
    \item b) Todo proyecto que tiene resultados fue financiado.
    \item c) Existe al menos un proyecto que fue financiado si tiene resultados.
    \item d) Todo proyecto financiado tiene resultados y viceversa.
\end{enumerate}
}
\end{question}


% ==================================================================
% PREGUNTA 219
% (Tema: lógica, dif:4, res:d, week:13)
% ==================================================================
\begin{question}{219}{lógica}{1}{d}{6}{
\\Defina $C:$ conjunto de cursos, $D:$ conjunto de cursos dictados en línea, $M:$ conjunto de cursos con material digital.\\ \textbf{¿Qué expresa $\forall x \in C,\ (x \in D \leftrightarrow x \in M)$?}
\begin{enumerate}
    \item a) Todo curso en línea tiene material digital y es curso.
    \item b) Todo curso tiene clases en línea o material digital.
    \item c) Existe al menos un curso que, si es en línea, tiene material digital.
    \item d) Todo curso es dictado en línea si y solo si tiene material digital.
\end{enumerate}
}
\end{question}


% ==================================================================
% PREGUNTA 220
% (Tema: lógica, dif:3, res:b, week:13)
% ==================================================================
\begin{question}{220}{lógica}{1}{b}{6}{
\\Defina $S:$ conjunto de seminarios, $P:$ conjunto de seminarios públicos, $T:$ conjunto de seminarios que requieren traducción.\\ \textbf{¿Qué expresa $\exists x \in S$ tq $x \in P \land x \in T$?}
\begin{enumerate}
    \item a) Todos los seminarios públicos requieren traducción.
    \item b) Hay al menos un seminario público que requiere traducción.
    \item c) Ningún seminario requiere traducción.
    \item d) Algunos seminarios no son públicos ni traducidos.
\end{enumerate}
}
\end{question}


% ==================================================================
% PREGUNTA 221
% (Tema: cuantificadores, dif:4, res:c, week:13)
% ==================================================================
\begin{question}{221}{cuantificadores}{1}{c}{6}{
\\Defina $I:$ conjunto de informes, $R:$ conjunto de informes revisados, $A:$ conjunto de informes archivados.\\ \textbf{¿Qué expresa $\forall x \in I,\ (x \in R \leftrightarrow x \in A)$?}
\begin{enumerate}
    \item a) Todo informe archivado ha sido revisado.
    \item b) Algunos informes no revisados están archivados.
    \item c) Todo informe ha sido revisado si y solo si ha sido archivado.
    \item d) Ningún informe está archivado.
\end{enumerate}
}
\end{question}

% ==================================================================
% PREGUNTA 222
% (Tema: lógica, dif:4, res:a, week:13)
% ==================================================================
\begin{question}{222}{lógica}{1}{a}{6}{
\\Defina $X:$ conjunto de experimentos, $S:$ conjunto de experimentos exitosos, $R:$ conjunto de experimentos repetibles.\\ \textbf{¿Qué expresa $\forall x \in X,\ (x \in S \leftrightarrow x \in R)$?}
\begin{enumerate}
    \item a) Todo experimento es exitoso si y solo si es repetible.
    \item b) Todo experimento repetible y exitoso pertenece al conjunto de experimentos.
    \item c) Existe al menos un experimento tal que si es exitoso, es repetible.
    \item d) Todo experimento es tal que, si es repetible, es exitoso.
\end{enumerate}
}
\end{question}


% ==================================================================
% PREGUNTA 223
% (Tema: lógica, dif:4, res:c, week:13)
% ==================================================================
\begin{question}{223}{lógica}{1}{c}{6}{
\\Defina $A:$ conjunto de algoritmos, $E:$ conjunto de algoritmos eficientes, $U:$ conjunto de algoritmos usados actualmente.\\ \textbf{¿Qué expresa $\exists x \in A$ tq $x \in E \land x \notin U$?}
\begin{enumerate}
    \item a) Todos los algoritmos eficientes están en uso actualmente.
    \item b) Los algoritmos usados no son eficientes.
    \item c) Existe un algoritmo eficiente que no se usa actualmente.
    \item d) Ningún algoritmo eficiente se encuentra en desuso.
\end{enumerate}
}
\end{question}


% ==================================================================
% PREGUNTA 224
% (Tema: lógica, dif:4, res:b, week:13)
% ==================================================================
\begin{question}{224}{lógica}{1}{b}{6}{
\\Defina $D:$ conjunto de documentos, $A:$ conjunto de documentos aprobados, $V:$ conjunto de documentos verificados.\\ \textbf{¿Qué expresión lógica traduce correctamente la frase:} Todo documento es aprobado si y solo si es verificado?
\begin{enumerate}
    \item a) $\forall x \in D,\ (x \in A \land x \notin V)$
    \item b) $\forall x \in D,\ (x \in A \leftrightarrow x \in V)$
    \item c) $\exists x \in D$ tq $(x \in A \lor x \in V)$
    \item d) $\forall x \in D,\ (x \in V \rightarrow x \in A)$
\end{enumerate}
}
\end{question}


% ==================================================================
% PREGUNTA 225
% (Tema: cuantificadores, dif:4, res:d, week:13)
% ==================================================================
\begin{question}{225}{cuantificadores}{1}{d}{6}{
\\Defina $P:$ conjunto de proyectos, $T:$ conjunto de proyectos con presupuesto, $C:$ conjunto de proyectos completos.\\ \textbf{¿Qué expresión lógica traduce correctamente la frase:} Todo proyecto esta completo y tiene presupuesto?
\begin{enumerate}
    \item a) $\forall x \in P,\ x \in C \rightarrow x \in P$ 
    \item b) $\exists x \in P$ tq $x \in C \land x \notin T$
    \item c) $\forall x \in P,\ x \in T \rightarrow x \in C$
    \item d) $\forall x \in P,\ x \in C \land x \in T$
\end{enumerate}
}
\end{question}


% ==================================================================
% PREGUNTA 226
% (Tema: lógica, dif:4, res:a, week:14)
% ==================================================================
\begin{question}{226}{lógica}{1}{a}{6}{
\\Defina $N:$ conjunto de noticias, $R:$ conjunto de noticias relevantes, $P:$ conjunto de noticias publicadas.\\ \textbf{¿Qué expresión lógica representa correctamente la frase:} Todas las noticias son relevantes y son publicadas?
\begin{enumerate}
    \item a) $\forall x \in N,\ x \in P \land x \in R$
    \item b) $\forall x \in N,\ x \in R \rightarrow x \in P$
    \item c) $\exists x \in N$ tq $x \in P \rightarrow x \in R$
    \item d) $\forall x \in N,\ x \in R \rightarrow x \in R$
\end{enumerate}
}
\end{question}


% ==================================================================
% PREGUNTA 227
% (Tema: lógica, dif:4, res:b, week:14)
% ==================================================================
\begin{question}{227}{lógica}{1}{b}{6}{
\\Defina $V:$ conjunto de vacunas, $E:$ subconjunto de vacunas efectivas, $D:$ subconjunto de vacunas distribuidas.\\ \textbf{¿Qué expresión representa correctamente la afirmación:}\\ $\forall x \in V,\ (x \in E) \lor (x \in D)$?
\begin{enumerate}
    \item a) Toda vacuna distribuida es efectiva.
    \item b) Toda vacuna  es efectiva ó es distribuida.
    \item c) Solo se distribuyen vacunas inefectivas.
    \item d) Las vacunas distribuidas son todas efectivas.
\end{enumerate}
}
\end{question}


% ==================================================================
% PREGUNTA 228
% (Tema: lógica, dif:4, res:d, week:14)
% ==================================================================
\begin{question}{228}{lógica}{1}{d}{6}{
\\Defina $T:$ conjunto de tesis, $A:$ subconjunto de tesis aprobadas, $R:$ subconjunto de tesis que requieren revisión.\\ \textbf{¿Qué significa la afirmación:} $\exists x \in T$ tq $x \in A \land x \in R$?
\begin{enumerate}
    \item a) Todas las tesis aprobadas requieren revisión.
    \item b) Ninguna tesis requiere revisión.
    \item c) No existen tesis aprobadas.
    \item d) Existe al menos una tesis aprobada que requiere revisión.
\end{enumerate}
}
\end{question}


% ==================================================================
% PREGUNTA 229
% (Tema: lógica, dif:4, res:c, week:14)
% ==================================================================
\begin{question}{229}{lógica}{1}{c}{6}{
\\Defina $I:$ conjunto de invitados, $C:$ subconjunto de quienes confirmaron asistencia, $P:$ subconjunto de quienes participaron.\\ \textbf{¿Qué representa la afirmación:} Todos invitados no confirmaron asistencia y no participaron? 
\begin{enumerate}
    \item a) $\forall x \in I,\ x \in C \lor x \in P$
    \item b) $\forall x \in I,\ x \in C \rightarrow x \in P$
    \item c) $\forall x \in I,\ x \notin C \land x \notin P$
    \item d) $\forall x,\ x \notin P \rightarrow x \notin C$
\end{enumerate}
}
\end{question}


% ==================================================================
% PREGUNTA 230
% (Tema: cuantificadores, dif:4, res:a, week:14)
% ==================================================================
\begin{question}{230}{cuantificadores}{1}{a}{6}{
\\Defina $D:$ conjunto de docentes, $I:$ conjunto de quienes investigan, $P:$ conjunto de quienes publican.\\ \textbf{¿Qué expresa la afirmación:} $\forall x \in D,\ x \in I \land x \in P$?
\begin{enumerate}
    \item a) Todo docente investiga y publica.
    \item b) Todo el que investiga publica.
    \item c) Algunos docentes no investigan ni publican.
    \item d) Solo los docentes que publican investigan.
\end{enumerate}
}
\end{question}


% ==================================================================
% PREGUNTA 231
% (Tema: lógica, dif:4, res:c, week:14)
% ==================================================================
\begin{question}{231}{lógica}{1}{c}{6}{
\\Defina $L:$ conjunto de libros, $F:$ conjunto de libros de ficción, $B:$ conjunto de libros que están en la biblioteca.\\ \textbf{¿Qué expresa la afirmación:} Hay libros de ficción que no están en la biblioteca?
\begin{enumerate}
    \item a) $\forall x \in L , X\in F \rightarrow x\in B$
    \item b) $\forall x \in L, x\in B$
    \item c) $\exists x \in L \textbf{ tal que }\ x \in F \land x \notin B$
    \item d) $\exists x \in L \textbf{ tal que }\ x\in F \rightarrow x\notin B$
\end{enumerate}
}
\end{question}


% ==================================================================
% PREGUNTA 232
% (Tema: lógica, dif:4, res:b, week:14)
% ==================================================================
\begin{question}{232}{lógica}{1}{d}{6}{
\\Defina $A:$ conjunto de aplicaciones, $M:$ conjunto de aplicaciones móviles, $S:$ conjunto de aplicaciones seguras.\\ \textbf{¿Qué expresa la afirmación:} Todas las aplicaciones son móviles y son seguras?
\begin{enumerate}
    \item a) $\forall x \in A, x\in S \rightarrow x\in M$
    \item b) $\forall x \in A,\ x \in M \rightarrow x \in S$
    \item c) $\exists x \in A \textbf{ tal que }\ x \in M \land x \in S$
    \item d) $\forall x \in A,\ x \in M \land x \in S$
\end{enumerate}
}
\end{question}

% ==================================================================
% PREGUNTA 233
% (Tema: lógica, dif:4, res:b, week:15)
% ==================================================================
\begin{question}{233}{lógica}{1}{b}{6}{
\\Defina $S:$ conjunto de servidores, $A:$ conjunto de servidores activos, $R:$ conjunto de servidores en reparación.\\ \textbf{¿Qué expresa la afirmación:} $\forall x \in S,\ x \in A \leftrightarrow x \notin R$?
\begin{enumerate}
    \item a) Todos los servidores en reparación están activos.
    \item b) Un servidor está activo si y solo si no está en reparación.
    \item c) Todos los servidores activos requieren reparación.
    \item d) Ningún servidor activo está en reparación.
\end{enumerate}
}
\end{question}

% ==================================================================
% PREGUNTA 234
% (Tema: cuantificadores, dif:4, res:c, week:15)
% ==================================================================
\begin{question}{234}{cuantificadores}{1}{c}{6}{
\\Defina $P:$ conjunto de participantes, $I:$ conjunto de participantes inscritos, $C:$ conjunto de participantes que compitieron.\\ \textbf{¿Qué expresa la afirmación:} Todos los participantes se inscribieron si y solo si compitieron?
\begin{enumerate}
    \item a) $\exists x \in P \textbf{ tal que }\ x \in C \rightarrow x \in I$
    \item b) $\forall x \in P,\ x \in I \rightarrow x \in C$
    \item c) $\forall x \in P,\ x \in I \leftrightarrow x \in C$
    \item d) $\forall x \in P,\ x \in P \rightarrow (x \in I \land x \in C)$
\end{enumerate}
}
\end{question}


% ==================================================================
% PREGUNTA 235
% (Tema: lógica, dif:4, res:a, week:15)
% ==================================================================
\begin{question}{235}{lógica}{1}{a}{6}{
\\Defina $T:$ conjunto de trámites, $E:$ conjunto de trámites en línea, $F:$ conjunto de trámites gratuitos.\\ \textbf{¿Qué expresa la afirmación:} Todos los trámites gratuitos están en línea?
\begin{enumerate}
    \item a) $\forall x \in T,\ x \in F \rightarrow x \in E$
    \item b) $\forall x \in T,\ x \in E \rightarrow x \in F$
    \item c) $\exists x \in T \textbf{ tal que }\ x \in F \land x \in E$
    \item d) $\neg (\exists x \in T \textbf{ tal que }\ x \in F \land x \notin E)$
\end{enumerate}
}
\end{question}


% ==================================================================
% PREGUNTA 236
% (Tema: lógica, dif:4, res:d, week:15)
% ==================================================================
\begin{question}{236}{lógica}{1}{a}{6}{
\\Defina $M:$ conjunto de modelos matemáticos, $A:$ conjunto de modelos aplicados, $V:$ conjunto de modelos validados.\\ \textbf{¿Qué expresa la afirmación:} todo modelo matemático es aplicado si y solo si es validado?
\begin{enumerate}
    \item a) $\forall x \in M,\ x \in A \leftrightarrow x \in V$
    \item b) $\exists x \in M \textbf{ tal que }\ x \in A \rightarrow x \in V$
    \item c) $\forall x \in M,\ x \in V \rightarrow x \in A$
    \item d) $\forall x \in M,\ x \in A \rightarrow x \in V$
\end{enumerate}
}
\end{question}


% ==================================================================
% PREGUNTA 237
% (Tema: lógica, dif:4, res:c, week:15)
% ==================================================================
\begin{question}{237}{lógica}{1}{c}{6}{
\\Defina $S:$ conjunto de seminarios, $I:$ conjunto de seminarios internacionales, $T:$ conjunto de seminarios transmitidos en vivo.\\ \textbf{¿Qué expresa la afirmación:} $\exists x \in S \textbf{ tal que }\ x \in I \land x \notin T$?
\begin{enumerate}
    \item a) Todos los seminarios internacionales se transmiten en vivo.
    \item b) Algunos seminarios son transmitidos en vivo sin ser internacionales.
    \item c) Existe un seminario internacional que no es transmitido en vivo.
    \item d) Ningún seminario internacional es transmitido en vivo.
\end{enumerate}
}
\end{question}


% ==================================================================
% PREGUNTA 238
% (Tema: lógica, dif:4, res:a, week:15)
% ==================================================================
\begin{question}{238}{lógica}{1}{a}{6}{
\\Defina $E:$ conjunto de estudiantes, $V:$ conjunto de estudiantes que viven en residencia, $C:$ conjunto de estudiantes que cocinan.\\ \textbf{¿Qué expresa la afirmación:} Todo estudiante vive en residencia y cocina?
\begin{enumerate}
    \item a) $\forall x \in E,\ x \in V \land x \in C$
    \item b) $\forall x \in E,\ x \in V \leftrightarrow x \in C$
    \item c) $\forall x \in E,\ x \in V \rightarrow (x \in E \land x \in C)$
    \item d) $\exists x \in E \textbf{ tal que }\ x \in V \land x \notin C$
\end{enumerate}
}
\end{question}



% ==================================================================
% PREGUNTA 239
% (Tema: cuantificadores, dif:4, res:b, week:15)
% ==================================================================
\begin{question}{239}{cuantificadores}{1}{b}{6}{
\\Defina $A:$ conjunto de asignaturas, $O:$ conjunto de asignaturas obligatorias, $P:$ conjunto de asignaturas con prerrequisitos.\\ \textbf{¿Qué expresa la afirmación:} $\forall x \in A,\ x \in O \lor x \in P$?
\begin{enumerate}
    \item a) Toda asignatura obligatoria tiene prerrequisitos.
    \item b) Toda asignatura es obligatoria o tiene prerrequisitos.
    \item c) Ninguna asignatura optativa tiene prerrequisitos.
    \item d) Solo las asignaturas obligatorias tienen prerrequisitos.
\end{enumerate}
}
\end{question}


% ==================================================================
% PREGUNTA 240
% (Tema: lógica, dif:4, res:d, week:15)
% ==================================================================
\begin{question}{240}{lógica}{1}{d}{6}{
\\Defina $L:$ conjunto de laptops, $B:$ conjunto de laptops con buena batería, $P:$ conjunto de laptops portátiles.\\ \textbf{¿Qué expresa la afirmación:} $\forall x \in L,\ x \in B \land x \in P$?
\begin{enumerate}
    \item a) Toda laptop es portátil o tiene buena batería.
    \item b) Solo las laptops portátiles tienen buena batería.
    \item c) Existen laptops con buena batería que no son portátiles.
    \item d) Toda laptop tiene buena batería y es portátil.
\end{enumerate}
}
\end{question}


% ==================================================================
% PREGUNTA 241
% (Tema: lógica, dif:4, res:c, week:16)
% ==================================================================
\begin{question}{241}{lógica}{1}{c}{6}{
\\Defina $C:$ conjunto de clientes, $F:$ conjunto de clientes que dejan reseña, $S:$ conjunto de clientes satisfechos.\\ \textbf{¿Qué expresa la afirmación:} Todos los clientes estan satisfechos y dejan reseña?
\begin{enumerate}
    \item a) $\forall x \in C,\ x \in F \rightarrow x \in C$
    \item b) $\forall x \in C,\ x \in F \rightarrow x \notin C$
    \item c) $\forall x \in C,\ x \in F \land x \in S$
    \item d) $\forall x \in C,\ x \in S \rightarrow x \in F$
\end{enumerate}
}
\end{question}

% ==================================================================
% PREGUNTA 242
% (Tema: lógica, dif:4, res:a, week:16)
% ==================================================================
\begin{question}{242}{lógica}{1}{a}{6}{
\\Defina $T:$ conjunto de textos, $L:$ conjunto de textos largos, $R:$ conjunto de textos revisados.\\ \textbf{¿Qué expresa la afirmación:} $\neg (\exists x \in T \textbf{ tal que}\ (x \in L \land x \notin R))$?
\begin{enumerate}
    \item a) No hay textos largos que no hayan sido revisados.
    \item b) Todos los textos largos han sido revisados o no existen textos largos.
    \item c) Todos los textos que han sido revisados son largos.
    \item d) Ningún texto largo ha sido revisado.
\end{enumerate}
}
\end{question}



% ==================================================================
% PREGUNTA 243
% (Tema: lógica, dif:4, res:c, week:16)
% ==================================================================
\begin{question}{243}{lógica}{1}{a}{6}{
\\Defina $C:$ conjunto de cursos, $E:$ conjunto de cursos evaluados, $F:$ conjunto de cursos con final acumulativo.\\ \textbf{¿Qué expresa la afirmación:} Todo curso tiene final acumulativo si y solo si es evaluado? 
\begin{enumerate}
    \item a) $\forall x \in C,\ x \in F \leftrightarrow x \in E$
    \item b) $\exists x \in C \textbf{ tal que }\ x \in F \rightarrow x \in E$
    \item c) $\forall x \in C,\ x \in F \rightarrow x \in E$
    \item d) $\forall x \in C,\ x \in E \rightarrow x \in F$
\end{enumerate}
}
\end{question}


% ==================================================================
% PREGUNTA 244
% (Tema: lógica, dif:4, res:a, week:16)
% ==================================================================
\begin{question}{244}{lógica}{1}{c}{6}{
\\Defina $S:$ conjunto de software, $L:$ conjunto de software libre, $P:$ conjunto de software popular.\\ \textbf{¿Qué expresa la afirmación:} Todo software es libre y es popular? 
\begin{enumerate}
    \item a) $\forall x \in S,\ x \in L \rightarrow x \in P$
    \item b) $\forall x \in S,\ x \in P \rightarrow x \in L$
    \item c) $\forall x \in S,\ x \in P \land x \in L$
    \item d) $\exists x \in S \textbf{ tal que}\ x \in L \land x \in P$
\end{enumerate}
}
\end{question}

% ==================================================================
% PREGUNTA 245
% (Tema: lógica, dif:4, res:d, week:16)
% ==================================================================
\begin{question}{245}{lógica}{2}{d}{6}{
\\Defina $U:$ conjunto de usuarios, $R:$ conjunto de usuarios que reportan errores, $C:$ conjunto de usuarios que colaboran con el desarrollo.\\ \textbf{¿Qué expresa la afirmación:} Todos los usuarios que reportan errores colaboran con el desarrollo? 
\begin{enumerate}
    \item a) $\exists x \in U \textbf{ tal que}\ x \in R \land x \in C$
    \item b) $\forall x \in U,\ x \in R \lor x \in C$
    \item c) $\forall x \in R,\ x \in U \land x \in C$
    \item d) $\forall x \in U,\ x \in R \rightarrow x \in C$
\end{enumerate}
}
\end{question}

% ==================================================================
% PREGUNTA 246
% (Tema: cuantificadores, dif:4, res:b, week:16)
% ==================================================================
\begin{question}{246}{cuantificadores}{1}{a}{6}{
\\Defina $M:$ conjunto de módulos, $F:$ conjunto de módulos que fallan frecuentemente, $R:$ conjunto de módulos que requieren mantenimiento.\\ \textbf{¿Qué expresa la afirmación:} Todo módulo falla frecuentemente ó requiere mantenimiento? 
\begin{enumerate}
    \item a) $\forall x \in M,\ x \in F \lor x \in R$
    \item b) $\forall x \in M,\ x \in F \rightarrow x \in R$
    \item c) $\forall x \in M,\ x \in F \land x \in R \rightarrow x \in R$
    \item d) $\forall x \in F,\ x \in M \land x \in R$
\end{enumerate}
}
\end{question}


% ==================================================================
% PREGUNTA 247
% (Tema: lógica, dif:4, res:c, week:16)
% ==================================================================
\begin{question}{247}{lógica}{1}{b}{6}{
\\Defina $D:$ conjunto de documentos, $C:$ conjunto de documentos que contienen datos personales, $E:$ conjunto de documentos encriptados.\\ \textbf{¿Qué expresa la afirmación:} Todos los documentos son de datos personales y están encriptados? 
\begin{enumerate}
    \item a) $\forall x \in D,\ x \in C \leftrightarrow x \in E$
    \item b) $\forall x \in D,\ x \in C \land x \in E$
    \item c) $\forall x \in D,\ x \in C \rightarrow x \in E$
    \item d) $\forall x \in E,\ x \in C \land x \in D$
\end{enumerate}
}
\end{question}

% ==================================================================
% PREGUNTA 248
% (Tema: lógica, dif:4, res:a, week:16)
% ==================================================================
\begin{question}{248}{lógica}{1}{a}{6}{
\\Defina $I:$ conjunto de informes, $E:$ conjunto de informes en español, $V:$ conjunto de informes verificados.\\ \textbf{¿Qué expresa la afirmación:} No hay informes en español que no hayan sido verificados? 
\begin{enumerate}
    \item a) $\neg (\exists x \in I\textbf{ tal que}\ x \in E \land x \notin V)$
    \item b) $\forall x \in I,\ x \in E \lor x \in V$
    \item c) $\forall x \in E,\ x \notin V$
    \item d) $\exists x \in I \textbf{ tal que}\ x \in E \land x \in V$
\end{enumerate}
}
\end{question}
% ==================================================================
% PREGUNTA 249
% (Tema: lógica, dif:4, res:d, week:16)
% ==================================================================
\begin{question}{249}{lógica}{1}{d}{6}{
\\Defina $T:$ conjunto de trabajos, $I:$ conjunto de trabajos intensivos, $P:$ conjunto de trabajos con pausas frecuentes.\\ \textbf{¿Qué expresa la afirmación:} $\forall x \in T,\ x \in I \rightarrow x \notin P$?
\begin{enumerate}
    \item a) Todos los trabajos tienen pausas frecuentes.
    \item b) Los trabajos con pausas frecuentes no son intensivos.
    \item c) Ningún trabajo intensivo es frecuente.
    \item d) Si un trabajo es intensivo, entonces no tiene pausas frecuentes.
\end{enumerate}
}
\end{question}

% ==================================================================
% PREGUNTA 250
% (Tema: lógica, dif:4, res:c, week:16)
% ==================================================================
\begin{question}{250}{lógica}{1}{a}{6}{
\\Defina $C:$ conjunto de ciudadanos, $I:$ conjunto de individuos con identificación, $V:$ conjunto de individuos que pueden votar.\\ \textbf{¿Qué expresa la afirmación:} Todo ciudadano puede votar tiene y identificación? 
\begin{enumerate}
    \item a) $\forall x \in C,\ x \in V \land x \in I$
    \item b) $\forall x \in C,\ x \in I \rightarrow x \in V$
    \item c) $\forall x,\ (x \in C \land x \in V) \rightarrow x \in I$
    \item d) $\exists x \in C \textbf{ tal que}\ x \in V \rightarrow x \in I$
\end{enumerate}
}
\end{question}

% ==================================================================
% PREGUNTA 251
% (Tema: lógica, dif:4, res:a, week:17)
% ==================================================================
\begin{question}{251}{lógica}{1}{a}{6}{
\\Defina $D:$ conjunto de dispositivos, $C:$ conjunto de dispositivos conectados a internet, $S:$ conjunto de dispositivos seguros.\\ \textbf{¿Qué expresa la afirmación:} Todo dispositivo está conectado a internet si y solo si es seguro? 
\begin{enumerate}
    \item a) $\forall x \in D,\ x \in C \leftrightarrow x \in S$
    \item b) $\forall x \in D,\ x \in C \land x \in S$
    \item c) $\exists x \in D \textbf{ tal que}\ x \in C \land x \in S$
    \item d) $\forall x \in C,\ x \in D \land x \in S$
\end{enumerate}
}
\end{question}
% ==================================================================
% PREGUNTA 252
% (Tema: cuantificadores, dif:4, res:c, week:17)
% ==================================================================
\begin{question}{252}{cuantificadores}{1}{c}{6}{
\\Defina $T:$ conjunto de tutores, $A:$ conjunto de tutores que atienden estudiantes, $R:$ conjunto de tutores que responden dudas.\\ \textbf{¿Qué expresa la afirmación:} Todo tutor atiende estudiantes ó responde dudas? 
\begin{enumerate}
    \item a) $\forall x \in T,\ x \in R \rightarrow x \in A$
    \item b) $\forall x \in A,\ x \in R$
    \item c) $\forall x \in T,\ x \in A \lor x \in R$
    \item d) $\exists x \in T \textbf{ tal que}\ x \in A \land x \in R$
\end{enumerate}
}
\end{question}

% ==================================================================
% PREGUNTA 253
% (Tema: lógica, dif:4, res:d, week:17)
% ==================================================================
\begin{question}{253}{lógica}{1}{d}{6}{
\\Defina $L:$ conjunto de laboratorios, $Q:$ conjunto de laboratorios con químicos peligrosos, $S:$ conjunto de laboratorios con supervisión.\\ \textbf{¿Qué expresa la afirmación:} Todos los laboratorios tienen supervisión pueden si y solo si tienen químicos peligrosos? 
\begin{enumerate}
    \item a) $\forall x \in L,\ x \in Q \leftrightarrow x \in S$
    \item b) $\forall x \in L,\ x \in Q \rightarrow x \in S$
    \item c) $\forall x \in Q,\ x \in L \land x \in S$
    \item d) $\forall x \in L,\ x \in Q \leftrightarrow x \in S$
\end{enumerate}
}
\end{question}

% ==================================================================
% PREGUNTA 254
% (Tema: lógica, dif:4, res:b, week:17)
% ==================================================================
\begin{question}{254}{lógica}{1}{b}{6}{
\\Defina $C:$ conjunto de candidatos, $E:$ conjunto de candidatos en campaña, $D:$ conjunto de candidatos que han debatido públicamente.\\ \textbf{¿Qué expresa la afirmación:} "Hay candidatos en campaña que no han debatido públicamente"? 
\begin{enumerate}
    \item a) $\forall x \in C,\ x \in E \land x \notin D$
    \item b) $\exists x \in C \textbf{ tal que}\ x \in E \land x \notin D$
    \item c) $\forall x \in E,\ x \notin D$
    \item d) $\neg (\exists x \in C \textbf{ tal que}\ x \in E \land x \notin D)$
\end{enumerate}
}
\end{question}
% ==================================================================
% PREGUNTA 255
% (Tema: lógica, dif:4, res:c, week:17)
% ==================================================================
\begin{question}{255}{lógica}{1}{a}{6}{
\\Defina $P:$ conjunto de proyectos, $F:$ conjunto de proyectos finalizados, $I:$ conjunto de proyectos implementados.\\ \textbf{¿Qué expresa la afirmación:} Todo proyecto es implementado y ha sido finalizado? 
\begin{enumerate}
    \item a) $\forall x \in P,\ x \in I \land x \in F$
    \item b) $\forall x,\ x \in P \land x \in I \rightarrow x \in F$
    \item c) $\forall x,\ (x \in P \land x \in I) \rightarrow x \in F$
    \item d) $\exists x \in P \textbf{ tal que}\ x \in I \land x \in F$
\end{enumerate}
}
\end{question}
% ==================================================================
% PREGUNTA 256
% (Tema: lógica, dif:4, res:a, week:17)
% ==================================================================
\begin{question}{256}{lógica}{1}{b}{6}{
\\Defina $T:$ conjunto de tareas, $P:$ conjunto de tareas pendientes, $S:$ conjunto de tareas enviadas.\\ \textbf{¿Qué expresa la afirmación:} $\forall x \in T,\ x \in S \land x \notin P$?
\begin{enumerate}
    \item a) Toda tarea esta pendiente.
    \item b) Toda tarea ya esta enviada y no esta pendiente.
    \item c) No existen tareas enviadas y pendientes.
    \item d) Solo las tareas no enviadas están pendientes.
\end{enumerate}
}
\end{question}
% ==================================================================
% PREGUNTA 257
% (Tema: lógica, dif:4, res:d, week:17)
% ==================================================================
\begin{question}{257}{lógica}{1}{d}{6}{
\\Defina $E:$ conjunto de ensayos, $R:$ conjunto de ensayos revisados, $A:$ conjunto de ensayos aprobados.\\ \textbf{¿Qué expresa la afirmación:} Todo ensayo es aprobado ó ha sido revisado? 
\begin{enumerate}
    \item a) $\forall  x \in E,  x \in A$
    \item b) $\forall x \in E,\ x \in A \rightarrow x \in R$
    \item c) $\forall x \in A,\ x \in E \land x \in R$
    \item d) $\forall x \in E, x \in A\lor x \in R$
\end{enumerate}
}
\end{question}
% ==================================================================
% PREGUNTA 258
% (Tema: lógica, dif:4, res:b, week:17)
% ==================================================================
\begin{question}{258}{lógica}{1}{b}{6}{
\\Defina $M:$ conjunto de materiales, $R:$ conjunto de materiales reciclables, $T:$ conjunto de materiales tóxicos.\\ \textbf{¿Qué expresa la afirmación:} todo material es tóxico  si y solo si no es reciclable? 
\begin{enumerate}
    \item a) $\forall x \in M,\ x \in T \rightarrow x \in R$
    \item b) $\forall x \in M,\ x \in T \leftrightarrow x \notin R$
    \item c) $\forall x \in M,\ x \in R \rightarrow x \notin T$
    \item d) $\neg (\exists x \in M \textbf{ tal que}\ x \in T \land x \in R)$
\end{enumerate}
}
\end{question}
% ==================================================================
% PREGUNTA 259
% (Tema: lógica, dif:4, res:c, week:17)
% ==================================================================
\begin{question}{259}{lógica}{1}{d}{6}{
\\Defina $S:$ conjunto de simulacros, $P:$ conjunto de simulacros planeados, $E:$ conjunto de simulacros ejecutados.\\ \textbf{¿Qué expresa la afirmación:} Todos los simulacros son planeados y son ejecutados? 
\begin{enumerate}
    \item a) $\forall x \in S,\ x \in E \rightarrow x \in P$
    \item b) $\forall x \in P,\ x \in S \land x \in E$
    \item c) $\forall x \in S,\ x \in P \rightarrow x \in E$
    \item d) $\forall x \in S,\ x \in P \land x \in E $
\end{enumerate}
}
\end{question}

\begin{question}{260}{lógica}{1}{a}{6}{
\\Defina $A:$ conjunto de alumnos, $R:$ conjunto de quienes rinden el examen, $G:$ conjunto de quienes ganan el curso.\\ \textbf{¿Qué expresa la afirmación:} Todo alumno  gana el curso si y solo si rindió el examen? 
\begin{enumerate}
    \item a) $\forall x \in A,\ x \in G \leftrightarrow x \in R$
    \item b) $\forall x \in G,\ x \in A \land x \in R$
    \item c) $\forall x \in R,\ x \in A \land x \in G$
    \item d) $\exists x \in A $ tal que $\ x \in R \land x \in G$
\end{enumerate}
}
\end{question}

\begin{question}{261}{lógica}{1}{b}{6}{
\\Defina $S:$ conjunto de servidores, $M:$ conjunto de servidores en mantenimiento, $F:$ conjunto de servidores que funcionan correctamente.\\ \textbf{¿Qué expresa la afirmación:} Todo servidor no está en mantenimiento ó funciona correctamente?
\begin{enumerate}
    \item a) $\forall x \in S,\ x \notin M \rightarrow x \notin F$
    \item b) $\forall x \in S,\ x \notin M \lor x \in F$
    \item c) $\forall x \in M,\ x \notin F$
    \item d) $\exists x \in S $ tal que $\ x \notin M \land x \in F$
\end{enumerate}
}
\end{question}
\begin{question}{262}{cuantificadores}{1}{c}{6}{
\\Defina $E:$ conjunto de empleados, $T:$ conjunto de quienes tienen tarjeta de acceso, $A:$ conjunto de quienes acceden al edificio.\\ \textbf{¿Qué expresa la afirmación:} Todos los empleados tienen tarjeta y pueden acceder al edificio? 
\begin{enumerate}
    \item a) $\forall x \in A,\ x \in E \land x \in T$
    \item b) $\forall x \in E,\ x \in T \rightarrow x \in A$
    \item c) $\forall x \in E,\ x \in T \land x \in A$
    \item d) $\forall x,\ x \in A \leftrightarrow (x \in E \land x \in T)$
\end{enumerate}
}
\end{question}
\begin{question}{263}{lógica}{1}{c}{6}{
\\Defina $A:$ conjunto de artículos, $R:$ conjunto de artículos revisados, $P:$ conjunto de artículos publicados.\\ \textbf{¿Qué expresa la afirmación:} Para todos los artículos  estos son revisados si y solo si son publicados?
\begin{enumerate}
    \item a) $\forall x \in P,\ x \in A$
    \item b) $\forall x \in R,\ x \in P$
    \item c) $\forall x \in A,\ x \in P \leftrightarrow x \in R$
    \item d) $\exists x \in A,\ x \in P \land x \in R$
\end{enumerate}
}
\end{question}
\begin{question}{264}{lógica}{1}{d}{6}{
\\Defina $S:$ conjunto de sistemas, $U:$ conjunto de sistemas actualizados, $V:$ conjunto de sistemas vulnerables.\\ \textbf{¿Qué expresa la afirmación:} Si un sistema no está actualizado, entonces es vulnerable? 
\begin{enumerate}
    \item a) $\forall x \in S,\ x \notin U \rightarrow x \in V$
    \item b) $\forall x \in S,\ x \notin U \leftrightarrow x \in V$
    \item c) $\forall x \in S,\ x \notin V \rightarrow x \in U$
    \item d) $\forall x \in S,\ x \notin U \rightarrow x \in V$
\end{enumerate}
}
\end{question}
\begin{question}{265}{lógica}{1}{a}{6}{
\\Defina $I:$ conjunto de ingenieros, $C:$ conjunto de quienes saben programar, $S:$ conjunto de quienes trabajan en software.\\ \textbf{¿Qué expresa la afirmación:} Todos los ingenieros trabajan en software ó saben programar? 
\begin{enumerate}
    \item a) $\forall x \in I,\ x \in S \lor x \in C$
    \item b) $\forall x \in S,\ x \in I \land x \in C$
    \item c) $\exists x \in I $ tal que $\ x \in S \land x \in C$
    \item d) $\forall x \in C,\ x \in I \land x \in S$
\end{enumerate}
}
\end{question}
\begin{question}{266}{lógica}{1}{c}{6}{
\\Defina $A:$ conjunto de actividades, $G:$ conjunto de actividades grupales, $P:$ conjunto de actividades que requieren participación activa.\\ \textbf{¿Qué expresa la afirmación:} Todas las actividades son grupales y requieren participación activa? 
\begin{enumerate}
    \item a) $\forall x \in A,\ x \in G \lor x \in P$
    \item b) $\forall x \in G,\ x \in A \land x \notin P$
    \item c) $\forall x \in A,\ x \in G \land x \in P$
    \item d) $\exists x \in A $ tal que $\ x \in G \land x \in P$
\end{enumerate}
}
\end{question}
\begin{question}{267}{lógica}{1}{b}{6}{
\\Defina $D:$ conjunto de documentos, $F:$ conjunto de documentos firmados, $A:$ conjunto de documentos válidos.\\ \textbf{¿Qué expresa la afirmación:} Todo documento esta firmado? 
\begin{enumerate}
    \item a) $\forall x \in F,\ x \in D \land x \in A$
    \item b) $\forall x \in D, x \in F$
    \item c) $\forall x \in A,\ x \in F \rightarrow x \in D$
    \item d) $\forall x \in D,\ x \in F \rightarrow x \in A$
\end{enumerate}
}
\end{question}
\begin{question}{268}{lógica}{1}{d}{6}{
\\Defina $R:$ conjunto de recursos, $E:$ conjunto de recursos en español, $T:$ conjunto de recursos traducidos.\\ \textbf{¿Qué expresa la afirmación:} Existen recursos en español que no han sido traducidos? 
\begin{enumerate}
    \item a) $\forall x \in R,\ x \in E \rightarrow x \in T$
    \item b) $\forall x \in T,\ x \in R \land x \in E$
    \item c) $\neg \exists x \in R,\ x \in E \land x \notin T$
    \item d) $\exists x \in R,\ x \in E \land x \notin T$
\end{enumerate}
}
\end{question}
\begin{question}{269}{lógica}{1}{a}{6}{
\\Defina $A:$ conjunto de alumnos, $T:$ conjunto de alumnos con tutor, $C:$ conjunto de alumnos que consultan dudas.\\ \textbf{¿Qué expresa la afirmación:} Solo los alumnos con tutor consultan dudas? 
\begin{enumerate}
    \item a) $\forall x \in C,\ x \in A \land x \in T$
    \item b) $\forall x \in A,\ x \in C \rightarrow x \in T$
    \item c) $\exists x \in A,\ x \in C \land x \in T$
    \item d) $\forall x \in A,\ x \in T \rightarrow x \in C$
\end{enumerate}
}
\end{question}

\begin{question}{270}{lógica}{1}{c}{6}{
\\Defina $V:$ conjunto de videos, $P:$ conjunto de videos con subtítulos, $A:$ conjunto de videos accesibles.\\ \textbf{¿Qué expresa la afirmación:} Todos los videos son accesibles si y solo si tienen subtítulos?
\begin{enumerate}
    \item a) $\forall x \in V,\ x \in P \leftrightarrow x \in A$
    \item b) $\forall x \in V,\ x \in P \land x \in A$
    \item c) $\forall x \in V,\ x \in A \leftrightarrow x \in P$
    \item d) $\exists x \in V $ tal que $\ x \in A \land x \in P$
\end{enumerate}
}
\end{question}
\begin{question}{271}{lógica}{1}{a}{6}{
\\Defina $T:$ conjunto de trabajos, $E:$ conjunto de trabajos en equipo, $S:$ conjunto de trabajos que requieren coordinación.\\ \textbf{¿Qué expresa la afirmación:} Todo trabajo es en equipo ó requiere coordinación?
\begin{enumerate}
    \item a) $\forall x \in T,\ x \in E \lor x \in S$
    \item b) $\forall x \in T,\ x \in E \land x \in S$
    \item c) $\forall x \in T,\ x \in S \rightarrow x \in E$
    \item d) $\forall x,\ x \in E \land x \in S \rightarrow x \in T$
\end{enumerate}
}
\end{question}

\begin{question}{272}{cuantificadores}{2}{d}{6}{
\\Defina $L:$ conjunto de libros, $N:$ conjunto de libros nuevos, $D:$ conjunto de libros disponibles.\\ \textbf{¿Qué expresa la afirmación:} Todos los libros nuevos están disponibles y hay libros nuevos?
\begin{enumerate}
    \item a) $\forall x \in L,\ x \in N \rightarrow x \in D$
    \item b) $\exists x \in L $ tal que $\ x \in N \land x \in D$
    \item c) $\forall x \in N,\ x \in L \land x \in D$
    \item d) $(\exists x \in L $ tal que $\ x \in N) \land (\forall y \in L,\ y \in N \rightarrow y \in D)$
\end{enumerate}
}
\end{question}
\begin{question}{273}{lógica}{1}{b}{6}{
\\Defina $P:$ conjunto de personas, $C:$ conjunto de personas capacitadas, $M:$ conjunto de personas que manejan maquinaria.\\ \textbf{¿Qué expresa la afirmación:} Todas las personas o manejas maquinaria ó se capacitan?
\begin{enumerate}
    \item a) $\forall x \in C,\ x \in P \land x \in M$
    \item b) $\forall x \in M,\ x \in P \land x \in C$
    \item c) $\forall x \in P,\ x \in C \lor x \in M$
    \item d) $\exists x \in P $ tal que $\ x \in M \land x \in C$
\end{enumerate}
}
\end{question}
\begin{question}{274}{lógica}{1}{c}{6}{
\\Defina $U:$ conjunto de usuarios, $R:$ conjunto de usuarios registrados, $A:$ conjunto de usuarios que acceden al sistema.\\ \textbf{¿Qué expresa la afirmación:} Todos los usuarios estan registrados y acceden al sistema?
\begin{enumerate}
    \item a) $\forall x \in U,\ x \in A \rightarrow x \in R$
    \item b) $\forall x \in A,\ x \in U \land x \in R$
    \item c) $\forall x \in U,\ x \in R \land x \in A$
    \item d) $\exists x \in U $ tal que $\ x \in R \land x \in A$
\end{enumerate}
}
\end{question}
\begin{question}{275}{lógica}{1}{d}{6}{
\\Defina $I:$ conjunto de informes, $F:$ conjunto de informes finalizados, $S:$ conjunto de informes enviados.\\ \textbf{¿Qué expresa la afirmación:} Todos los informes estan enviados pero no fueron finalizados?
\begin{enumerate}
    \item a) $\forall x \in I,\ x \in F \leftrightarrow x \in S$
    \item b) $\forall x \in I,\ x \in F \land x \in S$
    \item c) $\forall x \in F,\ x \in I \land x \in S$
    \item d) $\forall x \in I,\ x \in S \land x \notin F$
\end{enumerate}
}
\end{question}
\begin{question}{276}{lógica}{1}{a}{6}{
\\Defina $S:$ conjunto de sesiones, $V:$ conjunto de sesiones virtuales, $R:$ conjunto de sesiones que requieren inscripción.\\ \textbf{¿Qué expresa la afirmación:} Toda sesión virtual requiere inscripción?
\begin{enumerate}
    \item a) $\forall x \in S,\ x \in V \rightarrow x \in R$
    \item b) $\forall x \in V,\ x \in S \land x \in R$
    \item c) $\forall x \in R,\ x \in S \land x \in V$
    \item d) $\exists x \in S $ tal que $\ x \in V \land x \in R$
\end{enumerate}
}
\end{question}
\begin{question}{277}{lógica}{1}{c}{6}{
\\Defina $P:$ conjunto de proyectos, $F:$ conjunto de proyectos con financiación, $I:$ conjunto de proyectos en implementación.\\ \textbf{¿Qué expresa la afirmación:} Todos los proyectos estan en  implementación si y solo si tienen financiación?
\begin{enumerate}
    \item a) $\forall x \in P,\ x \in F \leftrightarrow x \in I$
    \item b) $\forall x \in F,\ x \in P \land x \in I$
    \item c) $\forall x \in P,\ x \in I \rightarrow x \in F$
    \item d) $\exists x \in P $ tal que $\ x \in I \land x \in F$
\end{enumerate}
}
\end{question}
\begin{question}{278}{lógica}{1}{b}{6}{
\\Defina $C:$ conjunto de contenidos, $O:$ conjunto de contenidos originales, $P:$ conjunto de contenidos que pueden ser publicados.\\ \textbf{¿Qué expresa la afirmación:} Solo el contenido original puede ser publicado?
\begin{enumerate}
    \item a) $\forall x \in O,\ x \in C \land x \in P$
    \item b) $\forall x \in P,\ x \in C \land x \in O$
    \item c) $\forall x \in C,\ x \in P \rightarrow x \in O$
    \item d) $\exists x \in C $ tal que $\ x \in O \land x \in P$
\end{enumerate}
}
\end{question}
\begin{question}{279}{lógica}{1}{d}{6}{
\\Defina $A:$ conjunto de artículos, $R:$ conjunto de artículos revisados, $V:$ conjunto de artículos validados.\\ \textbf{¿Qué expresa la afirmación:} Hay artículos revisados que no han sido validados?
\begin{enumerate}
    \item a) $\forall x \in R,\ x \notin V$
    \item b) $\forall x \in A,\ x \in R \rightarrow x \notin V$
    \item c) $\neg (\exists x \in A $ tal que $\ x \in R \land x \notin V)$
    \item d) $\exists x \in A $ tal que $\ x \in R \land x \notin V$
\end{enumerate}
}
\end{question}

\begin{question}{280}{lógica}{1}{c}{6}{
\\Sea $M:$ conjunto de mensajes, $E:$ conjunto de mensajes enviados, $L:$ conjunto de mensajes leídos.\\ \textbf{¿Qué representa:} Todo mensaje es leído y enviado?
\begin{enumerate}
    \item a) $\forall x \in M,\ x \in E \land x \in L$
    \item b) $\forall x \in L,\ x \in M$
    \item c) $\forall x \in L,\ x \in E$
    \item d) $\exists x \in L\ \text{tal que}\ x \in M \land x \in E$
\end{enumerate}
}
\end{question}

\begin{question}{281}{cuantificadores}{1}{a}{6}{
\\Sea $E:$ conjunto de estudiantes, $B:$ conjunto de quienes leen bibliografía.\\ \textbf{¿Qué significa:} Todos los estudiantes leen bibliografía?
\begin{enumerate}
    \item a) $\forall x \in E,\ x \in B$
    \item b) $\exists x \in E,\ x \in B$
    \item c) $\forall x \in B,\ x \in E$
    \item d) $\neg \exists x \in E\ \text{tal que}\ x \notin B$
\end{enumerate}
}
\end{question}
\begin{question}{282}{lógica}{1}{c}{6}{
\\Sea $P:$ conjunto de profesores, $E:$ conjunto de quienes evalúan exámenes.\\ \textbf{¿Qué representa:} Existe al menos un profesor que no evalúa exámenes?
\begin{enumerate}
    \item a) $\forall x \in P,\ x \notin E$
    \item b) $\neg \exists x \in P,\ x \in E$
    \item c) $\exists x \in P\ \text{tal que}\ x \notin E$
    \item d) $\forall x \in P,\ x \in E$
\end{enumerate}
}
\end{question}
\begin{question}{283}{lógica}{1}{b}{6}{
\\Sea $A:$ conjunto de artículos, $R:$ conjunto de artículos revisados, $P:$ conjunto de artículos publicados.\\ \textbf{¿Qué representa:} Todo lo que ha sido publicado es un artículo que fue revisado?
\begin{enumerate}
    \item a) $\forall x \in A,\ x \in R \rightarrow x \in P$
    \item b) $\forall x \in P,\ x \in A \land x \in R$
    \item c) $\forall x \in R,\ x \in P$
    \item d) $\exists x \in P\ \text{tal que}\ x \notin R$
\end{enumerate}
}
\end{question}
\begin{question}{284}{cuantificadores}{1}{a}{6}{
\\Sea $C:$ conjunto de cursos, $V:$ conjunto de cursos virtuales, $T:$ conjunto de cursos que tienen tutor.\\ \textbf{¿Qué representa:} Existen cursos virtuales que no tienen tutor?
\begin{enumerate}
    \item a) $\forall x \in C,\ x \in V \rightarrow x \in T$
    \item b) $\forall x \in T,\ x \in V$
    \item c) $\neg \exists x \in C,\ x \in V \land x \notin T$
    \item d) $\exists x \in V,\ x \notin T$
\end{enumerate}
}
\end{question}
\begin{question}{285}{lógica}{1}{d}{6}{
\\Sea $S:$ conjunto de sistemas, $A:$ conjunto de sistemas con acceso restringido.\\ \textbf{¿Qué representa:} Todos los sistemas de acceso restringido son sistemas?
\begin{enumerate}
    \item a) $\exists x \in S,\ x \notin A$
    \item b) $\neg \exists x \in S,\ x \in A$
    \item c) $\forall x \in S,\ x \in A$
    \item d) $\forall x \in A,\ x \in S$
\end{enumerate}
}
\end{question}
\begin{question}{286}{cuantificadores}{1}{d}{6}{
\\Sea $E:$ conjunto de estudiantes, $L:$ conjunto de quienes leen todos los días, $P:$ conjunto de quienes pasan el curso.\\ \textbf{¿Qué representa:} Todo estudiante lee todos los días si y solo si pasa el curso?
\begin{enumerate}
    \item a) $\forall x \in P,\ x \in E \land x \in L$
    \item b) $\forall x \in E,\ x \in L \land x \in P$
    \item c) $\exists x \in E,\ x \in L \land x \notin P$
    \item d) $\forall x \in E,\ x \in L \leftrightarrow x \in P$
\end{enumerate}
}
\end{question}
\begin{question}{287}{lógica}{1}{a}{6}{
\\Sea $D:$ conjunto de documentos, $E:$ conjunto de documentos electrónicos, $F:$ conjunto de documentos firmados.\\ \textbf{¿Qué representa:} Todo documento electrónico fue firmado?
\begin{enumerate}
    \item a) $\forall x \in D,\ x \in E \rightarrow x \in F$
    \item b) $\forall x \in F,\ x \in E$
    \item c) $\exists x \in E,\ x \notin F$
    \item d) $\neg \exists x \in E,\ x \in F$
\end{enumerate}
}
\end{question}
\begin{question}{288}{cuantificadores}{1}{c}{6}{
\\Sea $A:$ conjunto de alumnos, $T:$ conjunto de alumnos que toman tutorías, $M:$ conjunto de alumnos que mejoran su rendimiento.\\ \textbf{¿Qué representa:} Todo alumno toma tutorías si y solo si mejora su rendimiento?
\begin{enumerate}
    \item a) $\forall x \in T,\ x \in A \land x \in M$
    \item b) $\exists x \in A,\ x \in T \land x \notin M$
    \item c) $\forall x \in A,\ x \in T \leftrightarrow x \in M$
    \item d) $\forall x \in M,\ x \in T$
\end{enumerate}
}
\end{question}
\begin{question}{289}{lógica}{1}{b}{6}{
\\Sea $R:$ conjunto de recursos, $A:$ conjunto de recursos accesibles, $F:$ conjunto de recursos gratuitos.\\ \textbf{¿Qué representa:} Todo recurso accesible es gratuito?
\begin{enumerate}
    \item a) $\forall x \in R,\ x \in F$
    \item b) $\forall x \in R,\ x \in A \rightarrow x \in F$
    \item c) $\exists x \in A,\ x \notin F$
    \item d) $\forall x \in F,\ x \in A$
\end{enumerate}
}
\end{question}

\begin{question}{290}{cuantificadores}{1}{d}{6}{
\\ Defina $M:$ conjunto de módulos, $O:$ subconjunto de módulos online, $R:$ subconjunto de módulos que requieren registro.\\ \textbf{¿Qué expresa la afirmación:} Hay módulos online que no requieren registro?
\begin{enumerate}
    \item a) $O \subseteq R$
    \item b) $O \cap R = O$
    \item c) $M \subseteq R$
    \item d) $\exists x \in M$ tal que $x \in O \land x \notin R$
\end{enumerate}
}
\end{question}
\begin{question}{291}{cuantificadores}{1}{a}{6}{
\\ Defina $S:$ conjunto de servidores, $C:$ subconjunto de servidores en la nube, $F:$ subconjunto de servidores funcionales.\\ \textbf{¿Qué expresa la afirmación:} Todos los servidores en la nube son funcionales?
\begin{enumerate}
    \item a) $C \subseteq F$
    \item b) $F \subseteq C$
    \item c) $C \cap F = \emptyset$
    \item d) $\exists x \in C$ tal que $x \notin F$
\end{enumerate}
}
\end{question}
\begin{question}{292}{lógica}{1}{c}{6}{
\\ Defina $T:$ conjunto de talleres, $R:$ conjunto de talleres gratuitos, $I:$ conjunto de talleres que requieren inscripción.\\ \textbf{¿Qué expresa la afirmación:} No todos los talleres gratuitos requieren inscripción?
\begin{enumerate}
    \item a) $R \subseteq I$
    \item b) $\exists x \in R$ tal que $x \notin I$
    \item c) $R \not\subseteq I$
    \item d) $R \cap I = R$
\end{enumerate}
}
\end{question}
\begin{question}{293}{cuantificadores}{1}{d}{3}{
\\ Defina $U:$ conjunto de usuarios, $C:$ conjunto de usuarios que crean cuenta, $A:$ conjunto de usuarios que aceptan términos.\\ \textbf{¿Qué expresa la afirmación:} Todo usuario que crea cuenta acepta los términos?
\begin{enumerate}
    \item a) $U \subseteq A$
    \item b) $C \subseteq A$
    \item c) $A \subseteq C$
    \item d) $C \subseteq A$
\end{enumerate}
}
\end{question}
\begin{question}{294}{lógica}{1}{b}{6}{
\\ Defina $P:$ conjunto de publicaciones, $R:$ subconjunto de publicaciones revisadas, $A:$ subconjunto de publicaciones aprobadas.\\ \textbf{¿Qué expresa la afirmación:} Existe una publicación aprobada que no fue revisada?
\begin{enumerate}
    \item a) $A \subseteq R$
    \item b) $\exists x \in P$ tal que $x \in A \land x \notin R$
    \item c) $R = \emptyset$
    \item d) $R$ precede siempre a $A$
\end{enumerate}
}
\end{question}
\begin{question}{295}{lógica}{1}{c}{3}{
\\ Defina $E:$ conjunto de encuestas, $R:$ conjunto de encuestas respondidas, $V:$ conjunto de encuestas válidas.\\ \textbf{¿Qué expresa la afirmación:} Toda encuesta respondida es válida?
\begin{enumerate}
    \item a) $E \subseteq V$
    \item b) $V \subseteq R$
    \item c) $R \subseteq V$
    \item d) $\exists x \in R$ tal que $x \notin V$
\end{enumerate}
}
\end{question}
\begin{question}{296}{cuantificadores}{1}{a}{6}{
\\ Defina $D:$ conjunto de documentos, $E:$ subconjunto de documentos en español, $A:$ subconjunto de documentos aceptados.\\ \textbf{¿Qué expresa la afirmación:} Existe al menos un documento en español que fue aceptado?
\begin{enumerate}
    \item a) $\exists x \in D$ tal que $x \in E \cap A$
    \item b) $E \subseteq A$
    \item c) $A \subseteq E$
    \item d) $E \cap A = \emptyset$
\end{enumerate}
}
\end{question}
\begin{question}{297}{lógica}{1}{c}{3}{
\\ Defina $F:$ conjunto de formularios, $O:$ subconjunto de formularios obligatorios, $R:$ subconjunto de formularios respondidos.\\ \textbf{¿Qué expresa la afirmación:} Todos los formularios obligatorios han sido respondidos?
\begin{enumerate}
    \item a) $F \subseteq O \cap R$
    \item b) $O \cap R = \emptyset$
    \item c) $O \subseteq R$
    \item d) $R \subseteq O$
\end{enumerate}
}
\end{question}
\begin{question}{298}{cuantificadores}{1}{b}{3}{
\\ Defina $T:$ conjunto de trabajadores, $C:$ subconjunto con contrato, $D:$ subconjunto con derecho a beneficios.\\ \textbf{¿Qué expresa la afirmación:} Todo trabajador con contrato tiene derecho a beneficios?
\begin{enumerate}
    \item a) $T \subseteq D$
    \item b) $C \subseteq D$
    \item c) $T \setminus C \neq \emptyset$
    \item d) $C = T$
\end{enumerate}
}
\end{question}
\begin{question}{299}{lógica}{1}{d}{6}{
\\ Defina $S:$ conjunto de sistemas, $A:$ subconjunto con autenticación, $R:$ subconjunto de sistemas seguros.\\ \textbf{¿Qué expresa la afirmación:} Si un sistema tiene autenticación, entonces es seguro?
\begin{enumerate}
    \item a) $A \subseteq R$
    \item b) $S \subseteq A$
    \item c) $R \subseteq A$
    \item d) $\forall x \in S,\ x \in A \rightarrow x \in R$
\end{enumerate}
}
\end{question}
\begin{question}{300}{cuantificadores}{1}{a}{6}{
\\ Defina $A:$ conjunto de alumnos, $P:$ conjunto de los alumnos que participan, $C:$ conjunto de los alumnos que colaboran.\\ \textbf{¿Qué expresa la afirmación:} Todo alumno que participa también colabora?
\begin{enumerate}
    \item a) $P \subseteq C$
    \item b) $A \subseteq C$
    \item c) $C \subseteq A$
    \item d) $A \cap (P \cup C) = \emptyset$
\end{enumerate}
}
\end{question}
% ===============================
% PREGUNTA 301
% ===============================
\begin{question}{301}{cuantificadores}{1}{a}{4}{
\\ Sean $A$ y $B$ conjuntos. \\ 
\textbf{¿Qué debe cumplirse para que la afirmación Si $x \in A$, entonces $x \in B$ sea verdadera para todo $x$?}
\begin{enumerate}
    \item a) $A \subseteq B$
    \item b) $B \subseteq A$
    \item c) $A \cap B = \emptyset$
    \item d) $A = B$
\end{enumerate}
}
\end{question}

% ===============================
\begin{question}{302}{conjuntos}{1}{b}{4}{
\\ Sean $A$ y $B$ conjuntos. \\
\textbf{¿Cuál es el significado de:} $(A \cap B)^c \subseteq A^c \cup B^c$?
\begin{enumerate}
    \item a) Ningún elemento está en $A$ ni en $B$
    \item b) Todo lo que no está en la intersección, está fuera de al menos uno
    \item c) Todo lo que no está en $A$, está fuera de $B$
    \item d) Todo lo que está fuera de la unión, está fuera de la intersección
\end{enumerate}
}
\end{question}

% ===============================
\begin{question}{303}{conjuntos}{1}{c}{4}{
\\ Si $x \in A \cap B$, ¿cuál de las siguientes afirmaciones es necesariamente verdadera?
\begin{enumerate}
    \item a) $x \in A^c$
    \item b) $x \in B^c$
    \item c) $x \in A$ y $x \in B$
    \item d) $x \in A \cup B$
\end{enumerate}
}
\end{question}

% ===============================
\begin{question}{304}{lógica}{1}{b}{4}{
\\ Dada la implicación Si $p$ entonces $q$, ¿qué se debe asumir para comenzar una demostración directa?
\begin{enumerate}
    \item a) Que $q$ es verdadero
    \item b) Que $p$ es verdadero
    \item c) Que $p \Leftrightarrow q$
    \item d) Que $p$ es falso
\end{enumerate}
}
\end{question}

% ===============================
\begin{question}{305}{lógica}{1}{c}{4}{
\\ En una demostración directa de $p \Rightarrow q$, ¿qué representa el símbolo $\Rightarrow$ al inicio de una línea?
\begin{enumerate}
    \item a) Que el paso es un supuesto
    \item b) Que el paso es opcional
    \item c) Que el paso se deduce lógicamente de lo anterior
    \item d) Que es el último paso de la demostración
\end{enumerate}
}
\end{question}

% ===============================
\begin{question}{306}{conjuntos}{1}{a}{4}{
\\ ¿Qué representa la afirmación: Si $x \in (A \cap B)^c$ entonces $x \in A^c \cup B^c$?
\begin{enumerate}
    \item a) Una aplicación de la ley de De Morgan
    \item b) Una contradicción lógica
    \item c) Que $x$ no pertenece a ningún conjunto
    \item d) Que $x$ pertenece a la intersección
\end{enumerate}
}
\end{question}

% ===============================
\begin{question}{307}{demostraciones}{1}{d}{4}{
\\ ¿Cuál de los siguientes pasos sería inválido en una demostración directa?
\begin{enumerate}
    \item a) Suponer el antecedente
    \item b) Usar una definición
    \item c) Deducir del paso anterior
    \item d) Suponer el consecuente
\end{enumerate}
}
\end{question}

% ===============================
\begin{question}{308}{conjuntos}{1}{c}{2}{
\\ Sea $A = \{1,2,3\}$ y $B = \{3,4\}$. ¿Cuál es $A \cap B$?
\begin{enumerate}
    \item a) $\{1,2\}$
    \item b) $\{4\}$
    \item c) $\{3\}$
    \item d) $\{1,2,4\}$
\end{enumerate}
}
\end{question}

% ===============================
\begin{question}{309}{lógica}{1}{b}{4}{
\\ En una implicación $p \Rightarrow q$, ¿qué se concluye si $p$ es verdadero y $q$ es falso?
\begin{enumerate}
    \item a) Que la implicación es verdadera
    \item b) Que la implicación es falsa
    \item c) Que $p$ y $q$ son equivalentes
    \item d) Que no se puede determinar
\end{enumerate}
}
\end{question}

% ===============================
\begin{question}{310}{conjuntos}{1}{a}{4}{
\\ Si $x \in A \cup B$, ¿qué podemos afirmar?
\begin{enumerate}
    \item a) $x \in A$ o $x \in B$
    \item b) $x \in A$ y $x \in B$
    \item c) $x \notin A$ y $x \notin B$
    \item d) $x \in A^c \cup B^c$
\end{enumerate}
}
\end{question}
% ===============================
% PREGUNTA 311
% ===============================
\begin{question}{311}{implicaciones}{2}{c}{4}{
\\ Si se quiere demostrar que la proposición Si $p$ entonces $q$ es falsa, ¿qué se debe mostrar?
\begin{enumerate}
    \item a) Que $p$ es falsa y $q$ también
    \item b) Que $p$ es verdadera y $q$ también
    \item c) Que $p$ es verdadera y $q$ es falsa
    \item d) Que $q$ implica $p$
\end{enumerate}
}
\end{question}

% ===============================
\begin{question}{312}{implicaciones}{1}{a}{4}{
\\ Sea la proposición: “Si $A \cup B \subseteq A \cap B$ entonces $A = B$”. ¿Cuál de las siguientes pruebas demostraría que esta proposición es falsa?
\begin{enumerate}
    \item a) Encontrar $A$ y $B$ tales que $A \cup B \subseteq A \cap B$ y $A \neq B$
    \item b) Encontrar $A$ y $B$ tales que $A = B$
    \item c) Verificar que $A \cup B \subseteq A \cap B$ para cualquier $A,B$
    \item d) Mostrar que $A \cup B = A \cap B$ implica $A = B$
\end{enumerate}
}
\end{question}

% ===============================
\begin{question}{313}{conjuntos}{1}{d}{2}{
\\ Sean como conjunto universal $U =\{1,2,3,4,5,6,7,8,9,10,20\}, A = \{1,2,3\}$ y $B = \{1,4\}$. ¿Cuál es el conjunto $(A \cup B)^c$?
\begin{enumerate}
    \item a) $\{4\}$
    \item b) $\{1\}$
    \item c) $\{1,4,2\}$
    \item d) $\{5, 6, 7, 8, 9, 10, 20\}$
\end{enumerate}
}
\end{question}

% ===============================
\begin{question}{314}{conjuntos}{2}{b}{4}{
\\ ¿Qué se concluye si $x \in A$ pero $x \notin C$, sabiendo que $A \cap B \subseteq C$?
\begin{enumerate}
    \item a) Que $x$ no está en $B$
    \item b) Que $x$ podría estar en $A$ sin estar en $A \cap B$
    \item c) Que $x$ pertenece a $C$
    \item d) Que $x$ pertenece a $A \cap B$
\end{enumerate}
}
\end{question}

% ===============================
\begin{question}{315}{implicaciones}{1}{c}{4}{
\\ Si $A = \{1,2,3\}$ y $B = \{3,5,7\}$, ¿es verdadera la proposición $A \subseteq B$?
\begin{enumerate}
    \item a) Sí, porque $3$ está en ambos conjuntos
    \item b) Sí, porque $B$ tiene más elementos
    \item c) No, porque hay elementos de $A$ que no están en $B$
    \item d) No se puede saber sin diagrama de Venn
\end{enumerate}
}
\end{question}

% ===============================
\begin{question}{316}{implicaciones}{1}{a}{4}{
\\ ¿Qué tipo de prueba se debe hacer para verificar que una implicación es falsa?
\begin{enumerate}
    \item a) Exhibir un contraejemplo
    \item b) Calcular la diferencia de conjuntos
    \item c) Demostrar que el antecedente es equivalente al consecuente
    \item d) Aplicar la ley de Morgan
\end{enumerate}
}
\end{question}

% ===============================
\begin{question}{317}{implicaciones}{1}{c}{4}{
\\ Dada la proposición: Si $A \setminus B \subseteq C$ entonces $A \subseteq C$, ¿qué clase de prueba demostraría que es falsa?
\begin{enumerate}
    \item a) Verificar que se cumple para algunos ejemplos
    \item b) Usar un diagrama de Venn con tres conjuntos iguales
    \item c) Encontrar conjuntos $A, B, C$ tales que $A \setminus B \subseteq C$ pero $A \not\subseteq C$
    \item d) Aplicar la definición de subconjunto a $B \subseteq A$
\end{enumerate}
}
\end{question}

% ===============================
\begin{question}{318}{conjuntos}{1}{b}{2}{
\\ Sean $A = \{1, 2, 3\}$ y $B = \{3, 17\}$. ¿Cuál es $A \setminus B$?
\begin{enumerate}
    \item a) $\{3, 17\}$
    \item b) $\{1,2\}$
    \item c) $\{1,2,3\}$
    \item d) $\{17\}$
\end{enumerate}
}
\end{question}

% ===============================
\begin{question}{319}{lógica}{1}{a}{4}{
\\ ¿Cuál de las siguientes afirmaciones es equivalente a decir que $p \Rightarrow q$ es falsa?
\begin{enumerate}
    \item a) $p$ es verdadera y $q$ es falsa
    \item b) $p$ y $q$ son verdaderas
    \item c) $q$ es verdadera sin importar $p$
    \item d) $p$ y $q$ son falsas
\end{enumerate}
}
\end{question}

% ===============================
\begin{question}{320}{implicaciones}{1}{d}{4}{
\\ ¿Por qué un solo contraejemplo basta para rechazar una implicación?
\begin{enumerate}
    \item a) Porque simplifica el cálculo
    \item b) Porque es más fácil que demostrar la proposición
    \item c) Porque todos los elementos cumplen la implicación
    \item d) Porque basta con una instancia que haga verdadero el antecedente y falso el consecuente
\end{enumerate}
}
\end{question}

% ===============================
\begin{question}{321}{conjuntos}{1}{b}{2}{
\\ Si $A = \{x \in \mathbb{N} \mid x < 5\}$ y $B = \{3,5,6\}$, ¿qué contiene $A \cap B$?
\begin{enumerate}
    \item a) $\{5\}$
    \item b) $\{3\}$
    \item c) $\{3,5\}$
    \item d) $\{6\}$
\end{enumerate}
}
\end{question}

% ===============================
\begin{question}{322}{conjuntos}{1}{d}{4}{
\\ Si $x \in A$ y $x \notin B$, entonces necesariamente:
\begin{enumerate}
    \item a) $x \in B$
    \item b) $x \in A \cap B$
    \item c) $x \in B \setminus A$
    \item d) $x \in A \setminus B$
\end{enumerate}
}
\end{question}

% ===============================
\begin{question}{323}{exploración}{1}{c}{4}{
\\ ¿Qué se sugiere hacer si no sabes si una proposición es verdadera o falsa?
\begin{enumerate}
    \item a) Elegir la respuesta que más se repita
    \item b) Aplicar la ley de contraposición
    \item c) Intentar probar ambas posibilidades con ejemplos o diagramas
    \item d) Usar conjuntos vacíos en todos los casos
\end{enumerate}
}
\end{question}

% ===============================
\begin{question}{324}{exploración}{1}{a}{4}{
\\ ¿Cuál de las siguientes estrategias ayuda a explorar una implicación compleja?
\begin{enumerate}
    \item a) Probar con ejemplos donde se cumpla el antecedente
    \item b) Demostrar que el consecuente es falso en general
    \item c) Intercambiar el orden del antecedente y consecuente
    \item d) Reducir todos los conjuntos a $\{1\}$
\end{enumerate}
}
\end{question}

% ===============================
\begin{question}{325}{demostraciones}{1}{b}{2}{
\\ ¿Qué se necesita para afirmar con certeza que $A \subseteq B$?
\begin{enumerate}
    \item a) Que los conjuntos tengan igual cantidad de elementos
    \item b) Que todo elemento de $A$ esté en $B$
    \item c) Que $A \cap B = \emptyset$
    \item d) Que $B \setminus A = \emptyset$
\end{enumerate}
}
\end{question}
% ===============================
% PREGUNTA 326
% ===============================
\begin{question}{326}{implicaciones}{1}{b}{4}{
\\ ¿Cuál de las siguientes opciones equivale a negar que $A \subseteq B$?
\begin{enumerate}
    \item a) Todo elemento de $B$ pertenece a $A$
    \item b) Existe al menos un elemento de $A$ que no está en $B$
    \item c) Ningún elemento de $A$ pertenece a $B$
    \item d) Todos los elementos están fuera de $A$
\end{enumerate}
}
\end{question}

% ===============================
\begin{question}{327}{conjuntos}{1}{d}{2}{
\\ Sean $A = \{1\}$ y $B = \{3\}$. ¿Cuál es $A \cap B$?
\begin{enumerate}
    \item a) $\{1,2\}$
    \item b) $\{1,2,3,4\}$
    \item c) $\{4\}$
    \item d) $\emptyset$
\end{enumerate}
}
\end{question}

% ===============================
\begin{question}{328}{implicaciones}{2}{a}{4}{
\\ Sea una proposición de la forma “Si $p$ entonces $q$”. Suponga que no se sabe si es verdadera o falsa.\\
\textbf{¿Cuál de las siguientes estrategias permite demostrar que dicha proposición es falsa?}
\begin{enumerate}
    \item a) Encontrar un solo ejemplo donde $p$ sea verdadera y $q$ falsa
    \item b) Mostrar que $q$ se deduce de $p$ en un caso particular
    \item c) Probar que $q$ es falsa sin asumir $p$
    \item d) Confirmar que la negación de $p$ implica la negación de $q$
\end{enumerate}
}
\end{question}


% ===============================
\begin{question}{329}{implicaciones}{1}{c}{4}{
\\ Si se encuentra un único ejemplo donde el antecedente es cierto y el consecuente falso, ¿qué se puede concluir?
\begin{enumerate}
    \item a) Que la implicación es verdadera
    \item b) Que se requiere más información
    \item c) Que la implicación es falsa
    \item d) Que el ejemplo no basta
\end{enumerate}
}
\end{question}

% ===============================
\begin{question}{330}{lógica}{1}{a}{4}{
\\ En una prueba de tipo “si $p$ entonces $q$”, ¿qué representa iniciar con “Sea $x$ tal que $p$”?
\begin{enumerate}
    \item a) Suponer el antecedente
    \item b) Afirmar el consecuente
    \item c) Negar la proposición
    \item d) Contradecir el dominio
\end{enumerate}
}
\end{question}

% ===============================
\begin{question}{331}{conjuntos}{1}{b}{2}{
\\ Si $A = \{x \in \mathbb{N} \mid x$ par$ \}$ y $B = \{2, 4, 6, 8\}$, ¿cuál es la relación entre $B$ y $A$?
\begin{enumerate}
    \item a) $B = A$
    \item b) $B \subseteq A$
    \item c) $A \subseteq B$
    \item d) $A \cap B = \emptyset$
\end{enumerate}
}
\end{question}

% ===============================
\begin{question}{332}{implicaciones}{2}{d}{4}{
\\ Un estudiante afirma: “Como en todos los ejemplos que revisé, cuando $p$ es verdadera, también lo es $q$, entonces la implicación $p \Rightarrow q$ es verdadera”.\\
\textbf{¿Qué se necesita para refutar su afirmación?}
\begin{enumerate}
    \item a) Encontrar un ejemplo donde $q$ sea falsa, sin importar $p$
    \item b) Mostrar que $p$ y $q$ no pueden ser verdaderas al mismo tiempo
    \item c) Probar que $p$ implica una tercera proposición que contradice $q$
    \item d) Exhibir un solo caso donde $p$ sea verdadera y $q$ sea falsa
\end{enumerate}
}
\end{question}


% ===============================
\begin{question}{333}{conjuntos}{1}{c}{2}{
\\ ¿Cuál es el complemento de $A = \{1,2,3\}$ respecto del universo $U = \{1,2,3,4,5\}$?
\begin{enumerate}
    \item a) $\{1,2,3\}$
    \item b) $\{1,3,5\}$
    \item c) $\{4,5\}$
    \item d) $\emptyset$
\end{enumerate}
}
\end{question}

% ===============================
\begin{question}{334}{lógica}{1}{a}{4}{
\\ Si durante una demostración se obtiene que $x \in A$ y $x \notin B$, ¿cuál es una deducción válida?
\begin{enumerate}
    \item a) $x \in A \setminus B$
    \item b) $x \in A \cap B$
    \item c) $x \in B$
    \item d) $x \in B \setminus A$
\end{enumerate}
}
\end{question}

% ===============================
\begin{question}{335}{exploración}{1}{c}{4}{
\\ ¿Cuál de las siguientes es una manera adecuada de explorar si una afirmación lógica es cierta?
\begin{enumerate}
    \item a) Verificar únicamente con conjuntos vacíos
    \item b) Suponer que todo es verdadero
    \item c) Construir varios ejemplos y buscar un contraejemplo
    \item d) Tomar cualquier afirmación equivalente
\end{enumerate}
}
\end{question}

% ===============================
\begin{question}{336}{conjuntos}{1}{d}{4}{
\\ Si $x \in A \cup B$, ¿cuál de las siguientes afirmaciones es necesariamente verdadera?
\begin{enumerate}
    \item a) $x \in A$ y $x \in B$
    \item b) $x \notin A$
    \item c) $x \notin B$
    \item d) $x \in A$ o $x \in B$
\end{enumerate}
}
\end{question}

% ===============================
\begin{question}{337}{implicaciones}{1}{b}{4}{
\\ Si se sabe que $A \subseteq B$ y $B \subseteq C$, ¿cuál es la conclusión válida?
\begin{enumerate}
    \item a) $C \subseteq A$
    \item b) $A \subseteq C$
    \item c) $B = A$
    \item d) $A \cap C = \emptyset$
\end{enumerate}
}
\end{question}

% ===============================
\begin{question}{338}{lógica}{1}{a}{4}{
\\ Si queremos mostrar que $A \subseteq B$ no se cumple, ¿qué se debe exhibir?
\begin{enumerate}
    \item a) Un $x$ tal que $x \in A$ y $x \notin B$
    \item b) Un $x$ tal que $x \notin A$ y $x \in B$
    \item c) Una diferencia vacía
    \item d) Que $A = B$
\end{enumerate}
}
\end{question}

% ===============================
\begin{question}{339}{exploración}{3}{c}{4}{
\\ ¿Qué tipo de obstáculo puede indicar que una proposición tal vez no sea verdadera?
\begin{enumerate}
    \item a) Que los conjuntos sean finitos
    \item b) Que haya elementos comunes
    \item c) Que no se logre completar una demostración tras varios intentos razonables
    \item d) Que se pueda escribir su contraria
\end{enumerate}
}
\end{question}

% ===============================
\begin{question}{340}{demostraciones}{1}{b}{4}{
\\ En una prueba donde se parte de $p$ y se llega a $q$, ¿qué significa el símbolo “$\Rightarrow$” entre pasos?
\begin{enumerate}
    \item a) Que se está asumiendo una propiedad externa
    \item b) Que se deduce lógicamente del paso anterior
    \item c) Que se terminó la prueba
    \item d) Que se usó una definición auxiliar
\end{enumerate}
}
\end{question}
% ===============================
% PREGUNTA 341
% ===============================
\begin{question}{341}{índices}{1}{a}{7}{
\\ ¿Qué significa la expresión $N_{1995}$ si $N_i$ representa a las personas nacidas en el año $i$?
\begin{enumerate}
    \item a) El conjunto de personas nacidas en 1995
    \item b) El número total de personas vivas en 1995
    \item c) El número de fallecimientos en 1995
    \item d) El subconjunto de personas mayores de edad en 1995
\end{enumerate}
}
\end{question}

% ===============================
\begin{question}{342}{índices}{1}{b}{7}{
\\ ¿Qué representa la suma $\sum_{i=1980}^{1989} |N_i|$ si $N_i$ es el conjunto de nacidos en el año $i$?
\begin{enumerate}
    \item a) El total de nacimientos en la década de los 90
    \item b) El total de nacimientos entre 1980 y 1989
    \item c) El promedio de nacimientos entre 1980 y 1989
    \item d) El conjunto de años entre 1980 y 1989
\end{enumerate}
}
\end{question}

% ===============================
\begin{question}{343}{índices}{1}{d}{7}{
\\ Si $M_{p,q}$ representa lo que el país $p$ importó de $q$, ¿qué significa que $M_{p,q} = 0$?
\begin{enumerate}
    \item a) Que $p$ exportó todo a $q$
    \item b) Que no hubo importaciones de $p$ hacia $q$
    \item c) Que $p$ importó de todos los países menos $q$
    \item d) Que $p$ no importo nada de q
\end{enumerate}
}
\end{question}

% ===============================
\begin{question}{344}{índices}{1}{b}{7}{
\\ ¿Cómo se interpreta $\bigcup_{a=2000}^{2010} N_{a,\text{Col}}$ si $N_{a,\text{Col}}$ es el conjunto de personas nacidas en Colombia en el año $a$?
\begin{enumerate}
    \item a) La intersección de nacimientos entre 2000 y 2010 en Colombia
    \item b) Todos los nacidos en Colombia entre 2000 y 2010
    \item c) Solo los nacidos en 2000
    \item d) Personas vivas entre 2000 y 2010
\end{enumerate}
}
\end{question}

% ===============================
\begin{question}{345}{índices}{1}{c}{7}{
\\ Si $N_{a,b}$ es el conjunto de nacidos en el año a en el pais b y se tiene que:\\$|N_{2010,\text{China}}| > |N_{2010,\text{Venezuela}}|$, ¿qué se concluye?
\begin{enumerate}
    \item a) Que China tiene más años que Venezuela
    \item b) Que en 2010 Venezuela tuvo más nacimientos que China
    \item c) Que nacieron más personas en China que en Venezuela en 2010
    \item d) Que China y Venezuela tienen poblaciones iguales
\end{enumerate}
}
\end{question}
% ===============================
% PREGUNTA 346
% ===============================
\begin{question}{346}{sumatorias}{1}{a}{7}{
\\ ¿Qué representa la expresión $\sum_{i=1}^{5} (2i + 1)$?
\begin{enumerate}
    \item a) $3 + 5 + 7 + 9 + 11$
    \item b) $2 + 4 + 6 + 8 + 10$
    \item c) $1 + 3 + 5 + 7 + 9$
    \item d) $1 + 2 + 3 + 4 + 5$
\end{enumerate}
}
\end{question}

\begin{question}{347}{sumatorias}{1}{d}{7}{
\\ Si $N_{i}$ representa el número de nacimientos en el año $i$, ¿cómo se expresa el total de nacimientos entre 1990 y 1999?
\begin{enumerate}
    \item a) $\sum_{i=1991}^{2000} N_i$
    \item b) $\sum_{i=1989}^{1998} N_i$
    \item c) $\sum_{i=1990}^{2000} N_i$
    \item d) $\sum_{i=1990}^{1999} N_i$
\end{enumerate}
}
\end{question}

\begin{question}{348}{índices}{1}{c}{7}{
\\ Si $P$ es el conjunto de paises y $M_{p,q}$ representa importaciones de $p$ desde $q$, ¿cuál es el significado de la afirmación $\forall p, q \in P, \ M_{p,q} = M_{q,p}$?
\begin{enumerate}
    \item a) Cada país exporta lo mismo que importa
    \item b) Las importaciones son mayores a las exportaciones
    \item c) El comercio es perfectamente simétrico entre todos los países
    \item d) No se puede importar desde uno mismo
\end{enumerate}
}
\end{question}

\begin{question}{349}{conjuntos}{1}{a}{7}{
\\ Si $x \in \bigcap_{i=1}^{4} A_i$, ¿qué se cumple necesariamente?
\begin{enumerate}
    \item a) $x$ pertenece a todos los conjuntos $A_1, A_2, A_3, A_4$
    \item b) $x$ pertenece a al menos uno de los conjuntos
    \item c) $x$ no pertenece a ningún conjunto
    \item d) $x$ pertenece solo a $A_1$
\end{enumerate}
}
\end{question}

\begin{question}{350}{lógica}{1}{a}{7}{
\\ Si una implicación es de la forma $p \Rightarrow q$, ¿cuándo es falsa?
\begin{enumerate}
    \item a) Cuando $p$ es verdadera y $q$ es falsa
    \item b) Cuando $p$ es falsa y $q$ es verdadera
    \item c) Cuando ambos son falsos
    \item d) Cuando ambos son verdaderos
\end{enumerate}
}
\end{question}

% ===============================
% PREGUNTA 351 (Reemplazada)
% ===============================
\begin{question}{351}{índices}{1}{b}{7}{
\\ Sea $P_{i,t}$ la población del país $i$ en el año $t$.\\
\textbf{¿Qué representa la expresión $\sum_{t=2000}^{2010} P_{\text{Colombia},t}$?}
\begin{enumerate}
    \item a) El número total de países que existieron entre 2000 y 2010
    \item b) La suma de la población de Colombia durante el periodo 2000–2010
    \item c) La población de todos los países en el año 2005
    \item d) La población de Colombia en el año 2010
\end{enumerate}
}
\end{question}


\begin{question}{352}{índices}{1}{b}{7}{
\\ Si $N_{a,p}$ representa personas nacidas en el año $a$ en el país $p$, ¿cómo se expresa el total de nacidos en Colombia entre 1980 y 2000?
\begin{enumerate}
    \item a) $\bigcap_{a=1980}^{2000} N_{a,\text{Col}}$
    \item b) $\bigcup_{a=1980}^{2000} N_{a,\text{Col}}$
    \item c) $\sum_{a=1980}^{2000} |N_{a,\text{Col}}|$
    \item d) $N_{1980,\text{Col}}$
\end{enumerate}
}
\end{question}

% ===============================
% PREGUNTA 353 (Sustituida por una más difícil)
% ===============================
\begin{question}{353}{índices}{1}{c}{7}{
\\ Sea $E_{i,t}$ el precio de la matricula en el programa $i$ durante el año $t$.\\
\textbf{¿Qué representa la expresión $\sum_{i=1}^{5} (E_{i,2018} - E_{i,2017})/5$?}
\begin{enumerate}
    \item a) El total de la matricula en 2018 en todos los programas
    \item b) La diferencia total de matricula entre 2017 y 2018 en el programa 5
    \item c) El cambio promedio de matrícula para todos los programas entre 2017 y 2018
    \item d) La matrícula promedio entre 2017 y 2018 para todos los programas
\end{enumerate}
}
\end{question}


\begin{question}{354}{sumatorias}{1}{c}{7}{
\\ ¿Qué representa la sumatoria $\sum_{i=1}^{n} 2i$?
\begin{enumerate}
    \item a) La suma de los números impares
    \item b) La suma de los cuadrados
    \item c) La suma de los primeros $n$ números pares
    \item d) Una progresión aritmética de razón 3
\end{enumerate}
}
\end{question}

\begin{question}{355}{preferencias}{1}{a}{7}{
\\ Sea $P_i$ la relación de preferencia del individuo $i$ entre alternativas.\\
\textbf{¿Qué indica la expresión $hambureguesa \ P_i \ pizza$?}
\begin{enumerate}
    \item a) El individuo $i$ prefiere la $hamburguesa$ sobre la $pizza$
    \item b) El individuo $i$ no prefiere la $hamburguesa$ sobre la $pizza$
    \item c) El individuo $i$ es indiferente entre la $hamburguesa$ y la $pizza$
    \item d) El individuo $i$ prefiere la $pizza$ sobre la $hamburguesa$
\end{enumerate}
}
\end{question}
% ===============================
% PREGUNTA 355 (Reformulada para evaluar índices)
% ===============================
\begin{question}{356}{índices}{1}{c}{7}{
\\ Sea $D_{i,j}$ el número de dispositivos vendidos del tipo $i$ en la ciudad $j$. hay como tipos de dispositivos $tipos=\{1,2,3\}$ y como ciudades $ciudades=\{1,2,3,4,5,6,7,8,9,10\}$ \\
\textbf{¿Qué representa la expresión $\sum_{j=1}^{10} D_{3,j}$?}
\begin{enumerate}
    \item a) La cantidad total de dispositivos vendidos en la ciudad 3
    \item b) El número de ciudades en las que se vendió el dispositivo 3
    \item c) El total de unidades del dispositivo tipo 3 vendidas en todas las ciudades
    \item d) La suma de todos los dispositivos vendidos sin importar el tipo
\end{enumerate}
}
\end{question}


% ===============================
\begin{question}{357}{conjuntos}{1}{d}{7}{
\\ ¿Qué representa la expresión $\bigcap_{i \in I} A_i$?
\begin{enumerate}
    \item a) La unión de todos los conjuntos $A_i$
    \item b) La suma de los cardinales de los conjuntos $A_i$
    \item c) El conjunto vacío
    \item d) Los elementos que pertenecen a todos los $A_i$
\end{enumerate}
}
\end{question}

% ===============================
% ===============================
% PREGUNTA 358 (Reformulada para evaluar índices)
% ===============================
\begin{question}{358}{índices}{1}{b}{7}{
\\ Sea $ciudades$ el conjunto de ciudades $M_{a,b}$ el número de personas que migraron de la ciudad $a$ hacia la ciudad $b$ en un año determinado.\\
\textbf{¿Qué representa la expresión $\sum_{b\in ciudades} M_{a,b}$ para un valor fijo de $a$?}
\begin{enumerate}
    \item a) El total de personas que llegaron a la ciudad $a$ desde otras ciudades
    \item b) El total de personas que salieron de la ciudad $a$ hacia otras ciudades
    \item c) El número de ciudades a las que se puede migrar desde $a$
    \item d) El total de migrantes entre todas las ciudades en el año
\end{enumerate}
}
\end{question}


% ===============================
\begin{question}{359}{sumatorias}{1}{a}{7}{
\\ Si $a_i = 2i+1$, ¿qué representa $\sum_{i=1}^{n} a_i$?
\begin{enumerate}
    \item a) La suma de los primeros $n$ números impares
    \item b) La suma de los primeros $n$ números pares
    \item c) El doble del producto de $n$
    \item d) La diferencia entre $n$ y $2n$
\end{enumerate}
}
\end{question}
%===============================
% ===============================
% PREGUNTA 360 (Reformulada con índices y nuevo contexto)
% ===============================
\begin{question}{360}{índices}{1}{d}{7}{
\\ Sea $R=\{1,2,3,4,5\}$ el conjunto de regiones $P_{r,t}$ la cantidad de productos fabricados en la región $r$ durante el año $t$.\\
\textbf{¿Qué representa la expresión $\sum_{r=1}^{5} P_{r,2022}$?}
\begin{enumerate}
    \item a) El total de productos fabricados en la región 5 a lo largo de varios años
    \item b) La producción de todas las regiones durante todos los años
    \item c) El número de regiones donde hubo producción en 2022
    \item d) La cantidad total de productos fabricados en 2022 en las cinco regiones
\end{enumerate}
}
\end{question}

\begin{question}{361}{índices}{1}{d}{8}{
\\ Sea $E = \{e_1, e_2, \dots, e_{150}\}$ un conjunto de estudiantes, y $Cod_i$ el código del estudiante $e_i$.\\
\textbf{¿Qué representa la proposición $\exists i \in \{1,\dots,150\}$ tal que $\forall j \in \{1,\dots,150\},\; Cod_i \leq Cod_j$?}
\begin{enumerate}
    \item a) Que todos los códigos son iguales
    \item b) Que el código $Cod_i$ es el mayor entre todos
    \item c) Que $Cod_i$ está en la mitad del orden
    \item d) Que el estudiante $i$ tiene el código más bajo
\end{enumerate}
}
\end{question}

\begin{question}{362}{índices}{1}{a}{8}{
\\ Sea $E = \{e_1, e_2, \dots, e_{150}\}$ un conjunto de estudiantes, y $Cod_i$ el código del estudiante $e_i$.\\
\textbf{¿Qué representa la proposición $\forall i \in \{1,\dots,150\},\; Cod_{i+1} \neq Cod_i + 1$?}
\begin{enumerate}
    \item a) Que no hay dos estudiantes con códigos consecutivos
    \item b) Que todos los códigos son consecutivos
    \item c) Que los códigos están desordenados
    \item d) Que el código de $e_1$ es menor que el de $e_2$
\end{enumerate}
}
\end{question}

\begin{question}{363}{índices}{1}{b}{8}{
\\ Sea $E = \{e_1, e_2, \dots, e_{150}\}$ un conjunto de estudiantes, y $Cod_i$ el código del estudiante $e_i$.\\
\textbf{¿Qué representa la proposición $\forall i \in \{1,\dots,150\},\; \exists j \in \{1,\dots,150\}$ tal que $Cod_i \geq Cod_j$?}
\begin{enumerate}
    \item a) Que todos los códigos son distintos
    \item b) Que para cada estudiante, hay otro con un código menor o igual
    \item c) Que todos tienen el mismo código
    \item d) Que hay un estudiante con el código más bajo
\end{enumerate}
}
\end{question}

\begin{question}{364}{índices}{2}{c}{8}{
\\ Sea $E = \{e_1, e_2, \dots, e_{150}\}$ un conjunto de estudiantes, y $Cod_i$ el código del estudiante $e_i$.\\
\textbf{¿Qué representa la proposición $\forall i, j \in \{1,\dots,150\},\; i \leq j \Rightarrow Cod_i \leq Cod_j$?}
\begin{enumerate}
    \item a) Que los códigos están repetidos
    \item b) Que todos los códigos son iguales
    \item c) Que los estudiantes están ordenados por código de forma creciente
    \item d) Que los índices y los códigos no guardan relación
\end{enumerate}
}
\end{question}

\begin{question}{365}{índices}{1}{b}{8}{
\\ ¿Sea $E = \{e_1, e_2, \dots, e_{150}\}$ un conjunto de estudiantes, y $Cod_i$ el código del estudiante $e_i$.\\
\textbf{¿Cuál es la negación de la proposición $\forall i \in \{1,\dots,150\},\; Cod_{i+1} = Cod_i + 1$?}
\begin{enumerate}
    \item a) $\forall i \in \{1,\dots,150\},\; Cod_{i+1} \neq Cod_i + 1$
    \item b) $\exists i \in \{1,\dots,150\}$ tal que $Cod_{i+1} \neq Cod_i + 1$
    \item c) $\exists i \in \{1,\dots,150\}$ tal que $Cod_{i+1} = Cod_i + 1$
    \item d) $\forall i \in \{1,\dots,150\},\; Cod_{i+1} < Cod_i$
\end{enumerate}
}
\end{question}

\begin{question}{366}{índices}{1}{a}{8}{
\\ Sea $E = \{e_1, e_2, \dots, e_{150}\}$ un conjunto de estudiantes, y $Cod_i$ el código del estudiante $e_i$.\\
\textbf{¿Qué representa la proposición $\exists i \in \{1,\dots,150\}$ tal que $Cod_i \geq 202510000$?}
\begin{enumerate}
    \item a) Que existe al menos un estudiante cuyo código empieza por 20251
    \item b) Que todos los códigos empiezan por 20251
    \item c) Que ningún estudiante es de primer semestre
    \item d) Que todos los estudiantes tienen códigos mayores a 202510000
\end{enumerate}
}
\end{question}

\begin{question}{367}{índices}{1}{c}{8}{
\\ Sea $E = \{e_1, e_2, \dots, e_{150}\}$ un conjunto de estudiantes, y $Cod_i$ el código del estudiante $e_i$.\\
\textbf{¿Cuál es la negación de la proposición $\exists i \in \{1,\dots,150\}$ tal que $Cod_i \geq 202510000$?}
\begin{enumerate}
    \item a) $\forall i \in \{1,\dots,150\},\; Cod_i \geq 202510000$
    \item b) $\forall i \in \{1,\dots,150\},\; Cod_i > 202510000$
    \item c) $\forall i \in \{1,\dots,150\},\; Cod_i < 202510000$
    \item d) $\exists i \in \{1,\dots,150\}$ tal que $Cod_i < 202510000$
\end{enumerate}
}
\end{question}

\begin{question}{368}{índices}{1}{a}{8}{
\\ Sea $E = \{e_1, e_2, \dots, e_{150}\}$ un conjunto de estudiantes, y $Cod_i$ el código del estudiante $e_i$. Donde los primeros 4 numeros del codigo indican el año de ingreso a la universidad.\\
\textbf{¿Qué representa la proposición $\forall i \in \{1,\dots,150\},\; Cod_i < 202600000$?}
\begin{enumerate}
    \item a) Que todos los estudiantes están matriculados antes del año 2026
    \item b) Que al menos un estudiante tiene código mayor que 202600000
    \item c) Que todos los códigos terminan en ceros
    \item d) Que ningún estudiante está matriculado en 2025
\end{enumerate}
}
\end{question}

\begin{question}{369}{índices}{1}{d}{8}{
\\ Sea $E = \{e_1, e_2, \dots, e_{150}\}$ un conjunto de estudiantes, y $Cod_i$ el código del estudiante $e_i$.\\
\textbf{¿Qué representa la proposición $\forall i \in \{1,\dots,150\},\; \exists j \in \{1,\dots,150\}$ tal que $Cod_i = Cod_j$?}
\begin{enumerate}
    \item a) Que todos los códigos son diferentes
    \item b) Que algún estudiante repitió código
    \item c) Que todos tienen códigos pares
    \item d) Que todos los códigos se repiten al menos una vez
\end{enumerate}
}
\end{question}

\begin{question}{370}{índices}{3}{c}{8}{
\\ Sea $E = \{e_1, e_2, \dots, e_{150}\}$ un conjunto de estudiantes, y $Cod_i$ el código del estudiante $e_i$. Pista $a\rightarrow b$ es equivalente a $\neg a \lor b$\\
\textbf{¿Cuál es la negación de la proposición $\forall i,j \in \{1,\dots,150\},\; i \leq j \Rightarrow Cod_i \leq Cod_j$?}
\begin{enumerate}
    \item a) $\forall i,j,\; i > j \Rightarrow Cod_i > Cod_j$
    \item b) $\exists i,j$ tales que $Cod_i = Cod_j$
    \item c) $\exists i,j$ tales que $i \leq j$ y $Cod_i > Cod_j$
    \item d) $\exists i$ tal que $Cod_i > Cod_{i+1}$
\end{enumerate}
}
\end{question}

\begin{question}{371}{índices}{1}{d}{8}{
\\ Sea $T_{i,j}$ el tiempo de respuesta (en segundos) del sensor $i$ durante la prueba $j$, con $i \in \{1,2,3\}$ y $j \in \{1,2,3\}$.\\
\textbf{¿Qué representa la expresión $\sum_{i=1}^{3} \sum_{j=1}^{3} T_{i,j}$?}
\begin{enumerate}
    \item a) El mayor tiempo registrado por un sensor
    \item b) El tiempo promedio de la prueba 3
    \item c) El tiempo mínimo registrado en todas las pruebas
    \item d) El tiempo total registrado por todos los sensores en todas las pruebas
\end{enumerate}
}
\end{question}

\begin{question}{372}{índices}{3}{c}{8}{
\\ Sea $P_{r,a}$ la cantidad de producto vendido en la región $r$ durante el año $a$. Siendo las regiones $r\in\{1,2,3,4\}$\\
\textbf{¿Qué representa la expresión $\sum_{r=1}^{4} P_{r,2023}$?}
\begin{enumerate}
    \item a) La cantidad de productos vendidos durante varios años
    \item b) La venta total del producto 2023
    \item c) Las ventas de 2023 en todas las regiones
    \item d) Las ventas acumuladas de la región 4
\end{enumerate}
}
\end{question}

\begin{question}{373}{índices}{1}{b}{8}{
\\ Sea $A_{c}$ el número de alumnos inscritos en el curso $c$. Con cursos $c\in\{1,2,3,4,5\}$\\
\textbf{¿Qué representa la expresión $\exists c \in \{1,\dots,5\}$ tal que $A_c > 100$?}
\begin{enumerate}
    \item a) Todos los cursos tienen más de 100 estudiantes
    \item b) Al menos un curso tiene más de 100 estudiantes
    \item c) Ningún curso tiene más de 100 estudiantes
    \item d) Exactamente cinco cursos tienen más de 100 estudiantes
\end{enumerate}
}
\end{question}

\begin{question}{374}{índices}{1}{a}{8}{
\\ Sea $R_{i,j}$ la calificación del estudiante $i$ en el parcial $j$. Con parciales $j\in\{1,2,3,4,5\}$ y con estudiantes $i\in\{1,2,3,4,5,\dots,90\}$\\
\textbf{¿Qué significa la proposición $\forall i,\; \exists j,\; R_{i,j} \geq 3.0$?}
\begin{enumerate}
    \item a) Todos los estudiantes pasaron al menos un parcial
    \item b) Todos los estudiantes pasaron todos los parciales
    \item c) Algún estudiante pasó todos los parciales
    \item d) Ningún estudiante pasó el curso
\end{enumerate}
}
\end{question}

\begin{question}{375}{índices}{1}{d}{8}{
\\ Sea $V_{m,d}$ el número de vehículos que circularon por la vía $m$ durante el día $d$.\\
\textbf{¿Qué representa la expresión $\sum_{d=1}^{30} V_{2,d}$?}
\begin{enumerate}
    \item a) El total de vehículos en todas las vías durante 30 días
    \item b) El total de vehículos en el día 2
    \item c) El número promedio de vehículos por vía
    \item d) El total de vehículos en la vía 2 durante 30 días
\end{enumerate}
}
\end{question}

\begin{question}{376}{índices}{1}{c}{8}{
\\ Sea $C_{i,j}$ el costo de envío desde la ciudad $i$ a la ciudad $j$. Con ciudades $i,j\in Ciudades$\\
\textbf{¿Qué representa la proposición $\forall i\in Ciudades,\; \exists j\in Ciudades,\; C_{i,j} < 50$?}
\begin{enumerate}
    \item a) Todos los envíos cuestan menos de 50
    \item b) Algún envío cuesta más de 50
    \item c) Desde cada ciudad hay al menos una ruta que cuesta menos de 50
    \item d) Todas las ciudades tienen tarifas iguales
\end{enumerate}
}
\end{question}

\begin{question}{377}{índices}{1}{b}{8}{
\\ Sea $M_{i,j}$ la matrícula del estudiante $j$ en el curso $i$, con $M_{i,j} = 1$ si está inscrito y $0$ si no. Con $i \in cursos$ y $j\in estudiantes$ siendo estos los conjuntos de cursos y de estuidantes repectivamente.\\
\textbf{¿Qué significa $\sum_{i\in cursos} M_{i,3}$?}
\begin{enumerate}
    \item a) El número de estudiantes inscritos en el curso 3
    \item b) La cantidad de cursos en los que el estudiante 3 está inscrito
    \item c) El número total de cursos disponibles
    \item d) El total de estudiantes en todos los cursos
\end{enumerate}
}
\end{question}

\begin{question}{378}{índices}{1}{a}{8}{
\\ Sea $H_{p,d}$ el número de horas trabajadas por el proyecto $p$ durante el día $d$. con $P$ como conjunto de proyectos y $D$ como conjunto de días.\\
\textbf{¿Qué representa la proposición $\forall p\in P,\; \forall d\in D,\; H_{p,d} > 0$?}
\begin{enumerate}
    \item a) Todos los proyectos trabajaron todos los días
    \item b) Ningún proyecto trabajó
    \item c) Solo un proyecto trabajó todos los días
    \item d) Cada proyecto trabajó exactamente una vez
\end{enumerate}
}
\end{question}

\begin{question}{379}{índices}{1}{c}{8}{
\\ Sea $T_{i,j}$ el tiempo en segundos que tarda la transacción $j$ que este en el conjunto $T$ en el servidor $i$ que este en el conjunto S.\\
\textbf{¿Qué representa la negación de $\forall i\in S,\; \exists j\in T,\; T_{i,j} < 1$?}
\begin{enumerate}
    \item a) $\forall i\in S,\; \forall j\in T,\; T_{i,j} \geq 1$
    \item b) $\exists i\in S,\; \exists j\in T,\; T_{i,j} \geq 1$
    \item c) $\exists i\in S,\; \forall j\in T,\; T_{i,j} \geq 1$
    \item d) $\exists i\in S,\; \exists j\in T,\; T_{i,j} < 1$
\end{enumerate}
}
\end{question}

\begin{question}{380}{índices}{1}{d}{8}{
\\ Sea $L_{i,j}$ la longitud en metros de la calle $j$ que pertenezca al conjunto $Calles$ en el distrito $i$ tal que este en el conjunto $Distritos$.\\
\textbf{¿Cuál es la interpretación correcta de $\exists i \in Distritros$ tal que $\forall j \in Calles,\; L_{i,j} > 0$?}
\begin{enumerate}
    \item a) Todas las calles de todas las regiones son mayores a cero
    \item b) Al menos una calle tiene longitud positiva
    \item c) Todas las calles tienen longitud positiva
    \item d) Existe un distrito en el que todas sus calles tienen longitud positiva
\end{enumerate}
}
\end{question}

\begin{question}{381}{conjuntos}{1}{c}{2}{
\\ Sean $A$ y $B$ subconjuntos de $X$ para cada .\\
\textbf{¿Qué representa el conjunto } $(A \cap B)$?
\begin{enumerate}
    \item a) Los elementos que están en $A$ y en  $B$
    \item b) Los elementos que están en $A$ o en $B$
    \item c) Ninguna de las anteriores
    \item d) Los elementos que están en $A$ pero no en $B$
\end{enumerate}
}
\end{question}

\begin{question}{382}{conjuntos}{1}{b}{2}{
\\ Sean $A$, $B$ y $C$ conjuntos.\\
\textbf{¿Qué representa la expresión } $(A \cap B) \cup (A \cap C)$?
\begin{enumerate}
    \item a) $A \cap (B \cup C)$
    \item b) Los elementos que están en $A$ y en $B$ o en $C$
    \item c) Los elementos que están solo en $B$ o en $C$
    \item d) $A \cup (B \cap C)$
\end{enumerate}
}
\end{question}

\begin{question}{383}{conjuntos}{1}{d}{2}{
\\ Sea $X$ un conjunto universal y $A \subseteq X$.\\
\textbf{¿Qué representa } $X \setminus (X \setminus A)$?
\begin{enumerate}
    \item a) El conjunto vacío
    \item b) $X$
    \item c) $X \setminus A$
    \item d) $A$
\end{enumerate}
}
\end{question}

\begin{question}{384}{conjuntos}{1}{a}{2}{
\\ Sean $A$ y $B$ subconjuntos de un conjunto $X$.\\
\textbf{¿Cuál de las siguientes afirmaciones es verdadera?}
\begin{enumerate}
    \item a) $A \subseteq A \cup B$
    \item b) $A \subseteq A \cap B$
    \item c) $A \cup B \subseteq A$
    \item d) $A \cap B \subseteq A \cup B$ es falsa
\end{enumerate}
}
\end{question}

\begin{question}{385}{conjuntos}{1}{c}{2}{
\\ Sea $A \subseteq B$ y $C \subseteq D$.\\
\textbf{¿Cuál es una consecuencia válida de estas inclusiones?}
\begin{enumerate}
    \item a) $A \cap C \subseteq B \cup D$
    \item b) $A \cup C \subseteq B \cap D$
    \item c) $A \cup C \subseteq B \cup D$
    \item d) $B \cup D \subseteq A \cup C$
\end{enumerate}
}
\end{question}

\begin{question}{386}{conjuntos}{1}{d}{2}{
\\ Sean $A$ y $B$ conjuntos tales que $A \cap B = \emptyset$.\\
\textbf{¿Qué representa } $(A \cup B) \setminus B$?
\begin{enumerate}
    \item a) $A \cup B$
    \item b) $B$
    \item c) $\emptyset$
    \item d) $A$
\end{enumerate}
}
\end{question}

\begin{question}{387}{conjuntos}{1}{b}{2}{
\\ Sea $A$, $B$ y $C$ subconjuntos de $X$.\\
\textbf{¿Cuál es la forma correcta de distribuir la intersección sobre la unión?}
\begin{enumerate}
    \item a) $A \cap (B \cup C) = (A \cup B) \cap C$
    \item b) $A \cap (B \cup C) = (A \cap B) \cup (A \cap C)$
    \item c) $A \cap (B \cup C) = A \cup (B \cap C)$
    \item d) $A \cap (B \cup C) = (A \cup B) \cup (A \cup C)$
\end{enumerate}
}
\end{question}

\begin{question}{388}{conjuntos}{1}{a}{2}{
\\ Sean $A$ y $B$ subconjuntos de $X$ con $A \subseteq B$.\\
\textbf{¿Cuál de las siguientes afirmaciones es verdadera?}
\begin{enumerate}
    \item a) $(B \setminus A) \cup A = B$
    \item b) $A \setminus B = A$
    \item c) $B \subseteq A$
    \item d) $A \cup B = A$
\end{enumerate}
}
\end{question}

\begin{question}{389}{conjuntos}{1}{c}{2}{
\\ Sean $A$ y $B$ conjuntos.\\
\textbf{¿Qué representa } $(A \cup B)^c$ en un conjunto universal $U$?
\begin{enumerate}
    \item a) $A^c \cap B$
    \item b) $A \cap B$
    \item c) $A^c \cap B^c$
    \item d) $A^c \cup B^c$
\end{enumerate}
}
\end{question}

\begin{question}{390}{conjuntos}{1}{d}{2}{
\\ Sea $A$ subconjunto de un conjunto finito $X$.\\
\textbf{¿Cuál es la relación entre } $A$ y su complemento $A^c$?
\begin{enumerate}
    \item a) $A \cap A^c = A$
    \item b) $A \cup A^c = A$
    \item c) $A^c \subseteq A$
    \item d) $A \cap A^c = \emptyset$ y $A \cup A^c = X$
\end{enumerate}
}
\end{question}

\begin{question}{391}{conjuntos}{1}{a}{2}{
\\ Sean $A$, $B$ y $C$ subconjuntos de un conjunto universal $U$.\\
\textbf{¿Qué expresión es equivalente a } $(A \cap B)^c$?
\begin{enumerate}
    \item a) $A^c \cup B^c$
    \item b) $A^c \cap B^c$
    \item c) $A \cup B$
    \item d) $A^c \setminus B$
\end{enumerate}
}
\end{question}

\begin{question}{392}{conjuntos}{1}{b}{2}{
\\ Sean $A$ y $B$ subconjuntos de un conjunto universal $U$.\\
\textbf{¿Cuál de las siguientes expresiones representa los elementos que están solo en uno de los dos conjuntos, pero no en ambos?}
\begin{enumerate}
    \item a) $A \cap B$
    \item b) $(A \setminus B) \cup (B \setminus A)$
    \item c) $A \cup B$
    \item d) $U \setminus (A \cup B)$
\end{enumerate}
}
\end{question}

\begin{question}{393}{conjuntos}{1}{d}{2}{
\\ Sean $A$, $B$ subconjuntos de $X$.\\
\textbf{¿Cuál es la forma equivalente de } $A \setminus (A \cap B)$?
\begin{enumerate}
    \item a) $B \setminus A$
    \item b) $A \cap B$
    \item c) $(A \cup B)^c$
    \item d) $A \cap B^c$
\end{enumerate}
}
\end{question}

\begin{question}{394}{conjuntos}{1}{c}{2}{
\\ Sea $A \subseteq B \subseteq C$.\\
\textbf{¿Cuál de las siguientes afirmaciones es verdadera?}
\begin{enumerate}
    \item a) $A \cup C \subseteq B$
    \item b) $C \subseteq B \subseteq A$
    \item c) $A \subseteq C$
    \item d) $B \setminus C \neq \emptyset$
\end{enumerate}
}
\end{question}

\begin{question}{395}{conjuntos}{1}{b}{2}{
\\ Sean $A$, $B$ y $C$ subconjuntos de $X$.\\
\textbf{¿Qué representa la expresión } $(A \cup B) \cap (A \cup C)$?
\begin{enumerate}
    \item a) $(A \cup B) \cap C$
    \item b) $A \cup (B \cap C)$
    \item c) $A \cap (B \cup C)$
    \item d) $(A \cap B) \cup C$
\end{enumerate}
}
\end{question}

\begin{question}{396}{conjuntos}{1}{a}{2}{
\\ Sean $A$ y $B$ subconjuntos de $X$.\\
\textbf{¿Cuál es la diferencia simétrica entre $A$ y $B$?}
\begin{enumerate}
    \item a) $(A \setminus B) \cup (B \setminus A)$
    \item b) $(A \cup B)^c$
    \item c) $(A \cap B)^c$
    \item d) $A \cup B$
\end{enumerate}
}
\end{question}

\begin{question}{397}{conjuntos}{1}{c}{2}{
\\ Sean $A$ y $B$ subconjuntos de un conjunto $U$.\\
\textbf{¿Qué representa } $(A^c \cup B^c)^c$?
\begin{enumerate}
    \item a) $A \cup B$
    \item b) $A^c \cap B^c$
    \item c) $A \cap B$
    \item d) $(A \cup B)^c$
\end{enumerate}
}
\end{question}

\begin{question}{398}{conjuntos}{1}{d}{2}{
\\ Sea $A$ un subconjunto de $U$, con $U$ conjunto universal.\\
\textbf{¿Qué conjunto contiene exactamente los elementos que no están en $A$?}
\begin{enumerate}
    \item a) $U$
    \item b) $A$
    \item c) $\emptyset$
    \item d) $A^c$
\end{enumerate}
}
\end{question}

\begin{question}{399}{conjuntos}{1}{a}{2}{
\\ Sean $A = \{x \in \mathbb{N} \mid x \text{ es par}\}$ y $B = \{x \in \mathbb{N} \mid x \text{ es múltiplo de 4}\}$.\\
\textbf{¿Cuál de las siguientes es verdadera?}
\begin{enumerate}
    \item a) $B \subseteq A$
    \item b) $A \subseteq B$
    \item c) $A = B$
    \item d) $A \cap B = \emptyset$
\end{enumerate}
}
\end{question}

\begin{question}{400}{conjuntos}{1}{c}{2}{
\\ Sea $A = \{x \in \mathbb{Z} \mid x < 0\}$ y $B = \{x \in \mathbb{Z} \mid x > 0\}$.\\
\textbf{¿Cuál es el valor de } $A \cap B$?
\begin{enumerate}
    \item a) $\mathbb{Z}$
    \item b) $A$
    \item c) $\emptyset$
    \item d) $A \cup B$
\end{enumerate}
}
\end{question}


\begin{question}{401}{conjuntos}{1}{c}{2}{
\\ Sea $A = \{1,2,3,4,5\}$ y $B = \{4,5,6,7\}$.\\
\textbf{¿Cuántos elementos hay en } $A \cap B$?
\begin{enumerate}
    \item a) 1
    \item b) 3
    \item c) 2
    \item d) 4
\end{enumerate}
}
\end{question}

\begin{question}{402}{conjuntos}{1}{b}{2}{
\\ Sea $A = \{x \in \mathbb{N} \mid x \leq 6\}$ y $B = \{x \in \mathbb{N} \mid x \geq 4\}$.\\
\textbf{¿Cuántos elementos hay en } $A \cap B$?
\begin{enumerate}
    \item a) 2
    \item b) 3
    \item c) 4
    \item d) 5
\end{enumerate}
}
\end{question}

\begin{question}{403}{conjuntos}{1}{a}{2}{
\\ Un grupo de 20 estudiantes: 12 practican baloncesto y 8 practican voleibol. Si 5 practican ambos deportes, \\
\textbf{¿Cuántos practican únicamente baloncesto?}
\begin{enumerate}
    \item a) 7
    \item b) 5
    \item c) 8
    \item d) 12
\end{enumerate}
}
\end{question}

\begin{question}{404}{conjuntos}{1}{d}{2}{
\\ Sea $A = \{x \in \mathbb{Z} \mid -3 \leq x \leq 3\}$ y $B = \{x \in \mathbb{Z} \mid x \text{ es par}\}$.\\
\textbf{¿Cuántos elementos tiene } $A \cap B$?
\begin{enumerate}
    \item a) 2
    \item b) 3
    \item c) 4
    \item d) 5
\end{enumerate}
}
\end{question}

\begin{question}{405}{conjuntos}{1}{b}{2}{
\\ En una encuesta: 40 personas tienen mascota, 25 tienen perro y 20 gato. Si 10 tienen ambos, \\
\textbf{¿Cuántas personas tienen solo gato?}
\begin{enumerate}
    \item a) 8
    \item b) 10
    \item c) 12
    \item d) 15
\end{enumerate}
}
\end{question}

\begin{question}{406}{conjuntos}{1}{c}{2}{
\\ Sea $A = \{1,3,5,7,9\}$ y $B = \{2,3,4,5,6\}$.\\
\textbf{¿Cuántos elementos hay en } $A \cap B$?
\begin{enumerate}
    \item a) 1
    \item b) 3
    \item c) 2
    \item d) 4
\end{enumerate}
}
\end{question}

\begin{question}{407}{conjuntos}{1}{a}{2}{
\\ En un salón hay 30 estudiantes: 18 hablan inglés, 12 hablan francés y 5 hablan ambos. \\
\textbf{¿Cuántos hablan solo inglés?}
\begin{enumerate}
    \item a) 13
    \item b) 12
    \item c) 10
    \item d) 5
\end{enumerate}
}
\end{question}

\begin{question}{408}{conjuntos}{1}{b}{2}{
\\ Sea $A = \{x \in \mathbb{Z} \mid -2 \leq x \leq 4\}$ y $B = \{x \in \mathbb{Z} \mid x > 0\}$.\\
\textbf{¿Cuántos elementos tiene } $A \cap B$?
\begin{enumerate}
    \item a) 3
    \item b) 4
    \item c) 5
    \item d) 6
\end{enumerate}
}
\end{question}

\begin{question}{409}{conjuntos}{1}{c}{2}{
\\ En un club hay 50 personas: 30 practican natación, 25 tenis y 15 ambos deportes. \\
\textbf{¿Cuántas personas practican solo tenis?}
\begin{enumerate}
    \item a) 8
    \item b) 10
    \item c) 10
    \item d) 15
\end{enumerate}
}
\end{question}

\begin{question}{410}{conjuntos}{1}{d}{2}{
\\ Sea $A = \{2,4,6,8,10\}$ y $B = \{1,2,3,4,5,6\}$.\\
\textbf{¿Cuántos elementos hay en } $A \cap B$?
\begin{enumerate}
    \item a) 2
    \item b) 4
    \item c) 3
    \item d) 3
\end{enumerate}
}
\end{question}
\begin{question}{411}{conjuntos}{1}{b}{2}{
\\ Sea $A = \{x \in \mathbb{N} \mid x \leq 8\}$ y $B = \{x \in \mathbb{N} \mid x \text{ es par}\}$.\\
\textbf{¿Cuántos elementos tiene } $A \cap B$?
\begin{enumerate}
    \item a) 3
    \item b) 4
    \item c) 5
    \item d) 6
\end{enumerate}
}
\end{question}

\begin{question}{412}{conjuntos}{1}{c}{2}{
\\ En un grupo de 60 personas: 35 asisten a clases de pintura, 25 a clases de música y 15 a ambas. \\
\textbf{¿Cuántas personas asisten solo a música?}
\begin{enumerate}
    \item a) 8
    \item b) 12
    \item c) 10
    \item d) 15
\end{enumerate}
}
\end{question}

\begin{question}{413}{conjuntos}{1}{a}{2}{
\\ Sea $A = \{1,2,3,4\}$ y $B = \{3,4,5,6,7\}$.\\
\textbf{¿Cuántos elementos hay en } $A \cap B$?
\begin{enumerate}
    \item a) 2
    \item b) 3
    \item c) 4
    \item d) 5
\end{enumerate}
}
\end{question}

\begin{question}{414}{conjuntos}{1}{d}{2}{
\\ En un colegio: 40 estudiantes practican ajedrez, 30 practican fútbol y 20 practican ambos. \\
\textbf{¿Cuántos practican solo ajedrez?}
\begin{enumerate}
    \item a) 15
    \item b) 10
    \item c) 20
    \item d) 20
\end{enumerate}
}
\end{question}

\begin{question}{415}{conjuntos}{1}{b}{2}{
\\ Sea $A = \{x \in \mathbb{Z} \mid -5 \leq x \leq 5\}$ y $B = \{x \in \mathbb{Z} \mid x \text{ es múltiplo de 3}\}$.\\
\textbf{¿Cuántos elementos tiene } $A \cap B$?
\begin{enumerate}
    \item a) 3
    \item b) 4
    \item c) 5
    \item d) 6
\end{enumerate}
}
\end{question}

\begin{question}{416}{conjuntos}{1}{a}{2}{
\\ En un taller: 15 personas saben programar en Python, 10 en Java y 5 en ambos. \\
\textbf{¿Cuántas personas saben solo Python?}
\begin{enumerate}
    \item a) 10
    \item b) 5
    \item c) 8
    \item d) 15
\end{enumerate}
}
\end{question}

\begin{question}{417}{conjuntos}{1}{c}{2}{
\\ Sea $A = \{2,4,6,8\}$ y $B = \{1,3,5,7,9\}$.\\
\textbf{¿Cuántos elementos tiene } $A \cap B$?
\begin{enumerate}
    \item a) 1
    \item b) 2
    \item c) 0
    \item d) 3
\end{enumerate}
}
\end{question}

\begin{question}{418}{conjuntos}{1}{b}{2}{
\\ En un grupo de 50 personas: 20 tienen bicicleta, 30 tienen carro y 10 tienen ambos. \\
\textbf{¿Cuántas personas tienen solo carro?}
\begin{enumerate}
    \item a) 18
    \item b) 20
    \item c) 25
    \item d) 15
\end{enumerate}
}
\end{question}

\begin{question}{419}{conjuntos}{1}{a}{2}{
\\ Sea $A = \{x \in \mathbb{Z} \mid 1 \leq x \leq 5\}$ y $B = \{x \in \mathbb{Z} \mid x \geq 4\}$.\\
\textbf{¿Cuántos elementos hay en } $A \cap B$?
\begin{enumerate}
    \item a) 2
    \item b) 3
    \item c) 4
    \item d) 5
\end{enumerate}
}
\end{question}

\begin{question}{420}{conjuntos}{1}{d}{2}{
\\ En un vecindario: 25 personas tienen perro, 15 tienen gato y 5 tienen ambos. \\
\textbf{¿Cuántas personas tienen perro o gato (o ambos)?}
\begin{enumerate}
    \item a) 30
    \item b) 35
    \item c) 25
    \item d) 35
\end{enumerate}
}
\end{question}
\begin{question}{421}{conjuntos}{1}{c}{2}{
\\ Sea $A = \{x \in \mathbb{N} \mid x \leq 10\}$ y $B = \{x \in \mathbb{N} \mid x \text{ es múltiplo de 4}\}$.\\
\textbf{¿Cuántos elementos hay en } $A \cap B$?
\begin{enumerate}
    \item a) 2
    \item b) 4
    \item c) 3
    \item d) 5
\end{enumerate}
}
\end{question}

\begin{question}{422}{conjuntos}{1}{a}{2}{
\\ En un grupo de 40 personas: 20 tocan guitarra, 18 tocan piano y 8 tocan ambos. \\
\textbf{¿Cuántas personas tocan solo guitarra?}
\begin{enumerate}
    \item a) 12
    \item b) 10
    \item c) 15
    \item d) 8
\end{enumerate}
}
\end{question}

\begin{question}{423}{conjuntos}{1}{b}{2}{
\\ Sea $A = \{1,3,5,7,9\}$ y $B = \{2,3,4,5,6,7\}$.\\
\textbf{¿Cuántos elementos tiene } $A \cap B$?
\begin{enumerate}
    \item a) 2
    \item b) 3
    \item c) 4
    \item d) 5
\end{enumerate}
}
\end{question}

\begin{question}{424}{conjuntos}{1}{d}{2}{
\\ En una oficina: 12 empleados saben Excel, 15 saben Word y 5 saben ambos. \\
\textbf{¿Cuántos empleados saben Excel o Word (o ambos)?}
\begin{enumerate}
    \item a) 22
    \item b) 17
    \item c) 20
    \item d) 22
\end{enumerate}
}
\end{question}

\begin{question}{425}{conjuntos}{1}{a}{2}{
\\ Sea $A = \{x \in \mathbb{Z} \mid -4 \leq x \leq 4\}$ y $B = \{x \in \mathbb{Z} \mid x \text{ es impar}\}$.\\
\textbf{¿Cuántos elementos tiene } $A \cap B$?
\begin{enumerate}
    \item a) 5
    \item b) 6
    \item c) 4
    \item d) 7
\end{enumerate}
}
\end{question}

\begin{question}{426}{conjuntos}{1}{b}{2}{
\\ En una competencia: 50 personas corren, 30 nadan y 20 hacen ambas actividades. \\
\textbf{¿Cuántas personas solo nadan?}
\begin{enumerate}
    \item a) 15
    \item b) 10
    \item c) 12
    \item d) 20
\end{enumerate}
}
\end{question}

\begin{question}{427}{conjuntos}{1}{c}{2}{
\\ Sea $A = \{2,4,6,8,10\}$ y $B = \{5,6,7,8,9\}$.\\
\textbf{¿Cuántos elementos tiene } $A \cap B$?
\begin{enumerate}
    \item a) 1
    \item b) 3
    \item c) 2
    \item d) 4
\end{enumerate}
}
\end{question}

\begin{question}{428}{conjuntos}{1}{a}{2}{
\\ En un curso: 40 estudiantes aprobaron matemáticas, 35 aprobaron física y 25 aprobaron ambas. \\
\textbf{¿Cuántos aprobaron solo matemáticas?}
\begin{enumerate}
    \item a) 15
    \item b) 10
    \item c) 12
    \item d) 20
\end{enumerate}
}
\end{question}

\begin{question}{429}{conjuntos}{1}{b}{2}{
\\ Sea $A = \{x \in \mathbb{N} \mid 1 \leq x \leq 12\}$ y $B = \{x \in \mathbb{N} \mid x \text{ es múltiplo de 5}\}$.\\
\textbf{¿Cuántos elementos hay en } $A \cap B$?
\begin{enumerate}
    \item a) 1
    \item b) 2
    \item c) 3
    \item d) 4
\end{enumerate}
}
\end{question}

\begin{question}{430}{conjuntos}{1}{d}{2}{
\\ En un club: 28 personas practican ciclismo, 24 practican atletismo y 12 practican ambos. \\
\textbf{¿Cuántas personas practican ciclismo o atletismo (o ambos)?}
\begin{enumerate}
    \item a) 40
    \item b) 44
    \item c) 36
    \item d) 40
\end{enumerate}
}
\end{question}
\begin{question}{431}{proposiciones}{1}{a}{2}{
Sea $p$: Hoy llueve y $q$: Tengo paraguas.\\
¿Cuál de las siguientes proposiciones representa “Hoy no llueve y no tengo paraguas”?
\begin{enumerate}
    \item a) $\neg p \wedge \neg q$
    \item b) $\neg (p \wedge q)$
    \item c) $\neg p \vee \neg q$
    \item d) $p \vee q$
\end{enumerate}
}
\end{question}

\begin{question}{432}{proposiciones}{1}{b}{2}{
Sean las proposiciones $p$: El número es par y $q$: El número es múltiplo de 3.\\
¿Cuál es la proposición que expresa El número es par o múltiplo de 3?
\begin{enumerate}
    \item a) $p \wedge q$
    \item b) $p \vee q$
    \item c) $\neg p \vee q$
    \item d) $\neg (p \wedge q)$
\end{enumerate}
}
\end{question}

\begin{question}{433}{proposiciones}{1}{a}{2}{
Considere las preposiciones $p$: Estudió y $q$: Apruebo.\\
¿Cuál de las siguientes tablas de verdad corresponde a $p \wedge q$?
\begin{enumerate}
    \item a) 
    \begin{tabular}{|c|c|c|}
    \hline
    $p$ & $q$ & $p \wedge q$\\
    \hline
    V & V & V\\
    V & F & F\\
    F & V & F\\
    F & F & F\\
    \hline
    \end{tabular}
    \item b) 
    \begin{tabular}{|c|c|c|}
    \hline
    $p$ & $q$ & $p \wedge q$\\
    \hline
    V & V & F\\
    V & F & V\\
    F & V & F\\
    F & F & F\\
    \hline
    \end{tabular}
    \item c) 
    \begin{tabular}{|c|c|c|}
    \hline
    $p$ & $q$ & $p \wedge q$\\
    \hline
    V & V & V\\
    V & F & V\\
    F & V & F\\
    F & F & F\\
    \hline
    \end{tabular}
    \item d) 
    \begin{tabular}{|c|c|c|}
    \hline
    $p$ & $q$ & $p \wedge q$\\
    \hline
    V & V & F\\
    V & F & F\\
    F & V & F\\
    F & F & V\\
    \hline
    \end{tabular}
\end{enumerate}
}
\end{question}

\begin{question}{434}{traducciones}{1}{c}{2}{
Traduce al lenguaje matemático: El perro no es grande o el gato no es pequeño. Sea $p$: El perro es grande, $q$: El gato es pequeño.
\begin{enumerate}
    \item a) $\neg p \vee \neg q$
    \item b) $\neg (p \vee q)$
    \item c) $\neg (p \wedge q)$
    \item d) $p \wedge q$
\end{enumerate}
}
\end{question}

\begin{question}{435}{traducciones}{1}{a}{2}{
Escribe en palabras la proposición $\neg p \vee q$ donde:\\
$p$: Está soleado\\
$q$: Llevo sombrero
\begin{enumerate}
    \item a) “No está soleado o llevo sombrero”
    \item b) “Está soleado o llevo sombrero”
    \item c) “No está soleado y llevo sombrero”
    \item d) “Está soleado y llevo sombrero”
\end{enumerate}
}
\end{question}

\begin{question}{436}{Conjuntos}{1}{b}{2}{
Si $A = \{1,2,3\}$, $B = \{3,4,5\}$, y $p$: 3 es un elemento de $A$, $q$: 3 es un elemento de $B$.\\
El conjunto $\{3\}$ se obteniene con cual de las siguientes operaciones:
\begin{enumerate}
    \item a) La intersección $A \cap B$
    \item b) La unión $A \cup B$
    \item c) La diferencia $A - B$
    \item d) Un conjunto vacío
\end{enumerate}
}
\end{question}

\begin{question}{437}{conjuntos}{1}{d}{2}{
Sea el universo los naturales sin el 0 y los siguientes conjuntos:\\ $A = \{x \in \mathbb{N} \mid x \text{ es número par menor que 10}\}$ y $B = \{x \in \mathbb{N} \mid x \text{ es múltiplo de 3 menor que 10}\}$.
\\
\\
¿Cuántos elementos hay en $A \cap B$?
\begin{enumerate}
    \item a) 0
    \item b) 1
    \item c) 2
    \item d) 3
\end{enumerate}
}
\end{question}

\begin{question}{438}{proposiciones}{1}{a}{2}{
Sea $p$: Leo un libro y $q$: Escucho música.\\
¿Qué proposición corresponde a Leo un libro o no escucho música?
\begin{enumerate}
    \item a) $p \vee \neg q$
    \item b) $\neg p \vee q$
    \item c) $p \wedge \neg q$
    \item d) $\neg (p \vee q)$
\end{enumerate}
}
\end{question}

\begin{question}{439}{traducciones}{1}{a}{2}{
Traduce a palabras: $\neg (p \vee q)$ donde\\
$p$: Pedro estudia\\
$q$: Pedro trabaja.
\begin{enumerate}
    \item a) “Pedro no estudia y Pedro no trabaja”
    \item b) “Pedro no estudia o Pedro no trabaja”
    \item c) “Pedro estudia y trabaja”
    \item d) “Pedro estudia o trabaja”
\end{enumerate}
}
\end{question}

\begin{question}{440}{conjuntos}{1}{c}{2}{
Si $A = \{2,4,6,8\}$ y $B = \{4,8,12,16\}$, ¿qué representa $A \cup B$?
\begin{enumerate}
    \item a) Todos los elementos que están solo en $A$
    \item b) Todos los elementos que están en $A$ y en $B$
    \item c) Todos los elementos que están en $A$ o en $B$.
    \item d) Ninguna de las anteriores
\end{enumerate}
}
\end{question}
\begin{question}{441}{proposiciones}{1}{c}{2}{
Sea $p$: La lámpara está encendida y $q$: La ventana está abierta.\\
¿Cuál de las siguientes proposiciones representa “La lámpara está encendida o la ventana no está abierta”?
\begin{enumerate}
    \item a) $p \wedge \neg q$
    \item b) $\neg p \wedge q$
    \item c) $p \vee \neg q$
    \item d) $\neg p \vee q$
\end{enumerate}
}
\end{question}

\begin{question}{442}{proposiciones}{1}{a}{2}{
Sea $p$: Estudio para el examen y $q$: Obtengo buena nota.\\
¿Cuál proposición corresponde a “No estudio para el examen o obtengo buena nota”?
\begin{enumerate}
    \item a) $\neg p \vee q$
    \item b) $p \vee q$
    \item c) $\neg p \wedge q$
    \item d) $p \wedge q$
\end{enumerate}
}
\end{question}

\begin{question}{443}{proposiciones}{1}{a}{2}{
¿Cuál tabla de verdad corresponde a $p \vee q$?
\begin{enumerate}
    \item a) 
    \begin{tabular}{|c|c|c|}
    \hline
    $p$ & $q$ & $p \vee q$\\
    \hline
    V & V & V\\
    V & F & V\\
    F & V & V\\
    F & F & F\\
    \hline
    \end{tabular}
    \item b) 
    \begin{tabular}{|c|c|c|}
    \hline
    $p$ & $q$ & $p \vee q$\\
    \hline
    V & V & F\\
    V & F & V\\
    F & V & F\\
    F & F & F\\
    \hline
    \end{tabular}
    \item c) 
    \begin{tabular}{|c|c|c|}
    \hline
    $p$ & $q$ & $p \vee q$\\
    \hline
    V & V & V\\
    V & F & F\\
    F & V & V\\
    F & F & F\\
    \hline
    \end{tabular}
    \item d) 
    \begin{tabular}{|c|c|c|}
    \hline
    $p$ & $q$ & $p \vee q$\\
    \hline
    V & V & V\\
    V & F & V\\
    F & V & F\\
    F & F & V\\
    \hline
    \end{tabular}
\end{enumerate}
}
\end{question}

\begin{question}{444}{traducciones}{1}{c}{2}{
Sea $p$: Juan corre y $q$: Pedro camina.\\
Escribe en símbolos: Juan corre o Pedro no camina.
\begin{enumerate}
    \item a) $p \vee \neg q$
    \item b) $\neg p \vee q$
    \item c) $p \lor \neg q$
    \item d) $\neg p \wedge q$
\end{enumerate}
}
\end{question}

\begin{question}{445}{traducciones}{1}{a}{2}{
Interpreta la proposición $\neg p \wedge q$ donde\\
$p$: El tren ha llegado\\
$q$: Estoy en la estación.
\begin{enumerate}
    \item a) “El tren no ha llegado y estoy en la estación”
    \item b) “El tren ha llegado y estoy en la estación”
    \item c) “El tren no ha llegado o estoy en la estación”
    \item d) “Estoy en la estación y el tren ha llegado”
\end{enumerate}
}
\end{question}

\begin{question}{446}{conjuntos}{1}{b}{2}{
Si $A = \{1,3,5,7\}$ y $B = \{3,4,5,6\}$, ¿cuántos elementos tiene $A \cap B$?
\begin{enumerate}
    \item a) 1
    \item b) 2
    \item c) 3
    \item d) 4
\end{enumerate}
}
\end{question}

\begin{question}{447}{conjuntos}{1}{b}{2}{
Sea el univeso el conjunto de los naturales sin el 0 y los siguientes conjuntos: \\ $A = \{x \in \mathbb{N} \mid x \text{ es múltiplo de 2 menor que 12}\}$ y $B = \{x \in \mathbb{N} \mid x \text{ es múltiplo de 4 menor que 12}\}$.\\
\\
¿Cuál es $A \cap B$?
\begin{enumerate}
    \item a) $\{4,8,12\}$
    \item b) $\{4,8\}$
    \item c) $\{2,4,6,8,10\}$
    \item d) $\{2,6,10\}$
\end{enumerate}
}
\end{question}

\begin{question}{448}{proposiciones}{1}{a}{2}{
Sea $p$: Está nevando y $q$: Hace frío.\\
¿Cuál proposición representa “Está nevando y hace frío”?
\begin{enumerate}
    \item a) $p \wedge q$
    \item b) $p \vee q$
    \item c) $\neg p \wedge q$
    \item d) $\neg p \vee \neg q$
\end{enumerate}
}
\end{question}

\begin{question}{449}{traducciones}{1}{a}{2}{
Traduce a lenguaje natural: $\neg (p \wedge q)$, donde\\
$p$: Compro pan\\
$q$: Compro leche.
\begin{enumerate}
    \item a) “No compro pan o no compro leche”
    \item b) “No compro pan y no compro leche”
    \item c) “Compro pan o compro leche”
    \item d) “Compro pan y compro leche”
\end{enumerate}
}
\end{question}

\begin{question}{450}{conjunto}{1}{a}{2}{
Si $A = \{a,b,c\}$ y $B = \{b,c,d\}$, ¿qué conjunto representa $A \cup B$?
\begin{enumerate}
    \item a) $\{a,b,c,d\}$
    \item b) $\{b,c\}$
    \item c) $\{a,d\}$
    \item d) $\emptyset$
\end{enumerate}
}
\end{question}
% ---- ARGUMENTOS VÁLIDOS E INVÁLIDOS ----

\begin{question}{451}{argumentos}{1}{b}{6}{
Sea el argumento: Todos los mamíferos respiran aire. Los delfines respiran aire. Por lo tanto, los delfines son mamíferos. Este argumento es:
\begin{enumerate}
    \item a) Válido
    \item b) Inválido
    
\end{enumerate}
}
\end{question}

\begin{question}{452}{argumentos}{1}{a}{6}{
Considere: Solo si llueve, la calle se moja. La calle está mojada, por lo tanto llovió.\\
Este razonamiento es:
\begin{enumerate}
    \item a) Válido
    \item b) Inválido 
\end{enumerate}
}
\end{question}

\begin{question}{453}{argumentos}{1}{a}{6}{
Solo si estudio, apruebo el examen. No aprobé el examen, por lo tanto no estudié.\\
Este argumento es:
\begin{enumerate}
    \item a) Válido 
    \item b) Inválido
\end{enumerate}
}
\end{question}

\begin{question}{454}{argumentos}{1}{a}{2}{
Todos los pájaros vuelan. Los pingüinos no vuelan. Por lo tanto, los pingüinos no son pájaros. El argumento es:
\begin{enumerate}
    \item a) Válido
    \item b) Inválido 
\end{enumerate}
}
\end{question}

\begin{question}{455}{argumentos}{1}{a}{2}{
Los estudiantes aprobaron el examen. Juan es estudiante, por lo tanto Juan aprobó el examen.
Este argumento es:
\begin{enumerate}
    \item a) Válido
    \item b) Inválido 
\end{enumerate}
}
\end{question}

% ---- CUANTIFICADORES ----

\begin{question}{456}{cuantificadores}{1}{a}{7}{
Traduzca al lenguaje lógico: Todos los números naturales son mayores o iguales que 0.\\
\begin{enumerate}
    \item a) $\forall n \in \mathbb{N},\ n \geq 0$
    \item b) $\exists n \in \mathbb{N},\ n \geq 0$
    \item c) $\forall n \in \mathbb{N},\ n > 0$
    \item d) $\exists n \in \mathbb{N},\ n > 0$
\end{enumerate}
}
\end{question}

\begin{question}{457}{cuantificadores}{1}{a}{7}{
Traduzca al lenguaje matemático: Existe un número entero, tal que, su suma con 5 es igual a 0.\\
\begin{enumerate}
    \item a) $\exists x \in \mathbb{Z} $ tq $\ x + 5 = 0$
    \item b) $\forall x \in \mathbb{Z},\ x + 5 = 0$
    \item c) $\exists x \in \mathbb{N} $ tq $\ x + 5 = 0$
    \item d) $\forall x \in \mathbb{R},\ x + 5 = 0$
\end{enumerate}
}
\end{question}

\begin{question}{458}{cuantificadores}{1}{b}{7}{
Determine la negación lógica de: $\forall x \in \mathbb{R},\ x^2 \geq 0$.\\
\begin{enumerate}
    \item a) $\forall x \in \mathbb{R},\ x^2 < 0$
    \item b) $\exists x \in \mathbb{R}  $ tq $ \ x^2 < 0$
    \item c) $\exists x \in \mathbb{R} $ tq $ x^2 > 0$
    \item d) $\forall x \in \mathbb{R},\ x^2 > 0$
\end{enumerate}
}
\end{question}

\begin{question}{459}{cuantificadores}{1}{b}{7}{
Sea P el conjunto de Planetas y A el conjunto de planetas con atmosfera. El enunciado: No todos los planetas tienen atmósfera, se expresa lógicamente como:\\
\begin{enumerate}
    \item a) $\forall p \in P,\ p \in A$ 
    \item b) $\exists p \in P,\ p \notin A$ 
    \item c) $\exists p \in P,\ p \in A$
    
\end{enumerate}
}
\end{question}

\begin{question}{460}{cuantificadores}{1}{a}{7}{
Traduzca al español: $\exists x \in \mathbb{N},\ x^2 = 25$.\\
\begin{enumerate}
    \item a) Existe un número natural cuyo cuadrado es 25
    \item b) Todo número natural al cuadrado es 25
    \item c) Existe un número real cuyo cuadrado es 25
    \item d) Ninguna de las anteriores
\end{enumerate}
}
\end{question}
% ---- ARGUMENTOS VÁLIDOS E INVÁLIDOS ----

\begin{question}{461}{argumentos}{1}{a}{6}{
Un número es par si y solo si es divisible por 2. El número 14 es divisible por 2. Por lo tanto, el número 14 es par. Este argumento es:
\begin{enumerate}
    \item a) Válido
    \item b) Inválido 
    
\end{enumerate}
}
\end{question}

\begin{question}{462}{argumentos}{1}{a}{6}{
Solo los peces viven en el agua. Las ballenas viven en el agua. Por lo tanto, las ballenas son peces. Este razonamiento es:
\begin{enumerate}
    \item a) Válido
    \item b) Inválido
\end{enumerate}
}
\end{question}

\begin{question}{463}{argumentos}{1}{b}{6}{
Si una figura es cuadrado, entonces tiene cuatro lados iguales. La figura tiene cuatro lados. Por lo tanto, es un cuadrado. Este argumento es:
\begin{enumerate}
    \item a) Válido
    \item b) Inválido 
\end{enumerate}
}
\end{question}

\begin{question}{464}{argumentos}{1}{a}{6}{
Si un animal es mamífero, entonces es vertebrado. El perro es mamífero. Por lo tanto, el perro es vertebrado. Este argumento es:
\begin{enumerate}
    \item a) Válido 
    \item b) Inválido
    
\end{enumerate}
}
\end{question}

\begin{question}{465}{argumentos}{1}{b}{6}{
Si estudio, aprobaré el curso. Estudié, por lo tanto no aprobé el curso. Este razonamiento es:
\begin{enumerate}
    \item a) Válido 
    \item b) Inválido 
\end{enumerate}
}
\end{question}

% ---- CUANTIFICADORES ----

\begin{question}{466}{cuantificadores}{1}{a}{7}{
Traduzca: Para todo número real $x$, $x^2 \geq 0$.\\
\begin{enumerate}
    \item a) $\forall x \in \mathbb{R},\ x^2 \geq 0$
    \item b) $\exists x \in \mathbb{R}$ tq $\ x^2 \geq 0$
    \item c) $\forall x \in \mathbb{R},\ x^2 > 0$
    \item d) $\exists x \in \mathbb{R}$ tq $\ x^2 < 0$
\end{enumerate}
}
\end{question}

\begin{question}{467}{cuantificadores}{1}{b}{7}{
Escriba la negación: Todos los estudiantes aprobaron el examen.\\
\begin{enumerate}
    \item a) Ningún estudiante aprobó el examen
    \item b) Algún estudiante no aprobó el examen
    \item c) Existe un estudiante que aprobó el examen
    \item d) Ninguna de las anteriores
\end{enumerate}
}
\end{question}

\begin{question}{468}{cuantificadores}{3}{a}{7}{
Traduzca: Existe un número natural par mayor que 100.\\
\begin{enumerate}
    \item a) $\exists n \in \mathbb{N} $ tq $\ n \mod_2=0 \land n > 100$
    \item b) $\forall n \in \mathbb{N},\ n  \mod_2=0 \land n > 100$
    \item c) $\exists n \in \mathbb{Z}$ tq $\ n \mod_2=0 \lor n > 100$
    \item d) $\forall n \in \mathbb{Z},\ n  \mod_2=0 \lor n > 100$
\end{enumerate}
}
\end{question}

\begin{question}{469}{cuantificadores}{1}{a}{7}{
¿Cuál de las siguientes proposiciones es la negación correcta de $\exists x \in \mathbb{R} $ tq $\ x > 5$?\\
\begin{enumerate}
    \item a) $\forall x \in \mathbb{R},\ x \leq 5$
    \item b) $\exists x \in \mathbb{R}$ tq $\ x \leq 5$
    \item c) $\forall x \in \mathbb{R},\ x > 5$
    \item d) $\exists x \in \mathbb{R}$ tq $\ x \geq 5$
\end{enumerate}
}
\end{question}

\begin{question}{470}{cuantificadores}{1}{a}{7}{
Traduzca al español: $\forall x \in \mathbb{Z},\ x + 0 = x$.\\
\begin{enumerate}
    \item a) Todo número entero sumado con 0 es igual a sí mismo
    \item b) Existe un número entero que sumado con 0 es igual a sí mismo
    \item c) Ningún número entero sumado con 0 es igual a sí mismo
    \item d) Algún número entero sumado con 0 es igual a sí mismo
\end{enumerate}
}
\end{question}


\begin{question}{471}{proposiciones}{2}{a}{2}{
¿Cuál tabla de verdad corresponde a $p \rightarrow q$?
\begin{enumerate}
    \item a) 
    \begin{tabular}{|c|c|c|}
    \hline
    $p$ & $q$ & $p \rightarrow q$\\
    \hline
    V & V & V\\
    V & F & F\\
    F & V & V\\
    F & F & V\\
    \hline
    \end{tabular}
    \item b) 
    \begin{tabular}{|c|c|c|}
    \hline
    $p$ & $q$ & $p \rightarrow q$\\
    \hline
    V & V & F\\
    V & F & V\\
    F & V & V\\
    F & F & F\\
    \hline
    \end{tabular}
    \item c) 
    \begin{tabular}{|c|c|c|}
    \hline
    $p$ & $q$ & $p \rightarrow q$\\
    \hline
    V & V & V\\
    V & F & V\\
    F & V & F\\
    F & F & F\\
    \hline
    \end{tabular}
    \item d) 
    \begin{tabular}{|c|c|c|}
    \hline
    $p$ & $q$ & $p \rightarrow q$\\
    \hline
    V & V & V\\
    V & F & F\\
    F & V & F\\
    F & F & V\\
    \hline
    \end{tabular}
\end{enumerate}
}
\end{question}

\begin{question}{472}{proposiciones}{1}{a}{2}{
¿Cuál tabla de verdad corresponde a $p \leftrightarrow q$?
\begin{enumerate}
    \item a) 
    \begin{tabular}{|c|c|c|}
    \hline
    $p$ & $q$ & $p \leftrightarrow q$\\
    \hline
    V & V & V\\
    V & F & F\\
    F & V & F\\
    F & F & V\\
    \hline
    \end{tabular}
    \item b) 
    \begin{tabular}{|c|c|c|}
    \hline
    $p$ & $q$ & $p \leftrightarrow q$\\
    \hline
    V & V & F\\
    V & F & V\\
    F & V & V\\
    F & F & F\\
    \hline
    \end{tabular}
    \item c) 
    \begin{tabular}{|c|c|c|}
    \hline
    $p$ & $q$ & $p \leftrightarrow q$\\
    \hline
    V & V & V\\
    V & F & V\\
    F & V & V\\
    F & F & V\\
    \hline
    \end{tabular}
    \item d) 
    \begin{tabular}{|c|c|c|}
    \hline
    $p$ & $q$ & $p \leftrightarrow q$\\
    \hline
    V & V & F\\
    V & F & F\\
    F & V & V\\
    F & F & F\\
    \hline
    \end{tabular}
\end{enumerate}
}
\end{question}
\begin{question}{473}{conjuntos}{1}{a}{2}{
Sean $A$ y $B$ subconjuntos de un universo $U$. ¿Cuál enunciado corresponde a la proposición $A \cap B = \emptyset$?
\begin{enumerate}
    \item a) $A$ y $B$ no tienen elementos en común.
    \item b) Todo elemento de $A$ también está en $B$.
    \item c) $A$ y $B$ son idénticos.
    \item d) Ningún elemento de $U$ pertenece a $A$ ni a $B$.
\end{enumerate}
}
\end{question}

\begin{question}{474}{conjuntos}{1}{b}{2}{
Sea $U$ un universo y $A \subseteq U$. ¿Cuál es la traducción correcta de $A^c$?
\begin{enumerate}
    \item a) Todos los elementos que pertenecen tanto a $A$ como a $U$.
    \item b) Todos los elementos de $U$ que no están en $A$.
    \item c) Todos los subconjuntos de $A$.
    \item d) Ningún elemento pertenece a $U$.
\end{enumerate}
}
\end{question}

\begin{question}{475}{conjuntos}{1}{c}{2}{
Sea $E =$ {el conjunto de estudiantes}, $P =$ {el conjunto de profesores}, $A =$ {el conjunto de administrativos}. ¿Qué persona pertenecería a este conjunto $(E \cup P) \cap A$?
\begin{enumerate}
    \item a) Todos los administrativos que no son estudiantes ni profesores.
    \item b) Todos los estudiantes y profesores, pero no administrativos.
    \item c) Las personas que son administrativos y además son estudiantes o profesores.
    \item d) Nadie pertenece al conjunto.
\end{enumerate}
}
\end{question}

\begin{question}{476}{conjuntos}{1}{d}{2}{
Si en Bogota hay solo 2 cines que son $C_1 =$ {películas en Cine Colombia}, $C_2 =$ {películas en Cine-Mark}. ¿Qué representa el conjunto $(C_1 \setminus C_2) \cup (C_2 \setminus C_1)$?
\begin{enumerate}
    \item a) Las películas que están en ambos cines.
    \item b) Las películas que no están en ningún cine.
    \item c) Las películas que están exactamente en un cine, pero no en los dos.
    \item d) Todas las películas que se presentan en Bogotá.
\end{enumerate}
}
\end{question}
\begin{question}{477}{conjuntos}{1}{a}{2}{
Sean $A, B \subseteq U$. ¿Qué significa la proposición $A \subseteq B$?
\begin{enumerate}
    \item a) Todo elemento de $A$ está en $B$.
    \item b) Algún elemento de $A$ no está en $B$.
    \item c) $A$ y $B$ no tienen elementos en común.
    \item d) Ningún elemento de $B$ pertenece a $A$.
\end{enumerate}
}
\end{question}

\begin{question}{478}{conjuntos}{1}{b}{2}{
Sea $U =$ universo de profesionales, $M =$ {el conjunto de profesionales matemáticos}, $E =$ {el conjunto de profesionales economistas}. ¿Qué representa la proposición $M \cap E \neq \emptyset$?
\begin{enumerate}
    \item a) No existe ningún matemático que también sea economista.
    \item b) Al menos una persona es tanto matemático como economista.
    \item c) Todos los matemáticos son economistas.
    \item d) Ningún economista es matemático.
\end{enumerate}
}
\end{question}

\begin{question}{479}{conjuntos}{1}{c}{2}{
Sea el universo $U =$ {todas las asignaturas de la universidad}, y los conjuntos $M =$ {asignaturas de matemáticas}, $F =$ {asignaturas de filosofía}. ¿Qué significa $M \cup F$?
\begin{enumerate}
    \item a) Todas las asignaturas que son de matemáticas y de filosofía a la vez.
    \item b) Las asignaturas que son únicamente de matemáticas.
    \item c) Todas las asignaturas que son de matemáticas o de filosofía.
    \item d) Ninguna asignatura pertenece al conjunto.
\end{enumerate}
}
\end{question}

\begin{question}{480}{conjuntos}{1}{b}{2}{
Sean $A, B \subseteq U$. ¿Qué significa $(A \cup B)^c$?
\begin{enumerate}
    \item a) Los elementos que pertenecen a $A$ o a $B$.
    \item b) Los elementos que no están en $A$ ni en $B$.
    \item c) Los elementos que están en $A$ y en $B$ al mismo tiempo.
    \item d) Los elementos que no están en el universo $U$.
\end{enumerate}
}
\end{question}

\begin{question}{481}{conjuntos}{1}{a}{3}{
Sea el universo $U =$ {todos los equipos}, y los conjuntos $A =$ {equipos que ganaron su división}, $T =$ {equipos en el top 6}. ¿Qué cumple un elemento que este en el siguiente conjunto $(T \setminus A) \cup A$?
\begin{enumerate}
    \item a) es un equipo que esta en el top 6 pero no ganaron su división, o bien los que sí la ganaron.
    \item b) es un equipo que ganó su división y además están en el top 6.
    \item c) es un equipo que no ganó su división ni están en el top 6.
    \item d) Ningún equipo del universo.
\end{enumerate}
}
\end{question}
\begin{question}{482}{conjuntos}{1}{b}{2}{
Sean $A, B \subseteq U$. ¿Qué cumple un elemento que este en  $A \cap B^c$?
\begin{enumerate}
    \item a) Es un elemento que está en $A$ o en $B$.
    \item b) Es un elemento que está en $A$ pero no en $B$.
    \item c) Es un elemento que no está en $A$ ni en $B$.
    \item d) Es un elemento que pertenece tanto a $A$ como a $B$.
\end{enumerate}
}
\end{question}

\begin{question}{483}{conjuntos}{2}{c}{2}{
Sea el conjunto universal $U =$ Conjunto de todos los estudiantes de los Andes, $A =$ {el conjunto de alumnos de matemáticas de los andes}, $B =$ {el conjunto de alumnos de economía de los andes}. ¿Qué alumonos estarían en el siguiente conjunto $A \triangle B = (A \setminus B) \cup (B \setminus A)$?
\begin{enumerate}
    \item a) Los alumnos que estudian tanto matemáticas como economía.
    \item b) Todos los alumnos de matemáticas.
    \item c) Los alumnos que estudian exactamente una de las dos carreras, pero no ambas.
    \item d) Ningún alumno pertenece al conjunto.
\end{enumerate}
}
\end{question}

\begin{question}{484}{conjuntos}{1}{a}{2}{
Sean $A, B \subseteq U$. ¿Qué conjunto es equivalente al siguiente $(A \cup B) \cap A$?
\begin{enumerate}
    \item a) El conjunto $A$.
    \item b) El conjunto $B$.
    \item c) El conjunto vacío.
    \item d) El complemento de $B$.
\end{enumerate}
}
\end{question}

\begin{question}{485}{conjuntos}{1}{d}{3}{
Sea $U =$ {todas las películas}, $C =$ {películas en Cine Colombia}, $M =$ {películas en Cine-Mark}. ¿Qué significa $(C \cup M)^c$?
\begin{enumerate}
    \item a) Las películas que están en Cine Colombia o en Cine-Mark.
    \item b) Las películas que no están en Cine Colombia.
    \item c) Las películas que no están en Cine Colombia ni en Cine-Mark.
    \item d) Todas las películas que no están en ambos cines.
\end{enumerate}
}
\end{question}

\begin{question}{486}{conjuntos}{1}{b}{2}{
Sean $A, B \subseteq U$. ¿Qué significa la proposición $A = B$?
\begin{enumerate}
    \item a) $A$ y $B$ no tienen elementos en común.
    \item b) $A$ y $B$ contienen exactamente los mismos elementos.
    \item c) $A$ contiene más elementos que $B$.
    \item d) $B$ es un subconjunto propio de $A$.
\end{enumerate}
}
\end{question}

\begin{question}{487}{conjuntos}{1}{c}{3}{
Sea  $A =$ {equipos de la AFC}, $B =$ {equipos de la NFC}, $P =$ {equipos en playoffs}. ¿Qué representa $(A \cup B) \cap P$?
\begin{enumerate}
    \item a) Los equipos en playoffs que no pertenecen a ninguna conferencia.
    \item b) Todos los equipos de la AFC o NFC, sin importar playoffs.
    \item c) Los equipos de la AFC o NFC que además están en playoffs.
    \item d) Ningún equipo del universo.
\end{enumerate}
}
\end{question}

\begin{question}{488}{conjuntos}{1}{a}{2}{
Sean $A, B \subseteq U$. ¿Qué representa la proposición $(A \cap B) \subseteq A$?
\begin{enumerate}
    \item a) Todo elemento común a $A$ y $B$ está en $A$.
    \item b) Todo elemento de $A$ está en $B$.
    \item c) Ningún elemento de $A$ está en $B$.
    \item d) $A$ y $B$ son idénticos.
\end{enumerate}
}
\end{question}

\begin{question}{489}{conjuntos}{1}{d}{2}{
Sea $U =$ {personas}, $E =$ {estudiantes}, $P =$ {profesores}. ¿Qué representa la expresión $E \cup P = U$?
\begin{enumerate}
    \item a) Nadie en el universo es estudiante ni profesor.
    \item b) Todos son estudiantes y profesores al mismo tiempo.
    \item c) Algunos son estudiantes y algunos profesores, pero no todos.
    \item d) Cada persona en el universo es estudiante o profesor (o ambos).
\end{enumerate}
}
\end{question}

\begin{question}{490}{conjuntos}{1}{b}{3}{
Sean $X =$ {equipos que ganaron su división}, $Y =$ {equipos en el top 2}. ¿Qué representa la proposición $X \cap Y$?
\begin{enumerate}
    \item a) Equipos que ganaron su división o están en el top 2.
    \item b) Equipos que ganaron su división y además están en el top 2.
    \item c) Equipos que no ganaron su división ni están en el top 2.
    \item d) Todos los equipos de la liga.
\end{enumerate}
}
\end{question}

\begin{question}{491}{conjuntos}{1}{c}{3}{
Sean $A, B, C \subseteq U$. ¿Qué representa la proposición $(A \cap B) \cup C$?
\begin{enumerate}
    \item a) Elementos que están en $C$ y también en $A$ y $B$.
    \item b) Elementos que no están en $A$ ni en $B$ pero sí en $C$.
    \item c) Elementos que pertenecen a $C$, o bien que están en $A$ y en $B$ simultáneamente.
    \item d) El conjunto vacío.
\end{enumerate}
}
\end{question}
\begin{question}{492}{conjuntos}{1}{a}{2}{
Sean $A, B \subseteq U$. ¿Qué representa la proposición $A \setminus B$?
\begin{enumerate}
    \item a) Los elementos que están en $A$ pero no en $B$.
    \item b) Los elementos que están en $B$ pero no en $A$.
    \item c) Los elementos que están en $A$ y en $B$.
    \item d) Todos los elementos que no están en $A$ ni en $B$.
\end{enumerate}
}
\end{question}

\begin{question}{493}{conjuntos}{1}{b}{2}{
Sean $A, B \subseteq U$. ¿Qué significa $A \cup (B \cap A)$?
\begin{enumerate}
    \item a) El conjunto vacío.
    \item b) El conjunto $A$.
    \item c) El conjunto $B$.
    \item d) El universo $U$.
\end{enumerate}
}
\end{question}

\begin{question}{494}{conjuntos}{1}{c}{2}{
\\Sea el conjunto universal $U =$ {Conjunto de todas las asignaturas}, y los conjuntos $M =$ {Conjunto de las asignaturas de matemáticas}, $L =$ {Conjunto de las asignaturas de lógica}. ¿Qué materias estarian en $M \cap L$?
\begin{enumerate}
    \item a) Las materias que son de matemáticas o de lógica.
    \item b) Todas las materias que no son de matemáticas ni de lógica.
    \item c) Las materias que son de matemáticas y también de lógica.
    \item d) Ninguna materia.
\end{enumerate}
}
\end{question}

\begin{question}{495}{conjuntos}{1}{d}{3}{
Sean $A, B \subseteq U$. ¿Qué significa la proposición $(A^c \cap B^c)$?
\begin{enumerate}
    \item a) Los elementos que están en $A$ o en $B$.
    \item b) Los elementos que no están en $A$ pero sí en $B$.
    \item c) Los elementos que no están en $A$ y sí están en $B$.
    \item d) Los elementos que no están ni en $A$ ni en $B$.
\end{enumerate}
}
\end{question}

\begin{question}{496}{conjuntos}{1}{a}{2}{
\\Sea el conjunto universal $U =$ {conjunto de personas de la universidad de los andes}, y los conjuntos $E =$ {conjunto de estudiantes de los andes}, $P =$ {conjunto de profesores de los andes}, $A =$ {conjunto de administrativos de los andes}. ¿Qué significa $E \cup P \cup A = U$?
\begin{enumerate}
    \item a) Todas las personas del universo son estudiantes, profesores o administrativos.
    \item b) Nadie pertenece al universo.
    \item c) Solo los estudiantes forman parte del universo.
    \item d) Ningún profesor ni administrativo pertenece al universo.
\end{enumerate}
}
\end{question}

\begin{question}{497}{conjuntos}{1}{b}{2}{
Sean $A, B \subseteq U$. ¿Qué significa la proposición $(A \cup B) \setminus A$?
\begin{enumerate}
    \item a) Los elementos que están en $A$ y en $B$.
    \item b) Los elementos que están en $B$ pero no en $A$.
    \item c) Los elementos que no están ni en $A$ ni en $B$.
    \item d) Todo el universo $U$.
\end{enumerate}
}
\end{question}

\begin{question}{498}{conjuntos}{1}{c}{3}{
Sean los conjuntos $X =$ {equipos en playoffs}, $Y =$ {equipos que ganaron su división}. ¿Qué significa $X \cap Y^c$?
\begin{enumerate}
    \item a) Equipos que no están en playoffs ni ganaron su división.
    \item b) Equipos que ganaron su división y están en playoffs.
    \item c) Equipos que están en playoffs pero no ganaron su división.
    \item d) Todos los equipos que jugaron la temporada.
\end{enumerate}
}
\end{question}

\begin{question}{499}{conjuntos}{1}{a}{2}{
Sean $A, B \subseteq U$. ¿Qué significa la proposición $(A \cap B) \cup (A \setminus B)$?
\begin{enumerate}
    \item a) El conjunto $A$.
    \item b) El conjunto $B$.
    \item c) El conjunto vacío.
    \item d) El complemento de $A$.
\end{enumerate}
}
\end{question}

\begin{question}{500}{conjuntos}{1}{d}{3}{
Sea el conjunto universal $U =$ {todos los estudiantes de los andes}, $M =$ {estudiantes de matemáticas de los andes}, $E =$ {estudiantes de economía de los andes}. ¿Qué cumpliria un elemento que este en el siguiente conjunto $(M \cup E)^c$?
\begin{enumerate}
    \item a) Estudiantes que estudian matemáticas o economía.
    \item b) Estudiantes que estudian ambas carreras.
    \item c) Todos los estudiantes del universo.
    \item d) Estudiantes que no estudian ni matemáticas ni economía.
\end{enumerate}
}
\end{question}
\begin{question}{501}{funciones}{1}{b}{9}{
Sea $f : A \to B$ una función. ¿Cuál de las siguientes condiciones es necesaria para que $f$ sea una función bien definida?
\begin{enumerate}
    \item a) Cada elemento de $B$ debe tener al menos una imagen en $A$.
    \item b) Cada elemento de $A$ debe tener exactamente una imagen en $B$.
    \item c) Cada elemento de $B$ tiene una única preimagen en $A$.
    \item d) $A$ y $B$ deben tener la misma cantidad de elementos.
\end{enumerate}
}
\end{question}

\begin{question}{502}{funciones}{1}{b}{9}{
Sea $g : \{1,2,3\} \to \{4,5,6,7\}$ definida por $g(1)=4$, $g(2)=6$. ¿Por qué $g$ no es una función bien definida?
\begin{enumerate}
    \item a) Porque no está definido el codominio.
    \item b) Porque falta la imagen de un elemento del dominio.
    \item c) Porque un elemento del dominio tiene dos imágenes.
    \item d) Porque el codominio es más grande que el dominio.
\end{enumerate}
}
\end{question}

\begin{question}{503}{funciones}{1}{c}{9}{
Sea $h : A \to \mathbb{R}$ donde $A = \{$Camila, Andrés, Tomás$\}$ que son estudiantes de pp y $h(x)$ es la altura de $x$ en centimetros. ¿Qué representa $h$?
\begin{enumerate}
    \item a) Una relación binaria.
    \item b) Una correspondencia que puede asignar varias alturas.
    \item c) Una función que asigna a cada estudiante su altura.
    \item d) Una función sin dominio definido.
\end{enumerate}
}
\end{question}

\begin{question}{504}{funciones}{1}{b}{9}{
Sea $f : \mathbb{R} \to \mathbb{R}$ dada por $f(x) = \frac{1}{x}$. ¿Por qué no está bien definida como está escrita?
\begin{enumerate}
    \item a) Porque no se especificó el codominio.
    \item b) Porque $f(0)$ no está definido.
    \item c) Porque $f(1)$ tiene más de una imagen.
    \item d) Porque su dominio no es claro.
\end{enumerate}
}
\end{question}

\begin{question}{505}{utilidad}{1}{d}{9}{
Suponga que $u : A \to \mathbb{R}$ es una función de utilidad. ¿Qué significa que $u$ represente las preferencias $P$?
\begin{enumerate}
    \item a) Que $u(a) = u(b)$ siempre.
    \item b) Que $u(a) < u(b)$ si y solo si $a P b$.
    \item c) Que $a P b$ si $u(a) > u(b)$.
    \item d) Que $a P b$ si y solo si $u(a) > u(b)$.
\end{enumerate}
}
\end{question}

\begin{question}{506}{utilidad}{1}{c}{9}{
Sea $A=\{1,2,3\}$ y $u(x)=x^2$. ¿Qué propiedad cumple $u$?
\begin{enumerate}
    \item a) Es constante.
    \item b) No es función.
    \item c) Es Inyectiva.
    \item d) Es cubica.
\end{enumerate}
}
\end{question}

\begin{question}{507}{funciones}{1}{a}{9}{
Sea $f : \{1,2,\dots,100\} \to \mathbb{N}$ con $f(n)=n+1$ si $n$ es impar y $f(n)=n-1$ si $n$ es par. ¿Qué tipo de función es?
\begin{enumerate}
    \item a) Una función inyectiva.
    \item b) Una función constante.
    \item c) Una relación no funcional.
    \item d) Una función sin codominio definido.
\end{enumerate}
}
\end{question}

\begin{question}{508}{preferencias}{1}{b}{9}{
Suponga que una persona es indiferente entre todos los elementos de un conjunto $A$. ¿Qué debe cumplir $u:A\to \mathbb{R}$ para que sea la funcion de utilidad de sus preferencias?
\begin{enumerate}
    \item a) $u(x)$ crece estrictamente con $x$.
    \item b) $u(x)$ es constante para todo $x\in A$.
    \item c) $u(x)$ tiene un único máximo.
    \item d) $u(x)$ depende de la posición de $x$ en $A$.
\end{enumerate}
}
\end{question}

\begin{question}{509}{funciones}{1}{a}{9}{
Sea $f : P \to [0,5]$ donde $P$ es el conjunto de estudiantes de pp y $f(x)$ es la nota de $x$. ¿Qué representa este caso?
\begin{enumerate}
    \item a) Una función que asigna a cada estudiante su nota.
    \item b) Una función sin codominio.
    \item c) Una relación sin definición.
    \item d) Una función que no es bien definida.
\end{enumerate}
}
\end{question}

\begin{question}{510}{utilidad}{1}{d}{9}{
Sea la funcion de utilidad sobre las preferencias de un individuo $u : A \to \mathbb{R}$ con un máximo $m \in A$. ¿Qué condición debe cumplir?
\begin{enumerate}
    \item a) $u(m) = u(n)$ para todo $n \in A$.
    \item b) $\exists m \in A$ tal que $u(m) < u(n)$.
    \item c) $\forall n \in A, u(m) \geq u(n)$.
    \item d) $\forall n \in A \setminus \{m\}, u(m) > u(n)$.
\end{enumerate}
}
\end{question}
\begin{question}{511}{utilidad}{1}{c}{9}{
Sea $u : A \to \mathbb{R}$ una función de utilidad. Si $u(x) < u(y)$ si y solo si $y P x$, ¿qué significa esto?
\begin{enumerate}
    \item a) Que $u$ no representa ninguna preferencia.
    \item b) Que $u$ representa preferencias indiferentes.
    \item c) Que $u$ representa las preferencias $P$.
    \item d) Que $u$ es decreciente.
\end{enumerate}
}
\end{question}

\begin{question}{512}{preferencias}{1}{d}{9}{
Suponga que $u : \mathbb{R} \to \mathbb{R}$ es creciente. ¿Cuál de las siguientes afirmaciones es verdadera?
\begin{enumerate}
    \item a) $u(x)$ siempre es constante.
    \item b) Si $x < y$, entonces $u(x) = u(y)$.
    \item c) Si $x < y$, entonces $u(x) > u(y)$.
    \item d) Si $x < y$, entonces $u(x) < u(y)$.
\end{enumerate}
}
\end{question}

\begin{question}{513}{funciones}{1}{b}{9}{
Sea $f : \mathbb{R} \to \mathbb{R}$ dada por $f(x) = x^2$. ¿Cuál es una propiedad correcta de $f$?
\begin{enumerate}
    \item a) Es estrictamente creciente en todo $\mathbb{R}$.
    \item b) Tiene un mínimo en $x=0$.
    \item c) No es función.
    \item d) Tiene un máximo en $x=0$.
\end{enumerate}
}
\end{question}

\begin{question}{514}{utilidad}{1}{a}{9}{
Sea $u : A \to \mathbb{R}$ con $A = \{atletismo, pintura, teatro\}$ y $u(x)$ da el número de letras de $x$. ¿Qué rcumple la función?
\begin{enumerate}
    \item a) Es inyectiva.
    \item b) No esta bien definida.
    \item c) Es decreciente.
    \item d) Es una función sin codominio.
\end{enumerate}
}
\end{question}

\begin{question}{515}{preferencias}{1}{d}{9}{
Sea la función de utilidad $u : A \to \mathbb{R}$ con un máximo $m$. ¿Qué significa en términos de preferencias?
\begin{enumerate}
    \item a) Que $m$ es indiferente a todos los demás.
    \item b) Que $m$ no es comparable.
    \item c) Que $m$ es menor que todos los demás.
    \item d) Que $m$ es preferido a todo $n \neq m$.
\end{enumerate}
}
\end{question}

\begin{question}{516}{funciones}{1}{c}{9}{
Sea $f : P \to \{a,b,c,\dots,z\}$ con $f(x) =$ año de nacimiento de $x$, donde P es el conjunto de estudiantes de PP. ¿Qué tipo de función es?
\begin{enumerate}
    \item a) Una función constante.
    \item b) Una función sin dominio.
    \item c) Una función mal definida.
    \item d) Una función lineal.
\end{enumerate}
}
\end{question}

\begin{question}{517}{utilidad}{1}{a}{9}{
Suponga que la función de utilidad sobre las preferencias de algun individuo $u : A \to \mathbb{R}$ cumple que $\forall x,y \in A$, $u(x) = u(y)$. ¿Qué representan estas preferencias?
\begin{enumerate}
    \item a) Indiferencia total.
    \item b) Preferencia estricta por un único elemento.
    \item c) Preferencias crecientes.
    \item d) Preferencias decrecientes.
\end{enumerate}
}
\end{question}

\begin{question}{518}{funciones}{1}{b}{9}{
Sea $f : \mathbb{N} \to \mathbb{N}$ definida por $f(n) = n$. ¿Qué propiedad cumple?
\begin{enumerate}
    \item a) No está bien definida.
    \item b) Es biyectiva.
    \item c) Tiene un máximo.
    \item d) Es constante.
\end{enumerate}
}
\end{question}

\begin{question}{519}{preferencias}{1}{c}{9}{
Si $u : A \to \mathbb{R}$ tiene dos máximos $m_1, m_2 \in A$, ¿qué significa?
\begin{enumerate}
    \item a) Que no es una función.
    \item b) Que no existe preferencia representada.
    \item c) Que hay dos elementos igualmente preferidos.
    \item d) Que $u$ es decreciente.
\end{enumerate}
}
\end{question}

\begin{question}{520}{utilidad}{1}{a}{3}{
Sea $u(x)= -x^2$ con dominio $A = \mathbb{R}$. ¿Qué propiedad cumple $u$?
\begin{enumerate}
    \item a) Tiene un máximo en $x=0$.
    \item b) Es creciente en todo el dominio.
    \item c) No tiene extremos.
    \item d) Es constante.
\end{enumerate}
}
\end{question}
\begin{question}{521}{funciones}{1}{b}{9}{
Sea $f : A \to B$. ¿Qué condición no debe cumplirse para que $f$ sea una función?
\begin{enumerate}
    \item a) Cada elemento de $B$ tiene al menos una preimagen en $A$.
    \item b) Cada elemento de $A$ tiene más de una imagen en $B$.
    \item c) Cada elemento de $B$ tiene más de una preimagen en $A$.
    \item d) $A$ y $B$ deben ser conjuntos iguales.
\end{enumerate}
}
\end{question}

\begin{question}{522}{funciones}{1}{a}{9}{
Sea $f : \{1,2,3,4\} \to \{a,b,c\}$ definida por $f(1)=a, f(2)=b, f(3)=c, f(4)=1$. ¿Qué representa este caso?
\begin{enumerate}
    \item a) Una función sobreyectiva.
    \item b) Una función constante.
    \item c) Una relación no funcional.
    \item d) Una función no definida.
\end{enumerate}
}
\end{question}

\begin{question}{523}{funciones}{1}{c}{9}{
Sea $f : \mathbb{R} \to \mathbb{R}$ con $f(x) = x+3$. ¿Cuál es la imagen de $0$?
\begin{enumerate}
    \item a) $-3$
    \item b) $0$
    \item c) $3$
    \item d) $x$
\end{enumerate}
}
\end{question}

\begin{question}{524}{funciones}{1}{d}{9}{
Sea $f : \mathbb{R} \to \mathbb{R}$ con $f(x) = x^2$. ¿Cuál es el codominio de esta función?
\begin{enumerate}
    \item a) $\mathbb{Z}$
    \item b) $\mathbb{N}$
    \item c) $\mathbb{R}$
    \item d) $\mathbb{R}_{\geq 0}$
\end{enumerate}
}
\end{question}

\begin{question}{525}{funciones}{1}{b}{9}{
Sea $f : \mathbb{Z} \to \mathbb{Z}$ definida por $f(n) = n+1$. ¿Cuál es su imagen?
\begin{enumerate}
    \item a) $\mathbb{N}$
    \item b) $\mathbb{Z}$
    \item c) $\mathbb{R}$
    \item d) $\mathbb{Z}_{\geq 0}$
\end{enumerate}
}
\end{question}

\begin{question}{526}{funciones}{1}{c}{9}{
Sea $g : \{1,2,3\} \to \{a,b\}$ con $g(1)=a, g(2)=a, g(3)=b$. ¿Qué tipo de función es?
\begin{enumerate}
    \item a) Biyectiva.
    \item b) Inyectiva.
    \item c) Sobreyectiva.
    \item d) No es función.
\end{enumerate}
}
\end{question}

\begin{question}{527}{funciones}{1}{a}{9}{
Sea $h : \mathbb{R} \to \mathbb{R}$ con $h(x) = \sin(x)$. ¿Cuál es el rango de $h$?
\begin{enumerate}
    \item a) $[-1,1]$
    \item b) $(0, \infty)$
    \item c) $\mathbb{R}$
    \item d) $[0,1]$
\end{enumerate}
}
\end{question}

\begin{question}{528}{funciones}{1}{d}{9}{
Sea $f : \mathbb{R} \to \mathbb{R}$ con $f(x) = 1/x$. ¿Qué dominio hace que $f$ esté bien definida?
\begin{enumerate}
    \item a) $\mathbb{R}$
    \item b) $\mathbb{R}_{>0}$
    \item c) $\mathbb{Z}$
    \item d) $\mathbb{R} \setminus \{0\}$
\end{enumerate}
}
\end{question}

\begin{question}{529}{funciones}{1}{c}{9}{
Sea $f : \{a,b,c\} \to \{1,2,3,4\}$ con $f(a)=1$, $f(b)=2$, $f(c)=2$. ¿Qué propiedad cumple $f$?
\begin{enumerate}
    \item a) Es inyectiva.
    \item b) Es biyectiva.
    \item c) Es función función.
    \item d) No es función.
\end{enumerate}
}
\end{question}

\begin{question}{530}{funciones}{1}{b}{9}{
Sea $f : \mathbb{N} \to \mathbb{N}$ con $f(n) = n^2$. ¿Cuál de las siguientes afirmaciones es correcta?
\begin{enumerate}
    \item a) $f$ es sobreyectiva.
    \item b) $f$ no es sobreyectiva.
    \item c) $f$ es biyectiva.
    \item d) $f$ no es una función.
\end{enumerate}
}
\end{question}
\begin{question}{531}{funciones}{1}{a}{9}{
Sea $f : \{1,2,3\} \to \{a,b,c\}$ definida por $f(1)=a, f(2)=b, f(3)=c$. ¿Qué propiedad cumple $f$?
\begin{enumerate}
    \item a) Es biyectiva.
    \item b) Es constante.
    \item c) No es inyectiva.
    \item d) No es sobreyectiva.
\end{enumerate}
}
\end{question}

\begin{question}{532}{funciones}{1}{c}{9}{
Sea $f : \{1,2,3\} \to \{a,b\}$ definida por $f(1)=a, f(2)=a, f(3)=b$. ¿Qué propiedad cumple $f$?
\begin{enumerate}
    \item a) Es inyectiva.
    \item b) Es biyectiva.
    \item c) Es sobreyectiva.
    \item d) No es función.
\end{enumerate}
}
\end{question}

\begin{question}{533}{funciones}{1}{b}{9}{
Sea $f : \{a,b,c\} \to \{1,2,3,4\}$ definida por $f(a)=1, f(b)=2, f(c)=3$. ¿Qué propiedad cumple $f$?
\begin{enumerate}
    \item a) Es sobreyectiva.
    \item b) Es inyectiva pero no sobreyectiva.
    \item c) Es biyectiva.
    \item d) No es función.
\end{enumerate}
}
\end{question}

\begin{question}{534}{funciones}{1}{c}{9}{
Sea $f : \mathbb{Z} \to \mathbb{Z}$ con $f(n)=n+1$. ¿Qué propiedad cumple $f$?
\begin{enumerate}
    \item a) Es sobreyectiva pero no inyectiva.
    \item b) No es sobreyectiva.
    \item c) Es biyectiva.
    \item d) No es función.
\end{enumerate}
}
\end{question}

\begin{question}{535}{funciones}{1}{a}{9}{
Sea $f : \mathbb{N} \to \mathbb{N}$ con $f(n)=2n$. ¿Qué propiedad cumple $f$?
\begin{enumerate}
    \item a) Es inyectiva.
    \item b) Es biyectiva.
    \item c) No es función.
    \item d) Es constante.
\end{enumerate}
}
\end{question}

\begin{question}{536}{funciones}{1}{b}{9}{
Sea $f : \mathbb{R} \to \mathbb{R}$ con $f(x)=x^3$. ¿Qué propiedad cumple $f$?
\begin{enumerate}
    \item a) Es sobreyectiva pero no inyectiva.
    \item b) Es biyectiva.
    \item c) Es inyectiva pero no sobreyectiva.
    \item d) Es constante.
\end{enumerate}
}
\end{question}

\begin{question}{537}{funciones}{1}{c}{9}{
Sea $f : \mathbb{R} \to \mathbb{R}_{\geq0}$ con $f(x)=x^2$. ¿Qué propiedad cumple $f$?
\begin{enumerate}
    \item a) No es biyectiva.
    \item b) Es inyectiva.
    \item c) Es sobreyectiva.
    \item d) Es biyectiva.
\end{enumerate}
}
\end{question}

\begin{question}{538}{funciones}{1}{d}{9}{
Sea $f : \{1,2,3,4\} \to \{a,b\}$ definida por $f(1)=a, f(2)=b, f(3)=a, f(4)=b$. ¿Qué propiedad cumple $f$?
\begin{enumerate}
    \item a) Es inyectiva.
    \item b) Es biyectiva.
    \item c) No es función.
    \item d) Es sobreyectiva pero no inyectiva.
\end{enumerate}
}
\end{question}

\begin{question}{539}{funciones}{1}{c}{9}{
Sea $f : \{a,b,c,d\} \to \{1,2,3,4\}$ con $f(a)=1, f(b)=2, f(c)=3, f(d)=4$. ¿Qué propiedad cumple $f$?
\begin{enumerate}
    \item a) Es sobreyectiva pero no inyectiva.
    \item b) Es inyectiva pero no sobreyectiva.
    \item c) Es biyectiva.
    \item d) No es función.
\end{enumerate}
}
\end{question}

\begin{question}{540}{funciones}{1}{a}{9}{
Sea $f : \mathbb{R} \to \mathbb{R}$ con $f(x)=e^x$. ¿Qué propiedad cumple $f$?
\begin{enumerate}
    \item a) Es inyectiva pero no sobreyectiva.
    \item b) Es biyectiva.
    \item c) Es sobreyectiva pero no inyectiva.
    \item d) Es constante.
\end{enumerate}
}
\end{question}
\begin{question}{541}{funciones}{1}{a}{9}{
Sea $f : P \to \mathbb{R}$ donde $P$ es el conjunto de estudiantes y $f(x)$ es la altura de $x$. ¿Qué propiedad cumple esta función?
\begin{enumerate}
    \item a) Es inyectiva si todos los estudiantes tienen estaturas distintas.
    \item b) Es sobreyectiva sobre $\mathbb{R}$.
    \item c) No es función porque las alturas se repiten.
    \item d) Es constante.
\end{enumerate}
}
\end{question}
\begin{question}{542}{funciones}{1}{a}{9}{
Una función es idempotente si $f(f(x)) = f(x)$ para todo $x$. La función $f: \mathbb{R} \to \mathbb{R}$ definida por $f(x) = |x|$ es idempotente.
\begin{enumerate}
\item a) Verdadero
\item b) Falso
\end{enumerate}
}
\end{question}

\begin{question}{543}{funciones}{1}{a}{9}{
Una función es idempotente si $f(f(x)) = f(x)$ para todo $x$. La función $f: \mathbb{R} \to \mathbb{R}$ definida por $f(x) = x$ es idempotente.
\begin{enumerate}
\item a) Verdadero
\item b) Falso
\end{enumerate}
}
\end{question}

\begin{question}{544}{funciones}{1}{b}{9}{
Una función es idempotente si $f(f(x)) = f(x)$ para todo $x$. La función $f: \mathbb{R} \to \mathbb{R}$ definida por $f(x) = x + 1$ es idempotente.
\begin{enumerate}
\item a) Verdadero
\item b) Falso
\end{enumerate}
}
\end{question}

\begin{question}{545}{funciones}{2}{a}{9}{
Una función es idempotente si $f(f(x)) = f(x)$ para todo $x$. La función $f: \mathbb{R} \to \mathbb{R}$ definida por $f(x) = \max(0, x)$ es idempotente.
\begin{enumerate}
\item a) Verdadero
\item b) Falso
\end{enumerate}
}
\end{question}

\begin{question}{546}{funciones}{1}{b}{9}{
Una función es idempotente si $f(f(x)) = f(x)$ para todo $x$. La función $f: \mathbb{R} \to \mathbb{R}$ definida por $f(x) = x^2$ es idempotente.
\begin{enumerate}
\item a) Verdadero
\item b) Falso
\end{enumerate}
}
\end{question}

\begin{question}{547}{funciones}{2}{a}{9}{
Una función es idempotente si $f(f(x)) = f(x)$ para todo $x$. La función $f: \mathbb{R} \to \mathbb{R}$ definida por $f(x) = \lfloor x \rfloor$ (función piso) es idempotente.
\begin{enumerate}
\item a) Verdadero
\item b) Falso
\end{enumerate}
}
\end{question}

\begin{question}{548}{funciones}{1}{b}{9}{
Una función es idempotente si $f(f(x)) = f(x)$ para todo $x$. La función $f: \mathbb{R} \to \mathbb{R}$ definida por $f(x) = 2x$ es idempotente.
\begin{enumerate}
\item a) Verdadero
\item b) Falso
\end{enumerate}
}
\end{question}

\begin{question}{549}{funciones}{1}{a}{9}{
Una función es idempotente si $f(f(x)) = f(x)$ para todo $x$. La función $f: \mathbb{R} \to \mathbb{R}$ definida por $f(x) = \min(1, x)$ es idempotente.
\begin{enumerate}
\item a) Verdadero
\item b) Falso
\end{enumerate}
}
\end{question}

\begin{question}{550}{funciones}{1}{b}{9}{
Una función es idempotente si $f(f(x)) = f(x)$ para todo $x$. La función $f: \mathbb{R} \to \mathbb{R}$ definida por $f(x) = \sin(x)$ es idempotente.
\begin{enumerate}
\item a) Verdadero
\item b) Falso
\end{enumerate}
}
\end{question}

\begin{question}{551}{funciones}{1}{a}{9}{
Una función es idempotente si $f(f(x)) = f(x)$ para todo $x$. La función constante $f: \mathbb{R} \to \mathbb{R}$ definida por $f(x) = c$ (para cualquier constante $c$) es idempotente.
\begin{enumerate}
\item a) Verdadero
\item b) Falso
\end{enumerate}
}
\end{question}
\begin{question}{552}{funciones}{1}{a}{9}{
Una función es par si $f(-x) = f(x)$ para todo $x$ en su dominio. La función $f: \mathbb{R} \to \mathbb{R}$ definida por $f(x) = x^2$ es par.
\begin{enumerate}
\item a) Verdadero
\item b) Falso
\end{enumerate}
}
\end{question}

\begin{question}{553}{funciones}{1}{b}{9}{
Una función es impar si $f(-x) = -f(x)$ para todo $x$ en su dominio. La función $f: \mathbb{R} \to \mathbb{R}$ definida por $f(x) = x^2 + 1$ es impar.
\begin{enumerate}
\item a) Verdadero
\item b) Falso
\end{enumerate}
}
\end{question}

\begin{question}{554}{funciones}{1}{a}{9}{
Una función es periódica si existe $T > 0$ tal que $f(x+T) = f(x)$ para todo $x$. La función $f: \mathbb{R} \to \mathbb{R}$ definida por $f(x) = \sin(x)$ es periódica.
\begin{enumerate}
\item a) Verdadero
\item b) Falso
\end{enumerate}
}
\end{question}

\begin{question}{555}{funciones}{2}{b}{9}{
Una función es contractiva si existe $0 < k < 1$ tal que $|f(x)-f(y)| \leq k|x-y|$ para todo $x,y$. La función $f: \mathbb{R} \to \mathbb{R}$ definida por $f(x) = 2x$ es contractiva.
\begin{enumerate}
\item a) Verdadero
\item b) Falso
\end{enumerate}
}
\end{question}

\begin{question}{556}{funciones}{3}{a}{9}{
Una función es convexa si $f(tx+(1-t)y) \leq tf(x)+(1-t)f(y)$ para todo $x,y$ y $t \in [0,1]$. La función $f: \mathbb{R} \to \mathbb{R}$ definida por $f(x) = x^2$ es convexa.
\begin{enumerate}
\item a) Verdadero
\item b) Falso
\end{enumerate}
}
\end{question}

\begin{question}{557}{funciones}{3}{b}{9}{
Una función es Lipschitz si existe $L > 0$ tal que $|f(x)-f(y)| \leq L|x-y|$ para todo $x,y$. La función $f: \mathbb{R} \to \mathbb{R}$ definida por $f(x) = x^3$ es Lipschitz en todo $\mathbb{R}$.
\begin{enumerate}
\item a) Verdadero
\item b) Falso
\end{enumerate}
}
\end{question}

\begin{question}{558}{funciones}{1}{a}{9}{
Una función es acotada si existe $M > 0$ tal que $|f(x)| \leq M$ para todo $x$. La función $f: \mathbb{R} \to \mathbb{R}$ definida por $f(x) = \arctan(x)$ es acotada.
\begin{enumerate}
\item a) Verdadero
\item b) Falso
\end{enumerate}
}
\end{question}

\begin{question}{559}{funciones}{1}{b}{9}{
Una función es monótona creciente si $x \leq y$ implica $f(x) \leq f(y)$. La función $f: \mathbb{R} \to \mathbb{R}$ definida por $f(x) = -x$ es monótona creciente.
\begin{enumerate}
\item a) Verdadero
\item b) Falso
\end{enumerate}
}
\end{question}

\begin{question}{560}{funciones}{3}{a}{9}{
Una función es simétrica si $f(x,y) = f(y,x)$ para todos sus argumentos. La función $f: \mathbb{R}^2 \to \mathbb{R}$ definida por $f(x,y) = x + y$ es simétrica.
\begin{enumerate}
\item a) Verdadero
\item b) Falso
\end{enumerate}
}
\end{question}

\begin{question}{561}{funciones}{1}{b}{9}{
Una función es homogénea de grado $k$ si $f(tx) = t^k f(x)$ para todo $t > 0$. La función $f: \mathbb{R} \to \mathbb{R}$ definida por $f(x) = x + 1$ es homogénea de grado 1.
\begin{enumerate}
\item a) Verdadero
\item b) Falso
\end{enumerate}
}
\end{question}
\begin{question}{562}{funciones}{1}{a}{9}{
Una función es homogénea de grado $k$ si $f(tx) = t^k f(x)$ para todo $t > 0$. La función $f(x,y) = 3x^2y + 2xy^2$ es homogénea de grado 3.
\begin{enumerate}
\item a) Verdadero
\item b) Falso
\end{enumerate}
}
\end{question}

\begin{question}{563}{funciones}{1}{b}{9}{
Una función es homogénea de grado $k$ si $f(tx) = t^k f(x)$ para todo $t > 0$. La función $f(x) = x^2  + 1$ es homogénea de grado 2.
\begin{enumerate}
\item a) Verdadero
\item b) Falso
\end{enumerate}
}
\end{question}

\begin{question}{564}{funciones}{3}{a}{9}{
Una función es homogénea de grado $k$ si $f(tx) = t^k f(x)$ para todo $t > 0$. La función $f(x,y,z) = \frac{x^3}{\sqrt{y^2 + z^2}}$ es homogénea de grado 2.
\begin{enumerate}
\item a) Verdadero
\item b) Falso
\end{enumerate}
}
\end{question}

\begin{question}{565}{funciones}{2}{b}{9}{
Una función es homogénea de grado $k$ si $f(tx) = t^k f(x)$ para todo $t > 0$. La función $f(x,y) = \sqrt{x^2 + y^2}$ es homogénea de grado 1.
\begin{enumerate}
\item a) Verdadero
\item b) Falso
\end{enumerate}
}
\end{question}

\begin{question}{566}{funciones}{2}{a}{9}{
Una función es homogénea de grado $k$ si $f(tx) = t^k f(x)$ para todo $t > 0$. La función $f(x,y) = 5x^4y^2$ es homogénea de grado 6.
\begin{enumerate}
\item a) Verdadero
\item b) Falso
\end{enumerate}
}
\end{question}

\begin{question}{567}{funciones}{2}{b}{9}{
Una función es homogénea de grado $k$ si $f(tx) = t^k f(x)$ para todo $t > 0$. La función $f(x,y) = \frac{x^2 + y^2}{x + y}$ es homogénea de grado 2.
\begin{enumerate}
\item a) Verdadero
\item b) Falso
\end{enumerate}
}
\end{question}

\begin{question}{568}{funciones}{2}{a}{9}{
Una función es homogénea de grado $k$ si $f(tx) = t^k f(x)$ para todo $t > 0$. La función $f(x,y) = \sqrt[3]{x^3 + y^3}$ es homogénea de grado 1.
\begin{enumerate}
\item a) Verdadero
\item b) Falso
\end{enumerate}
}
\end{question}

\begin{question}{569}{funciones}{2}{b}{9}{
Una función es homogénea de grado $k$ si $f(tx) = t^k f(x)$ para todo $t > 0$. La función $f(x,y) = x^2 \ln\left(\frac{y}{x}\right)$ es homogénea de grado 3.
\begin{enumerate}
\item a) Verdadero
\item b) Falso
\end{enumerate}
}
\end{question}

\begin{question}{570}{funciones}{2}{a}{9}{
Una función es homogénea de grado $k$ si $f(tx) = t^k f(x)$ para todo $t > 0$. La función $f(x,y,z) = \frac{xy + yz + zx}{x^2 + y^2 + z^2}$ es homogénea de grado 0.
\begin{enumerate}
\item a) Verdadero
\item b) Falso
\end{enumerate}
}
\end{question}

\begin{question}{571}{funciones}{2}{b}{9}{
Una función es homogénea de grado $k$ si $f(tx) = t^k f(x)$ para todo $t > 0$. La función $f(x,y) = 2^x + 3^y$ es homogénea de grado 0.
\begin{enumerate}
\item a) Verdadero
\item b) Falso
\end{enumerate}
}
\end{question}
\begin{question}{572}{funciones}{1}{a}{9}{
Una función es acotada si existe $M > 0$ tal que $|f(x)| \leq M$ para todo $x$ en su dominio. La función $f(x) = \sin(x)$ es acotada.
\begin{enumerate}
\item a) Verdadero
\item b) Falso
\end{enumerate}
}
\end{question}

\begin{question}{573}{funciones}{1}{b}{9}{
Una función es acotada si existe $M > 0$ tal que $|f(x)| \leq M$ para todo $x$ en su dominio. La función $f(x) = x^3$ es acotada en $\mathbb{R}$.
\begin{enumerate}
\item a) Verdadero
\item b) Falso
\end{enumerate}
}
\end{question}

\begin{question}{574}{funciones}{1}{a}{9}{
Una función es acotada si existe $M > 0$ tal que $|f(x)| \leq M$ para todo $x$ en su dominio. La función $f(x) = \frac{1}{1+x^2}$ es acotada en $\mathbb{R}$.
\begin{enumerate}
\item a) Verdadero
\item b) Falso
\end{enumerate}
}
\end{question}

\begin{question}{575}{funciones}{1}{b}{9}{
Una función es acotada si existe $M > 0$ tal que $|f(x)| \leq M$ para todo $x$ en su dominio. La función $f(x) = \tan(x)$ es acotada en su dominio.
\begin{enumerate}
\item a) Verdadero
\item b) Falso
\end{enumerate}
}
\end{question}

\begin{question}{576}{funciones}{3}{a}{9}{
Una función es acotada si existe $M > 0$ tal que $|f(x)| \leq M$ para todo $x$ en su dominio. La función $f(x) = e^{-x^2}$ es acotada en $\mathbb{R}$.
\begin{enumerate}
\item a) Verdadero
\item b) Falso
\end{enumerate}
}
\end{question}

\begin{question}{577}{funciones}{1}{b}{9}{
Una función es acotada si existe $M > 0$ tal que $|f(x)| \leq M$ para todo $x$ en su dominio. La función $f(x) = \ln(x)$ es acotada en $(0, \infty)$.
\begin{enumerate}
\item a) Verdadero
\item b) Falso
\end{enumerate}
}
\end{question}

\begin{question}{578}{funciones}{1}{a}{9}{
Una función es acotada si existe $M > 0$ tal que $|f(x)| \leq M$ para todo $x$ en su dominio. La función $f(x) = \frac{\sin(x)}{x}$ es acotada en $\mathbb{R}\setminus\{0\}$.
\begin{enumerate}
\item a) Verdadero
\item b) Falso
\end{enumerate}
}
\end{question}

\begin{question}{579}{funciones}{1}{b}{9}{
Una función es acotada si existe $M > 0$ tal que $|f(x)| \leq M$ para todo $x$ en su dominio. La función $f(x) = x\sin(x)$ es acotada en $\mathbb{R}$.
\begin{enumerate}
\item a) Verdadero
\item b) Falso
\end{enumerate}
}
\end{question}

\begin{question}{580}{funciones}{1}{a}{9}{
Una función es acotada si existe $M > 0$ tal que $|f(x)| \leq M$ para todo $x$ en su dominio. La función $f(x) = \arctan(x)$ es acotada en $\mathbb{R}$.
\begin{enumerate}
\item a) Verdadero
\item b) Falso
\end{enumerate}
}
\end{question}

\begin{question}{581}{funciones}{2}{b}{9}{
Una función es acotada si existe $M > 0$ tal que $|f(x)| \leq M$ para todo $x$ en su dominio. La función $f(x) = \frac{x}{x^2+1}$ es acotada en $\mathbb{R}$.
\begin{enumerate}
\item a) Verdadero
\item b) Falso
\end{enumerate}
}
\end{question}
\begin{question}{582}{funciones}{1}{a}{9}{
Una función es monótona creciente si $x \leq y$ implica $f(x) \leq f(y)$. La función $f(x) = 2x + 3$ es monótona creciente.
\begin{enumerate}
\item a) Verdadero
\item b) Falso
\end{enumerate}
}
\end{question}

\begin{question}{583}{funciones}{1}{b}{9}{
Una función es monótona creciente si $x \leq y$ implica $f(x) \leq f(y)$. La función $f(x) = -x + 1$ es monótona creciente.
\begin{enumerate}
\item a) Verdadero
\item b) Falso
\end{enumerate}
}
\end{question}

\begin{question}{584}{funciones}{1}{a}{9}{
Una función es monótona decreciente si $x \leq y$ implica $f(x) \geq f(y)$. La función $f(x) = -5x$ es monótona decreciente.
\begin{enumerate}
\item a) Verdadero
\item b) Falso
\end{enumerate}
}
\end{question}

\begin{question}{585}{funciones}{1}{b}{9}{
Una función es monótona creciente si $x \leq y$ implica $f(x) \leq f(y)$. La función $f(x) = x^2$ es monótona creciente en $\mathbb{R}$.
\begin{enumerate}
\item a) Verdadero
\item b) Falso
\end{enumerate}
}
\end{question}

\begin{question}{586}{funciones}{1}{a}{9}{
Una función es monótona creciente si $x \leq y$ implica $f(x) \leq f(y)$. La función constante $f(x) = 7$ es monótona creciente.
\begin{enumerate}
\item a) Verdadero
\item b) Falso
\end{enumerate}
}
\end{question}

\begin{question}{587}{funciones}{1}{b}{9}{
Una función es monótona decreciente si $x \leq y$ implica $f(x) \geq f(y)$. La función $f(x) = \sqrt{x}$ es monótona decreciente en $[0, \infty)$.
\begin{enumerate}
\item a) Verdadero
\item b) Falso
\end{enumerate}
}
\end{question}

\begin{question}{588}{funciones}{1}{a}{9}{
Una función es monótona decreciente si $x \leq y$ implica $f(x) \geq f(y)$. La función $f(x) = \frac{1}{x}$ es monótona decreciente en $(0, \infty)$.
\begin{enumerate}
\item a) Verdadero
\item b) Falso
\end{enumerate}
}
\end{question}

\begin{question}{589}{funciones}{1}{b}{9}{
Una función es monótona creciente si $x \leq y$ implica $f(x) \leq f(y)$. La función $f(x) = |x|$ es monótona creciente en $\mathbb{R}$.
\begin{enumerate}
\item a) Verdadero
\item b) Falso
\end{enumerate}
}
\end{question}

\begin{question}{590}{funciones}{1}{a}{9}{
Una función es monótona creciente si $x \leq y$ implica $f(x) \leq f(y)$. La función $f(x) = x^3$ es monótona creciente en $\mathbb{R}$.
\begin{enumerate}
\item a) Verdadero
\item b) Falso
\end{enumerate}
}
\end{question}

\begin{question}{591}{funciones}{1}{b}{9}{
Una función es monótona decreciente si $x \leq y$ implica $f(x) \geq f(y)$. La función $f(x) = 2^x$ es monótona decreciente en $\mathbb{R}$.
\begin{enumerate}
\item a) Verdadero
\item b) Falso
\end{enumerate}
}
\end{question}
\begin{question}{592}{funciones}{1}{a}{9}{
Una función es simétrica si $f(-x) = f(x)$ para todo $x$ en su dominio. La función $f(x) = x^4 - 2x^2 + 1$ es simétrica.
\begin{enumerate}
\item a) Verdadero
\item b) Falso
\end{enumerate}
}
\end{question}

\begin{question}{593}{funciones}{1}{b}{9}{
Una función es antisimétrica si $f(-x) = -f(x)$ para todo $x$ en su dominio. La función $f(x) = x^2 + x$ es antisimétrica.
\begin{enumerate}
\item a) Verdadero
\item b) Falso
\end{enumerate}
}
\end{question}

\begin{question}{594}{funciones}{1}{a}{9}{
Una función es periódica si existe $T > 0$ tal que $f(x+T) = f(x)$ para todo $x$. La función $f(x) = \cos(2x)$ es periódica.
\begin{enumerate}
\item a) Verdadero
\item b) Falso
\end{enumerate}
}
\end{question}

\begin{question}{595}{funciones}{2}{b}{9}{
Una función es convexa si $f(tx+(1-t)y) \leq tf(x)+(1-t)f(y)$ para $t \in [0,1]$. La función $f(x) = -x^2$ es convexa en $\mathbb{R}$.
\begin{enumerate}
\item a) Verdadero
\item b) Falso
\end{enumerate}
}
\end{question}

\begin{question}{596}{funciones}{2}{a}{9}{
Una función es cóncava si $f(tx+(1-t)y) \geq tf(x)+(1-t)f(y)$ para $t \in [0,1]$. La función $f(x) = \sqrt{x}$ es cóncava en $[0,\infty)$.
\begin{enumerate}
\item a) Verdadero
\item b) Falso
\end{enumerate}
}
\end{question}

\begin{question}{597}{funciones}{3}{a}{9}{
Una función es Lipschitz si existe $L > 0$ tal que $|f(x)-f(y)| \leq L|x-y|$ para todo $x,y$. La función $f(x) = |x|$ es Lipschitz en $\mathbb{R}$.
\begin{enumerate}
\item a) Verdadero
\item b) Falso
\end{enumerate}
}
\end{question}

\begin{question}{598}{funciones}{3}{a}{9}{
Una función es contractiva si existe $0 < k < 1$ tal que $|f(x)-f(y)| \leq k|x-y|$ para todo $x,y$. La función $f(x) = \frac{1}{2}x$ es contractiva.
\begin{enumerate}
\item a) Verdadero
\item b) Falso
\end{enumerate}
}
\end{question}

\begin{question}{599}{funciones}{1}{b}{9}{
Una función es idempotente si $f(f(x)) = f(x)$ para todo $x$. La función $f(x) = x + 1$ es idempotente.
\begin{enumerate}
\item a) Verdadero
\item b) Falso
\end{enumerate}
}
\end{question}

\begin{question}{600}{funciones}{1}{a}{9}{
Una función es involutiva si $f(f(x)) = x$ para todo $x$. La función $f(x) = -x$ es involutiva.
\begin{enumerate}
\item a) Verdadero
\item b) Falso
\end{enumerate}
}
\end{question}
\begin{question}{601}{funciones}{1}{a}{9}{
Si una regla asigna a un elemento del dominio dos imágenes distintas, sigue siendo una función.
\begin{enumerate}
\item a) Verdadero
\item b) Falso
\end{enumerate}
}
\end{question}

\begin{question}{602}{funciones}{1}{a}{9}{
Si una regla no asigna imagen a un elemento del dominio, no define una función.
\begin{enumerate}
\item a) Verdadero
\item b) Falso
\end{enumerate}
}
\end{question}

\begin{question}{603}{funciones}{1}{b}{9}{
La función $f(x)=\frac{1}{x}$ es una función de $\mathbb{R}$ en $\mathbb{R}$.
\begin{enumerate}
\item a) Verdadero
\item b) Falso
\end{enumerate}
}
\end{question}

\begin{question}{604}{funciones}{1}{a}{9}{
Si $f:A\to B$ y $g:B\to C$ funciones bien definidas, entonces $g\circ f$ siempre está bien definida.
\begin{enumerate}
\item a) Verdadero
\item b) Falso
\end{enumerate}
}
\end{question}

\begin{question}{605}{funciones}{1}{a}{9}{
La función $f:\mathbb{N}\to\mathbb{N}$ dada por $f(x)=x+1$ está bien definida.
\begin{enumerate}
\item a) Verdadero
\item b) Falso
\end{enumerate}
}
\end{question}

\begin{question}{606}{funciones}{1}{a}{9}{
Si dos funciones tienen el mismo dominio y la misma regla, son la misma función.
\begin{enumerate}
\item a) Verdadero
\item b) Falso
\end{enumerate}
}
\end{question}

\begin{question}{607}{funciones}{1}{b}{9}{
Toda función tiene una inversa.
\begin{enumerate}
\item a) Verdadero
\item b) Falso
\end{enumerate}
}
\end{question}

\begin{question}{608}{funciones}{1}{a}{9}{
Si $f:A\to B$, el dominio de $f$ es el conjunto $A$.
\begin{enumerate}
\item a) Verdadero
\item b) Falso
\end{enumerate}
}
\end{question}

\begin{question}{609}{funciones}{1}{b}{9}{
La imagen de un elemento $x$ bajo $f$ se denota $f^{-1}(x)$.
\begin{enumerate}
\item a) Verdadero
\item b) Falso
\end{enumerate}
}
\end{question}

\begin{question}{610}{funciones}{1}{b}{9}{
La regla “asignar a cada número su raíz cuadrada positiva y negativa” define una función.
\begin{enumerate}
\item a) Verdadero
\item b) Falso
\end{enumerate}
}
\end{question}

\begin{question}{611}{funciones}{1}{a}{9}{
Si $f(x)=x^2$, entonces $f(-3)=f(3)$.
\begin{enumerate}
\item a) Verdadero
\item b) Falso
\end{enumerate}
}
\end{question}

\begin{question}{612}{funciones}{1}{a}{9}{
La función identidad cumple que $f(x)=x$ para todo $x$.
\begin{enumerate}
\item a) Verdadero
\item b) Falso
\end{enumerate}
}
\end{question}

\begin{question}{613}{funciones}{1}{b}{9}{
Si $f:\mathbb{R}\to\mathbb{R}$ está dada por $f(x)=\frac{x^3}{x-1}$, entonces $f$ está bien definida.
\begin{enumerate}
\item a) Verdadero
\item b) Falso
\end{enumerate}
}
\end{question}

\begin{question}{614}{funciones}{1}{b}{9}{
La gráfica de una función puede tener dos valores de $y$ para el mismo $x$.
\begin{enumerate}
\item a) Verdadero
\item b) Falso
\end{enumerate}
}
\end{question}

\begin{question}{615}{funciones}{1}{a}{9}{
Si una relación pasa la prueba de la recta vertical, es una función.
\begin{enumerate}
\item a) Verdadero
\item b) Falso
\end{enumerate}
}
\end{question}

\begin{question}{616}{funciones}{1}{a}{9}{
La función constante $f(x)=2$ es una función bien definida de $\mathbb{R}$ en $\mathbb{R}$.
\begin{enumerate}
\item a) Verdadero
\item b) Falso
\end{enumerate}
}
\end{question}

\begin{question}{617}{funciones}{1}{b}{9}{
El dominio de $f(x)=\frac{1}{x-2}$ es $\mathbb{R}$.
\begin{enumerate}
\item a) Verdadero
\item b) Falso
\end{enumerate}
}
\end{question}

\begin{question}{618}{funciones}{1}{a}{9}{
El codominio y la imagen de una función son siempre iguales.
\begin{enumerate}
\item a) Verdadero
\item b) Falso
\end{enumerate}
}
\end{question}

\begin{question}{619}{funciones}{1}{a}{9}{
Si $f(x)=|x|$, entonces $f(x)\geq0$ para todo $x\in\mathbb{R}$.
\begin{enumerate}
\item a) Verdadero
\item b) Falso
\end{enumerate}
}
\end{question}

\begin{question}{620}{funciones}{1}{a}{9}{
Si $f$ y $g$ son funciones bien definidas, entonces $f+g$ siempre está bien definida.
\begin{enumerate}
\item a) Verdadero
\item b) Falso
\end{enumerate}
}
\end{question}

\begin{question}{621}{funciones}{1}{a}{9}{
La función $f(x)=x^2$ asigna un único valor a cada número real.
\begin{enumerate}
\item a) Verdadero
\item b) Falso
\end{enumerate}
}
\end{question}

\begin{question}{622}{funciones}{1}{a}{9}{
Si $f(x)=3x-2$, entonces $f(0)=-2$.
\begin{enumerate}
\item a) Verdadero
\item b) Falso
\end{enumerate}
}
\end{question}

\begin{question}{623}{funciones}{1}{a}{9}{
El conjunto de valores que toma una función se llama imagen.
\begin{enumerate}
\item a) Verdadero
\item b) Falso
\end{enumerate}
}
\end{question}

\begin{question}{624}{funciones}{1}{a}{9}{
Una función puede definirse mediante una tabla, una fórmula o una descripción verbal.
\begin{enumerate}
\item a) Verdadero
\item b) Falso
\end{enumerate}
}
\end{question}


%%%%%%%%%%%%%%%%%%%%%%
%%%% BLOQUE 2: INYECTIVIDAD (625–649)
%%%%%%%%%%%%%%%%%%%%%%

\begin{question}{625}{funciones}{1}{a}{10}{
Una función $f$ es inyectiva si distintos elementos del dominio tienen imágenes distintas.
\begin{enumerate}
\item a) Verdadero
\item b) Falso
\end{enumerate}
}
\end{question}

\begin{question}{626}{funciones}{1}{b}{10}{
La función $f(x)=x^2$ es inyectiva en $\mathbb{R}$.
\begin{enumerate}
\item a) Verdadero
\item b) Falso
\end{enumerate}
}
\end{question}

\begin{question}{627}{funciones}{1}{a}{10}{
La función $f(x)=x^3$ es inyectiva en $\mathbb{R}$.
\begin{enumerate}
\item a) Verdadero
\item b) Falso
\end{enumerate}
}
\end{question}

\begin{question}{628}{funciones}{1}{b}{10}{
$f(x)=|x|$ es inyectiva en $\mathbb{R}$.
\begin{enumerate}
\item a) Verdadero
\item b) Falso
\end{enumerate}
}
\end{question}

\begin{question}{629}{funciones}{1}{a}{10}{
Toda función lineal $f(x)=ax+b$ con $a\neq0$ es inyectiva.
\begin{enumerate}
\item a) Verdadero
\item b) Falso
\end{enumerate}
}
\end{question}

\begin{question}{630}{funciones}{2}{b}{10}{
$f(x)=\sin x$ es inyectiva en todo $\mathbb{R}$.
\begin{enumerate}
\item a) Verdadero
\item b) Falso
\end{enumerate}
}
\end{question}

\begin{question}{631}{funciones}{1}{a}{10}{
Si $f(x)=3x-1$, entonces $f$ es inyectiva en $\mathbb{R}$.
\begin{enumerate}
\item a) Verdadero
\item b) Falso
\end{enumerate}
}
\end{question}

\begin{question}{632}{funciones}{1}{b}{10}{
La función constante $f(x)=5$ es inyectiva.
\begin{enumerate}
\item a) Verdadero
\item b) Falso
\end{enumerate}
}
\end{question}

\begin{question}{633}{funciones}{1}{a}{10}{
Si $f(x)=2x+7$ y $f(a)=f(b)$, entonces $a=b$.
\begin{enumerate}
\item a) Verdadero
\item b) Falso
\end{enumerate}
}
\end{question}

\begin{question}{634}{funciones}{1}{a}{10}{
Si $f$ es inyectiva, entonces nunca puede tomar el mismo valor para dos elementos distintos del dominio.
\begin{enumerate}
\item a) Verdadero
\item b) Falso
\end{enumerate}
}
\end{question}

\begin{question}{635}{funciones}{1}{a}{10}{
$f(x)=x^2$ es inyectiva si el dominio se restringe a $x\geq0$.
\begin{enumerate}
\item a) Verdadero
\item b) Falso
\end{enumerate}
}
\end{question}

\begin{question}{636}{funciones}{1}{a}{10}{
Si $f$ y $g$ son inyectivas, entonces $g\circ f$ también es inyectiva.
\begin{enumerate}
\item a) Verdadero
\item b) Falso
\end{enumerate}
}
\end{question}

\begin{question}{637}{funciones}{1}{a}{10}{
Si $f$ es inyectiva, entonces existe una función $g$ tal que $g(f(x))=x$.
\begin{enumerate}
\item a) Verdadero
\item b) Falso
\end{enumerate}
}
\end{question}

\begin{question}{638}{funciones}{1}{a}{10}{
Si $f(x)=e^x$, entonces $f$ es inyectiva.
\begin{enumerate}
\item a) Verdadero
\item b) Falso
\end{enumerate}
}
\end{question}

\begin{question}{639}{funciones}{1}{a}{10}{
Si $f(x)=\tan x$, $f$ es inyectiva en $\left(-\frac{\pi}{2},\frac{\pi}{2}\right)$.
\begin{enumerate}
\item a) Verdadero
\item b) Falso
\end{enumerate}
}
\end{question}

\begin{question}{640}{funciones}{3}{a}{10}{
Una función creciente estricta en un intervalo es inyectiva en ese intervalo.
\begin{enumerate}
\item a) Verdadero
\item b) Falso
\end{enumerate}
}
\end{question}
%%%%%%%%%%%%%%%%%%%%%%
%%%% 640–649: PARIDAD (FUNCIONES PARES E IMPARES)
%%%%%%%%%%%%%%%%%%%%%%

\begin{question}{641}{funciones}{1}{c}{10}{
Seleccione la opción correcta. Una función $f$ es par si:
\begin{enumerate}
\item a) $f(x+y)=f(x)+f(y)$
\item b) $f(-x)=-f(x)$
\item c) $f(-x)=f(x)$
\item d) $f(x)=x$
\end{enumerate}
}
\end{question}

\begin{question}{642}{funciones}{1}{b}{10}{
La función $f(x)=\sin(x)$ es:
\begin{enumerate}
\item a) Par
\item b) Impar
\item c) Ninguna de las dos
\item d) Constante
\end{enumerate}
}
\end{question}

\begin{question}{643}{funciones}{1}{a}{10}{
Determine si la función $f(x)=x^4+1$ es:
\begin{enumerate}
\item a) Par
\item b) Impar
\item c) Ni par ni impar
\item d) Periódica
\end{enumerate}
}
\end{question}

\begin{question}{644}{funciones}{1}{b}{10}{
La función $f(x)=x^3+x$ es:
\begin{enumerate}
\item a) Par
\item b) Impar
\item c) Periódica
\item d) Constante
\end{enumerate}
}
\end{question}

\begin{question}{645}{funciones}{1}{b}{10}{
Seleccione la afirmación correcta respecto a la simetría de las funciones pares.
\begin{enumerate}
\item a) Son simétricas respecto al eje $x$.
\item b) Son simétricas respecto al eje $y$.
\item c) Son simétricas respecto al origen.
\item d) No tienen simetría.
\end{enumerate}
}
\end{question}

\begin{question}{646}{funciones}{1}{b}{10}{
Si $f$ es impar, entonces para todo $x$ se cumple:
\begin{enumerate}
\item a) $f(-x)=f(x)$
\item b) $f(-x)=-f(x)$
\item c) $f(x)=-x$
\item d) $f(x)=x$
\end{enumerate}
}
\end{question}

\begin{question}{647}{funciones}{1}{a}{10}{
La función $f(x)=\cos(x)$ es:
\begin{enumerate}
\item a) Par
\item b) Impar
\item c) Ni par ni impar
\item d) Ninguna de las anteriores
\end{enumerate}
}
\end{question}

\begin{question}{648}{funciones}{1}{a}{10}{
Una función puede ser simultáneamente par e impar si y solo si:
\begin{enumerate}
\item a) Es constante nula ($f(x)=0$)
\item b) Es lineal
\item c) Es cuadrática
\item d) Es periódica
\end{enumerate}
}
\end{question}

\begin{question}{649}{funciones}{1}{a}{10}{
Determine cuál de las siguientes funciones no es par ni impar.
\begin{enumerate}
\item a) $f(x)=x^3$
\item b) $f(x)=x^2$
\item c) $f(x)=x^2+1$
\item d) $f(x)=1$
\end{enumerate}
}
\end{question}

%%%%%%%%%%%%%%%%%%%%%%
%%%% 650–659: MONOTONÍA
%%%%%%%%%%%%%%%%%%%%%%

\begin{question}{650}{funciones}{1}{a}{10}{
Una función $f$ es estrictamente creciente si:
\begin{enumerate}
\item a) $x_1<x_2$ implica $f(x_1)<f(x_2)$
\item b) $x_1<x_2$ implica $f(x_1)>f(x_2)$
\item c) $f(x_1)=f(x_2)$
\item d) $f(x)$ es constante
\end{enumerate}
}
\end{question}

\begin{question}{651}{funciones}{1}{a}{10}{
Determine cuál de las siguientes funciones es decreciente en todo $\mathbb{R}$. Con definicion de decrecienete que si $x_1<x_2$ ocurre que $f(x_2)<f(x_1)$
\begin{enumerate}
\item a) $f(x)=-x$
\item b) $f(x)=x^3$
\item c) $f(x)=x^2$
\item d) $f(x)=|x|$
\end{enumerate}
}
\end{question}

\begin{question}{652}{funciones}{1}{a}{10}{
Si $f(x)=3x+1$, entonces:
\begin{enumerate}
\item a) Es creciente
\item b) Es decreciente
\item c) Es constante
\item d) No es monótona
\end{enumerate}
}
\end{question}

\begin{question}{653}{funciones}{1}{a}{10}{
Seleccione la opción correcta. Una función $f$ se dice acotada superiormente si:
\begin{enumerate}
\item a) Existe un número $M$ tal que $f(x)\leq M$ para todo $x$ en su dominio.
\item b) Existe un número $M$ tal que $f(x)\geq M$ para todo $x$ en su dominio.
\item c) $f(x)$ no tiene límite cuando $x$ crece.
\item d) $f(x)$ toma infinitos valores negativos.
\end{enumerate}
}
\end{question}


\begin{question}{654}{funciones}{2}{a}{10}{
$f(x)=e^x$ es:
\begin{enumerate}
\item a) Creciente en todo $\mathbb{R}$
\item b) Decreciente en todo $\mathbb{R}$
\item c) Constante
\item d) Ninguna
\end{enumerate}
}
\end{question}

\begin{question}{655}{funciones}{1}{a}{10}{
Determine el intervalo donde $f(x)=x^2$ es creciente.
\begin{enumerate}
\item a) $(-\infty,0)$
\item b) $(0,\infty)$
\item c) $(0,1)$
\item d) Todo $\mathbb{R}$
\end{enumerate}
}
\end{question}

\begin{question}{656}{funciones}{2}{a}{10}{
La función $f(x)=\ln(x)$ es:
\begin{enumerate}
\item a) Creciente en $(0,\infty)$
\item b) Decreciente en $(0,\infty)$
\item c) Par
\item d) Constante
\end{enumerate}
}
\end{question}

\begin{question}{657}{funciones}{1}{a}{10}{
Seleccione la opción correcta sobre funciones periódicas.
\begin{enumerate}
\item a) Una función es periódica si existe $T>0$ tal que $f(x+T)=f(x)$ para todo $x$.
\item b) Una función es periódica si $f(x+T)=-f(x)$ para todo $x$.
\item c) Una función periódica solo puede tomar valores positivos.
\item d) Ninguna función continua es periódica.
\end{enumerate}
}
\end{question}


\begin{question}{658}{funciones}{2}{a}{10}{
Una función es monótona si:
\begin{enumerate}
\item a) Es creciente o decreciente en todo su dominio.
\item b) Tiene derivada igual a cero.
\item c) Es par.
\item d) Tiene periodo finito.
\end{enumerate}
}
\end{question}

\begin{question}{659}{funciones}{1}{a}{10}{
La función $f(x)=-x^3$ es:
\begin{enumerate}
\item a) Creciente en todo $\mathbb{R}$
\item b) Decreciente en todo $\mathbb{R}$
\item c) Par
\item d) Constante
\end{enumerate}
}
\end{question}

%%%%%%%%%%%%%%%%%%%%%%
%%%% 660–680: COMPOSICIÓN Y PROPIEDADES FUNCIONALES
%%%%%%%%%%%%%%%%%%%%%%

\begin{question}{660}{funciones}{1}{a}{10}{
Si $f(x)=2x$ y $g(x)=x+3$, entonces $(g\circ f)(x)$ es:
\begin{enumerate}
\item a) $2x+3$
\item b) $2x+6$
\item c) $x+5$
\item d) $x^2+3$
\end{enumerate}
}
\end{question}

\begin{question}{661}{funciones}{1}{a}{10}{
Si $f(x)=x+1$ y $g(x)=x-1$, entonces $(f\circ g)(x)$ es:
\begin{enumerate}
\item a) $x$
\item b) $x+2$
\item c) $x-2$
\item d) $x+1$
\end{enumerate}
}
\end{question}

\begin{question}{662}{funciones}{1}{a}{10}{
Seleccione la opción correcta. La composición de funciones está definida si:
\begin{enumerate}
\item a) El codominio de la primera coincide con el dominio de la segunda.
\item b) Los dominios de ambas funciones son iguales.
\item c) Ambas funciones son crecientes.
\item d) Las funciones son lineales.
\end{enumerate}
}
\end{question}

\begin{question}{663}{funciones}{1}{a}{10}{
Si $f(x)=x^2$ y $g(x)=x+1$, determine $(f\circ g)(x)$.
\begin{enumerate}
\item a) $(x+1)^2$
\item b) $x^2+1$
\item c) $x^2+x+1$
\item d) $2x+1$
\end{enumerate}
}
\end{question}

\begin{question}{664}{funciones}{1}{a}{10}{
La función identidad $\operatorname{id}(x)$ se caracteriza porque:
\begin{enumerate}
\item a) Satisface $\operatorname{id}(x)=x$ para todo $x$.
\item b) Es igual a cero en todo su dominio.
\item c) Cumple $\operatorname{id}(x)=1/x$.
\item d) No está definida en $\mathbb{R}$.
\end{enumerate}
}
\end{question}

\begin{question}{665}{funciones}{1}{a}{10}{
Si $f(x)=x+2$ y $g(x)=x-2$, entonces $(f\circ g)(x)$ y $(g\circ f)(x)$ son:
\begin{enumerate}
\item a) Iguales entre sí e iguales a $x$.
\item b) Diferentes, pero ambas son funciones identidad.
\item c) Iguales y valen $x+4$.
\item d) Diferentes: una es $x$ y la otra $x+4$.
\end{enumerate}
}
\end{question}

\begin{question}{666}{funciones}{1}{a}{10}{
Si $f$ y $g$ son funciones tales que $f(g(x))=x$ para todo $x$, se dice que:
\begin{enumerate}
\item a) $f$ es la inversa de $g$.
\item b) $f$ es par.
\item c) $f$ es constante.
\item d) $f$ es idénticamente nula.
\end{enumerate}
}
\end{question}

\begin{question}{667}{funciones}{1}{a}{10}{
Seleccione la afirmación correcta sobre la igualdad de funciones.
\begin{enumerate}
\item a) Dos funciones son iguales si tienen el mismo dominio y asignan el mismo valor a cada elemento.
\item b) Dos funciones son iguales si tienen la misma fórmula algebraica.
\item c) Dos funciones son iguales si tienen el mismo codominio.
\item d) Dos funciones son iguales si coinciden en un punto.
\end{enumerate}
}
\end{question}

\begin{question}{668}{funciones}{1}{a}{10}{
Determine el dominio de la función $f(x)=\frac{1}{x-3}$.
\begin{enumerate}
\item a) $\mathbb{R}\setminus\{3\}$
\item b) $\mathbb{R}$
\item c) $\{3\}$
\item d) $\mathbb{R}^+$
\end{enumerate}
}
\end{question}

\begin{question}{669}{funciones}{1}{a}{10}{
Si $f(x)=|x|$, entonces la imagen de $f$ es:
\begin{enumerate}
\item a) $[0,\infty)$
\item b) $\mathbb{R}$
\item c) $(-\infty,0]$
\item d) $\{0,1\}$
\end{enumerate}
}
\end{question}

\begin{question}{670}{funciones}{1}{a}{10}{
Una función constante tiene la siguiente propiedad:
\begin{enumerate}
\item a) Todos los elementos del dominio tienen la misma imagen.
\item b) Cada elemento del dominio tiene una imagen distinta.
\item c) Su gráfica pasa por el origen.
\item d) No tiene dominio.
\end{enumerate}
}
\end{question}

\begin{question}{671}{funciones}{1}{a}{10}{
Si $f(x)=x^2$ y $g(x)=-x^2$, entonces:
\begin{enumerate}
\item a) Tienen el mismo dominio pero imágenes distintas.
\item b) Tienen imágenes iguales y dominios distintos.
\item c) Son idénticas.
\item d) No están bien definidas.
\end{enumerate}
}
\end{question}

\begin{question}{672}{funciones}{1}{a}{10}{
Seleccione la opción correcta. Una función definida por partes:
\begin{enumerate}
\item a) Asigna un valor distinto a $x$ según la condición que cumpla.
\item b) No está bien definida.
\item c) Solo puede ser lineal.
\item d) Tiene dominio vacío.
\end{enumerate}
}
\end{question}

\begin{question}{673}{funciones}{1}{a}{10}{
Si $f(x)=x+1$ para $x\ge0$ y $f(x)=-x$ para $x<0$, entonces $f(2)$ es:
\begin{enumerate}
\item a) $3$
\item b) $-2$
\item c) $1$
\item d) $0$
\end{enumerate}
}
\end{question}

\begin{question}{674}{funciones}{1}{a}{10}{
Si $f(x)=x+1$ y $g(x)=2x$, entonces $(f+g)(x)$ es:
\begin{enumerate}
\item a) $3x+1$
\item b) $x^2+1$
\item c) $2x+1$
\item d) $x+2$
\end{enumerate}
}
\end{question}

\begin{question}{675}{funciones}{1}{a}{10}{
Si $f(x)=x^2$ y $g(x)=3x$, entonces $(f-g)(x)$ es:
\begin{enumerate}
\item a) $x^2-3x$
\item b) $3x-x^2$
\item c) $x^2+3x$
\item d) $3x^2-x$
\end{enumerate}
}
\end{question}

\begin{question}{676}{funciones}{2}{a}{10}{
Seleccione la opción correcta. La gráfica de una función $f$:
\begin{enumerate}
\item a) Es el conjunto de pares ordenados $(x,f(x))$.
\item b) Está formada solo por los valores de $x$.
\item c) Coincide con el dominio.
\item d) Es siempre una línea recta.
\end{enumerate}
}
\end{question}

\begin{question}{677}{funciones}{1}{a}{10}{
Si $f$ es una función tal que $f(x)=x$ para todo $x$ del dominio, entonces:
\begin{enumerate}
\item a) Es la función identidad.
\item b) Es la función constante.
\item c) Es la función nula.
\item d) No está bien definida.
\end{enumerate}
}
\end{question}

\begin{question}{678}{funciones}{1}{a}{10}{
Dos funciones $f$ y $g$ tienen la misma imagen si:
\begin{enumerate}
\item a) Toman los mismos valores, aunque sus reglas sean distintas.
\item b) Son idénticas.
\item c) Tienen distinto dominio.
\item d) Coinciden solo en un punto.
\end{enumerate}
}
\end{question}

\begin{question}{679}{funciones}{1}{a}{10}{
Si $f(x)=x^2$ y $g(x)=|x|$, entonces $f$ y $g$:
\begin{enumerate}
\item a) Tienen la misma imagen.
\item b) Tienen dominios distintos.
\item c) No son funciones.
\item d) Tienen diferentes imágenes.
\end{enumerate}
}
\end{question}

\begin{question}{680}{funciones}{1}{a}{10}{
Seleccione la opción correcta. Una función parcial:
\begin{enumerate}
\item a) No está definida para todos los elementos del conjunto inicial.
\item b) Siempre está definida en todo $\mathbb{R}$.
\item c) Asigna más de una imagen a cada elemento.
\item d) No tiene codominio.
\end{enumerate}
}
\end{question}
%%%%%%%%%%%%%%%%%%%%%%
%%%% 681–700: PROPIEDADES FUNCIONALES DIVERSAS
%%%%%%%%%%%%%%%%%%%%%%

\begin{question}{681}{funciones}{1}{a}{10}{
Si $f(x)=x+1$ y $g(x)=2x$, entonces $(g\circ f)(x)$ es:
\begin{enumerate}
\item a) $2x+2$
\item b) $2x+1$
\item c) $x+3$
\item d) $x^2+1$
\end{enumerate}
}
\end{question}

\begin{question}{682}{funciones}{1}{a}{10}{
Seleccione la opción correcta. En general, la composición de funciones:
\begin{enumerate}
\item a) No es conmutativa.
\item b) Siempre es conmutativa.
\item c) Solo está definida si las funciones son lineales.
\item d) Siempre da como resultado la función identidad.
\end{enumerate}
}
\end{question}

\begin{question}{683}{funciones}{1}{a}{10}{
Si $f(x)=x^2$ y $g(x)=x+2$, entonces $(g\circ f)(x)$ es:
\begin{enumerate}
\item a) $x^2+2$
\item b) $(x+2)^2$
\item c) $x+4$
\item d) $x^2+4x+4$
\end{enumerate}
}
\end{question}

\begin{question}{684}{funciones}{1}{a}{10}{
Seleccione la afirmación verdadera sobre la composición de funciones.
\begin{enumerate}
\item a) $(f\circ g)(x)=f(g(x))$
\item b) $(f\circ g)(x)=g(f(x))$
\item c) $(f\circ g)(x)=f(x)\cdot g(x)$
\item d) $(f\circ g)(x)=f(x)+g(x)$
\end{enumerate}
}
\end{question}

\begin{question}{685}{funciones}{1}{a}{10}{
Si $f(x)=2x+1$ y $g(x)=x-1$, entonces $(f\circ g)(x)$ es:
\begin{enumerate}
\item a) $2x-1$
\item b) $2x+1$
\item c) $2x-3$
\item d) $x^2-1$
\end{enumerate}
}
\end{question}

\begin{question}{686}{funciones}{1}{a}{10}{
Seleccione la opción correcta sobre la función inversa.
\begin{enumerate}
\item a) Si $f$ tiene inversa, entonces $f^{-1}(f(x))=x$.
\item b) La inversa de una función siempre es idéntica a la función original.
\item c) Solo las funciones constantes tienen inversa.
\item d) Las funciones pares siempre son inversibles.
\end{enumerate}
}
\end{question}

\begin{question}{687}{funciones}{1}{a}{10}{
Si $f(x)=3x+2$, la inversa de $f$ es:
\begin{enumerate}
\item a) $f^{-1}(x)=\frac{x-2}{3}$
\item b) $f^{-1}(x)=3x-2$
\item c) $f^{-1}(x)=\frac{x+2}{3}$
\item d) $f^{-1}(x)=x-3$
\end{enumerate}
}
\end{question}

\begin{question}{688}{funciones}{1}{a}{10}{
Seleccione la opción correcta sobre la función identidad.
\begin{enumerate}
\item a) La identidad compuesta con cualquier función devuelve la función original.
\item b) La identidad no se puede componer con otras funciones.
\item c) La identidad cambia el dominio de la función.
\item d) La identidad solo se define en $\mathbb{N}$.
\end{enumerate}
}
\end{question}

\begin{question}{689}{funciones}{1}{a}{10}{
Si $f$ es la función identidad y $g(x)=x+2$, entonces $(g\circ f)(x)$ es:
\begin{enumerate}
\item a) $x+2$
\item b) $x$
\item c) $2x$
\item d) $x-2$
\end{enumerate}
}
\end{question}

\begin{question}{690}{funciones}{1}{a}{10}{
Si $f(x)=x+2$ y $g(x)=x-2$, entonces $(f\circ g)(x)$ es igual a:
\begin{enumerate}
\item a) $x$
\item b) $x+4$
\item c) $x-4$
\item d) $x^2$
\end{enumerate}
}
\end{question}

\begin{question}{691}{funciones}{1}{a}{10}{
Seleccione la opción correcta. Una función $f$ es involutiva si:
\begin{enumerate}
\item a) $f(f(x))=x$
\item b) $f(x)=x$
\item c) $f(-x)=-f(x)$
\item d) $f(x)=f(y)$ para todo $x,y$
\end{enumerate}
}
\end{question}

\begin{question}{692}{funciones}{1}{a}{10}{
Determine cuál de las siguientes funciones es involutiva.
\begin{enumerate}
\item a) $f(x)=-x$
\item b) $f(x)=x+1$
\item c) $f(x)=x^2$
\item d) $f(x)=1/x$
\end{enumerate}
}
\end{question}

\begin{question}{693}{funciones}{1}{a}{10}{
Si $f(x)=1/x$, entonces $f(f(x))$ es:
\begin{enumerate}
\item a) $x$
\item b) $1/x$
\item c) $-x$
\item d) $x^2$
\end{enumerate}
}
\end{question}

\begin{question}{694}{funciones}{1}{a}{10}{
Seleccione la afirmación verdadera sobre la composición con la inversa.
\begin{enumerate}
\item a) $(f^{-1}\circ f)(x)=x$
\item b) $(f\circ f^{-1})(x)=f^{-1}(x)$
\item c) $(f^{-1}\circ f)(x)=f(x)$
\item d) $(f\circ f^{-1})(x)=x^2$
\end{enumerate}
}
\end{question}

\begin{question}{695}{funciones}{1}{a}{10}{
Si $f(x)=x+1$ y $g(x)=x-1$, entonces $f$ y $g$ son:
\begin{enumerate}
\item a) Funciones inversas entre sí.
\item b) Funciones idénticas.
\item c) Funciones constantes.
\item d) Funciones sin dominio común.
\end{enumerate}
}
\end{question}

\begin{question}{696}{funciones}{1}{a}{10}{
Seleccione la afirmación correcta. Dos funciones son inversas si:
\begin{enumerate}
\item a) Compuestas en cualquier orden devuelven la identidad.
\item b) Son iguales en todo punto.
\item c) Son ambas constantes.
\item d) Tienen el mismo dominio pero distinto codominio.
\end{enumerate}
}
\end{question}

\begin{question}{697}{funciones}{1}{a}{10}{
Si $f(x)=x+5$ y $g(x)=x-5$, entonces $(g\circ f)(x)$ es:
\begin{enumerate}
\item a) $x$
\item b) $x+10$
\item c) $x-10$
\item d) $x^2$
\end{enumerate}
}
\end{question}

\begin{question}{698}{funciones}{1}{a}{10}{
Seleccione la opción correcta. La imagen de la función $f(x)=x^2$ es:
\begin{enumerate}
\item a) $[0,\infty)$
\item b) $\mathbb{R}$
\item c) $(-\infty,0]$
\item d) $\{0,1\}$
\end{enumerate}
}
\end{question}

\begin{question}{699}{funciones}{1}{a}{10}{
Si dos funciones $f$ y $g$ tienen la misma imagen y el mismo dominio, entonces:
\begin{enumerate}
\item a) Son iguales.
\item b) Son distintas.
\item c) Una de ellas no está bien definida.
\item d) No se pueden comparar.
\end{enumerate}
}
\end{question}
